\chapter{Geometry-based End to End visual odometry}
Visual odoemtry is an important method for localization and ego-motion estimation for mobile robots. Learning-based solutions for visual odometry become more and more popular as it reduce the efforts for sensor calibration and parameter tuning. However, the current performance of learning-based method is not good. We argue that one reason should be that most current learning-based methods utilize multiple  convolution layers to model the mapping from image-pair to ego-motion, however convolution layers are designed for modeling feature extraction and description, not for modeling geometry calculation. We proposed to model the ego-motion mapping with a new structure which contains new kind of layer in order to improve the performance. 

The structure of proposed ego-motion estimation network is shown in Fig. \ref{}. Taking the geometry-based method in to consideration, we can know that, the solution for visual odometry should contains three parts: feature extraction, feature matching and ego-motion calculation. We imitate that structure and keep the three parts in our designed structure. Convolution layers are used for feature extraction; we proposed a matching layer to imitate the feature matching process; ego-motion calculation model is realized by fully connected layers.

The main contribution of this work are: 
\begin{itemize}

\item We proposed a new neural network structure for ego-motion estimation. In this new structure, we utilize a novel matching layer. 

\item We test our proposed method in Kitti dataset, TUM dataset and our collected dataset, and found that( the performance overcome the state of the art.)

\item We test the training model on new unknown datasets  and found that our method is with good generalization ability. 
\end{itemize}

\section{Geometry-based Ego-motion Estimation Model}
Imitating geometry-based ego-motion estimation method, we separate the ego-motion estimation model into three parts: feature extraction, feature matching and ego-motion estimation. They are trained end to end. We descried our proposed method in case which it is trained supervised, but the proposed method can also be trained unsupervised. 

\subsection{Model Structure}
We utilize convolution layers for feature extraction.
\begin{equation}
    \mathbf{F} =M_f \left( \mathbf{I}\right)
\end{equation}