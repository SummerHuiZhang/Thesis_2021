\chapter{单目视觉里程计尺度计算:从手动建模到自主学习}
\section{引言}
前一章中介绍了路面几何模型的单目视觉里程计绝对尺度计算方法,该方 法将相对稳定的路面几何模型作为先验信息,本章将介绍一种场景中稳定区域的自主建模方法。
\section{基于场景建模的尺度计算方法} 
本文提出了一种基于场景建模的尺度计算方法,本节中将依次介绍场景建模方法以及基于场景模型的尺度计算方法。
\subsection{场景建模} 
\subsubsection{场景模型表征}
绝对尺度计算直接依赖于场景中特征点的绝对深度,故本文使用像素深度 概率模型来表征机器人所在的场景。该方法将机器人视野纵横分为多个栅格区 域$G_{ij}$,并使用高斯分布建模每个栅格区域的深度模型.
\begin{equation}
D(G_{ij} \mid S^k) N(\mu_{ij}^{k}-\sigma_{ij}^{k})>0
\end{equation}
其中$\mu_{ij}^{k}$ 和$\sigma_{ij}^{k}$分别为场景$S_k$中栅格$G_{ij}$所服从的高斯分布中的均值和标准差。

\subsubsection{场景建模方法}
场景建模的训练集需要包扩在特定场景中拍摄的连续图像序列${\textbf{I}_t}$和相邻图片拍摄位置的绝对距离$l_{t-1}^t$。首先依据相邻帧
图像$\textbf{I}_{t-1}$和$\textbf{I}_t$进行特征检测匹配和相对运动估计以及相对深度估计,具体操作如下
在图像$\textbf{I}_{t-1}$过量提取特征点,得到初始特征点集合$\Omega_{\tilde{f}}$和$u_{f^{t}}$可计算出机器人的相对运动$\textbf{R}$ $\bar{\textbf{t}}$,由于单目的尺度歧义性,无法准确获取$\bar{\textbf{t}}$的绝对大小,但可以
根据此运动通过三角测量的方式获取每个像素点的像素深度$\bar{d}_f$
在获取特征点的相对深度后,本文对特征点不同区域的深度分布进行统计学建模。首先根据相邻图片的绝对距离$l_{t-1}^t$,确定绝对尺度与相对尺度之前的系数
\begin{equation}
    s = \frac{l_{t-1}^t}{||\bar{\textbf{t}}_{t-1}^t||}
\end{equation}
进而可以获取,每个像素点在真实世界中的绝对深度$d =\textit{s}\bar{d}$。同时根据像素位置,将特征点归属于不同的栅格$G_{ij}$。在执行完场景中全部帧之后,使用高斯分布建模每个栅格内的特征点深度均值$\mu_{ij}$和标准差$\sigma_{ij}$
考虑到训练数据中可能存在多种分布不同的场景,简单的将所有场景用统一中模型表征会是偏差较大。本文提出使用聚类方法,将场景按照深度分为$K$个场景,对于每一个单独进行建模。场景聚类方法简述如下:首先我们将每一帧图像$\textbf{I}^t$按照特征点的分布位置和深度进行编码表示为$C_t \in \mathbb{R}^{h×w×2}$
\begin{equation}
    C_t[i_w,i_h] = (N(f),\mu(d))
\end{equation}
其中$N(f)$为特征点数量,$μ(d)$为特征点深度的均值。即我们使用各个栅格的特征
点数量和深度均值作为编码基本单元来表征每帧图像的结构信息(查查参考文献)。我们定义图像结构的距离为
\begin{equation}
    ||C^{t1}-C^{t2}|| = \sum(|N(f)^{t1}-N(f)^{t2}|+N(f)^{t1}N(f)^{t2}(\mu(d)^{t1}-\mu(d)^t2))    
\end{equation}
依据如上编码方式和编码距离的定义,可以完成聚类。
\subsection{尺度计算}
获取场景绝对深度模型之后,使用绝对深度与相对深度的比值即可计算尺度系数,在计算过程中,首先根据相对深度计算场景结构编码,
然后选取与当前结构最近的场景用于计算尺度
\begin{equation}
    S_k = argmin S_k
\end{equation}
在计算过程中,以栅格深度的均值作为该位置区域的绝对深度,通过栅格深度的标准差评价改绝对深度的可靠性,并根据如下公式计算尺度的最优值
\begin{equation}
    s = \frac{\sum f_i(\frac{1}{\sigma_{fi}^2})\frac{u_i}{d_{f_i}}}{\sum f_i(\frac{1}{\sigma_{f_i}^2})}
\end{equation}
获取尺度系数之后,即可恢复机器人相对运动的绝对尺度。

\subsection{基于场景建模的尺度计算验证实验} 
\subsubsection{模型参数测试实验}
\subsubsection{算法有效性消冗实验}
\subsubsection{与其它方法性能对比}
\section{本章小结}