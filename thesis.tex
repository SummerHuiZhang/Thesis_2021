\documentclass[degree=doctor,bibtype=numeric]{tongjithesis}
%   degree=[master|doctor], 							% 必选
%   bibtype=[numeric|authoryear], 						% 可选,数字式引用|作者-年份引用,默认为数字式(上标)引用
%   degreetype=[academic|profession|equaleducation],  	% 可选, 学术型|专业型|同等学力,默认为学术型
% 	electronic,                                 		% 可选, 电子版,(打印时删除)
%   secret,                                     		% 可选,是否保密,基本不用
%   pifootnote,                                 		% 可选,默认已打开
%   romantitle                                  		% 可选,默认已打开
%   注:默认已打开的选项可以使用arialtitle=false的形式关闭。

% 所有其它可能用到的包都统一放到这里了,可以根据自己的实际添加或者删除。
\usepackage{tongjiutils}
\usepackage[ruled]{algorithm2e}
\usepackage{caption}
\usepackage{subcaption}
\usepackage{spreadtab}
\usepackage{array}
\usepackage{graphicx}
\usepackage{verbatim}

\DeclareMathOperator*{\argmax}{argmax}
\DeclareMathOperator*{\argmin}{argmin}
\newcommand{\mysubfigure}[4][name]{\begin{subfigure}[b]{#2\textwidth}#3\caption{#1} #4 \end{subfigure}}
%参考文献更新使用biblatex包, 使用gb7714-2015标准, 具体参数设置可在cls文件中搜索biblatex进行了解
%加入bib文件(老版本文件依然能够使用)
\addbibresource{ref/refs.bib}   %
% 定义所有的eps文件在 figures 子目录下
\graphicspath{{figures/}}

\begin{document}

%%% 封面部分
\frontmatter

\tongjisetup{
  %******************************
  % 注意:
  %   1. 配置里面不要出现空行
  %   2. 不需要的配置信息可以删除
  %******************************
  %
  %=====
  % 秘级
  %=====
  secretlevel={保密},
  secretyear={2},
  %
  %=========
  % 中文信息
  %=========
  % 题目过长可以换行(推荐手动加入换行符,这样就可以控制换行的地方啦)。
  ctitle={机器人视觉定位},
  cheadingtitle={单目视觉里程计的尺度计算与映射学习},    %用于页眉的标题,不要换行
  cauthor={张会},  
  studentnumber={1510456},
  cmajorfirst={工学},
  cmajorsecond={控制科学与工程系},
  cdepartment={电子与信息工程学院},
  csupervisor={陈启军 教授}, 
  % 如果没有副指导老师或者校外指导老师,把{}中内容留空即可,或者直接注释掉。
  %cassosupervisor={Sebastian Scherer (校外)}, % 副指导老师
  % 日期自动使用当前时间,若需手动指定,按如下方式修改:
  % cdate={\zhdigits{2018}年\zhnumber{11}月},
  % 没有基金的话就注释掉吧。
  %cfunds={(本文受中国自然科学基金重点项目(项目编号:)和国家受留学基金委联合培养博士生项目(学号:201706260065)支持)},
  %
  %=========
  % 英文信息
  %=========
  etitle={Robots Monocular Visual Localization}, 
  eauthor={Hui Zhang},
  emajorfirst={Engineering},
  emajorsecond={Control Science and Engineering},
  edepartment={School of Electronic and Information Engineering},
  % 日期自动使用当前时间,若需手动指定,按如下方式修改:
  % edate={November,\ 2018},
  %efunds={(Supported by the Natural Science Foundation of China (Grant No.) and Joint Ph.D Student Program of China Scholarshhip Council(Student No. 201706260065))},    
  esupervisor={Prof. Qijun Chen},
  %eassosupervisor={)}
  }
% 定义中英文摘要和关键字
\begin{cabstract}  
%移动机器人
%定位与单目视觉定位

%单目视觉状态估计存在的问题 %本文的贡献
本文主要贡献和创新点为:

\end{cabstract}
\ckeywords{单目相机,视觉定位,绝对定位,增量定位,卷积神经网络,几何约束}
\begin{eabstract}
The main contributions and innovations of this article are:

\end{eabstract}
\ekeywords{Visual Localization, State Estimation, Graph Optimization, Probabilistic Graphical Model, Convolutional Nerual Network, Homomorphism,Patch Agreement}
\makecover


% 目录
\tableofcontents
% 符号对照表
\begin{denotation}
\item[$\mathbb{R}$] 实数集合
\item[$\mathbb{I}$] 0-255闭区间内所有整数集合 
\item[$\text{SO}(3)$] 3维空间特殊正交群
\item[$\text{SE}(3)$] 3维空间特殊欧式群 
\item[$\mathbf{R}$] 旋转矩阵 $\mathbf{R} \in \text{SO}(3)$
\item[$\mathbf{R}_{ij}$] 矩阵$\mathbf{R}$的第$i$行第$j$列的元素
\item[$\mathbf{r}$] 旋转向量 
\item[$\mathbf{t}$] 平移向量
\item[$\mathbf{t}_i$]  向量 $\mathbf{t}$的第$i$个元素
\item[$\mathbf{\bar{t}}$] 相对尺度下的平移向量
\item[$\mathbf{[t]}_\times$] 斜对称矩阵 $\mathbf{[t]}_\times = \begin{pmatrix}
    0  & -t_3& t_2 \\
    t_3& 0   & -t_1\\
    t_1& -t_2& 0   \\
\end{pmatrix}$
\item[$\mathbf{T}$] 运动矩阵 $\mathbf{T} = \begin{pmatrix}
    \mathbf{R}&  \mathbf{t}\\
    0&  1\\
\end{pmatrix} \in \text{SE3}$  
\item[$\mathbf{T}_{t}^{t+1}$] $t$时刻到$t+1$时刻机器人的运动矩阵
\item[$\mathbf{\tau}$] 运动向量 $\mathbf{\tau} = \log(\mathbf{T}) \in \text{se3} $
\item[$\mathbf{\underline{\tau}}$] 训练集中$\tau$的真值
\item[$\log(\mathbf{R})$] 旋转矩阵的对数映射(结果为旋转运动的角轴表示)
\item[$\exp(\mathbf{r})$] 旋转向量反对称矩阵的指数映射 
\item[$\mathbf{I}^t$] 第t帧图像
\item[$\mathbf{D}^t$] 第t帧图像对应的深度图
\item[$\Omega_{\mathbf{I}}$] 图像集合
\item[$\mathbf{I}_m^n$]  图像和图像组成的图像对
\item[$\Omega_{\mathbf{I}_m^n}$] 图像对集合
\item[$\otimes$] 定义在图像对集合上的二元运算,$\mathbf{I}_m^t \otimes \mathbf{I}_t^n= \mathbf{I}_m^n$
\item[$^p\mathbf{I}^t$] 第t帧图像的第p块子区域
\item[$^p\mathbf{I}_m^n$] 图像对的第p块子区域
\item[$\Omega_f$]  特征点集合$\Omega_f=\{f_0,f_1,...\}$
\item[$\Omega_{\mathbf{I}^t}$]  图像$\mathbf{I}^t$中的像素点集合 
\item[$f^t_i$] 第t帧图像的第i个特征点
\item[$\check{f}$] 特征点$f$属于路面区域
\item[$\hat{f}$] 特征点$f$不属于路面区域
\item[$\mathbf{n}_r$] 路面所在平面在相机坐标系下的法向  
\item[$\mathbf{u}^t_i$] 第t帧第i个像素点的2维像素坐标 $\mathbf{u}_i=(u_i,v_i)$
\item[$\mathbf{u}_i$] 第i个像素点的2维像素坐标$\mathbf{u}_i=(u_i,v_i)$ 
\item[$\vec{\mathbf{u}}_i$] 第i个像素点的扩充像素坐标$\vec{\mathbf{u}}_i=(u_i,v_i,1)$ 
\item[$\mathbf{x}_i$] 第i个像素点的3维空间坐标$\mathbf{x}_i=(x_i,y_i,z_i)$
\item[$d_i$] 第i个像素点的深度
\item[$I_p$] 像素点$p$的像素值
\item[$\mathbf{u}_p$] 像素点$p$的像素位置
\item[$d_p$] 像素点$p$的深度值
\item[$\underline{d_p}$] 训练集中像素点$p$深度值的真值
\item[$\mathbf{x}_p$] 像素点$p$的空间坐标 
\item[$I^t_p$] 第t帧图像中像素点$p$的像素值
\item[$a \propto b $] $a$与$b$正相关
\item[$\mathbf{v}^\text{T}$] 向量的转置
\item[$\mathbf{M}^\text{T}$] 矩阵的转置 
\item[$\mu(\mathbf{x})$] $\mathbf{x}$的均值 
\item[$\sigma(\mathbf{x})$] $\mathbf{x}$的标准差
\item[$\sigma^2(\mathbf{x})$] $\mathbf{x}$的方差  
\item[$Q_{\frac{1}{2}}(\mathbf{x})$] $\mathbf{x}$的中位数
\end{denotation}

%%% 以下索引按需要选择
% 插图索引
%\listoffigures
% 表格索引
%\listoftables
% 公式索引
% \listofequations

%%% 正文 
\mainmatter
\chapter{引言}
\label{ch:introduction}
\section{机器人定位研究背景和意义}

本论文以移动机器人在已知和未知环境中基于单目相机的自我定位为研究背景,主要研究绝对式定位的深度特征提取与压缩、增量式定位之单目视觉里程计(VO)的绝对尺度运算及VO问题的求解框架简化。本章节将从国际人工智能发展背景、国内机器人行业发展需求、移动机器人的发展需求及研究热点等多个角度逐层剖析。

国内背景:中国制造2025\cite{2017Made},从第四代工业革命介绍人工智能的发展潮流,其中机器人尤其是无人驾驶的地位,简述机器人的研究难点,VO在其中的重要性,VO的难点,VO的关键技术
%定位的重要性与必要性: %早期人类如何定位的,对于人们的日常,以及航海、外太空探索等重要的研究项目,定位的重要性:
人工智能技术的出现,为人们生活带来翻天覆地的转变,机器人正在以多种存在形式逐渐融入人类的日常生活,大大提高了人类的生活质量与幸福感。人工智能技术的发展历经曲折却又在一代又一代人的努力下,不断创新与颠覆。1956年达特茅斯会议,标志着AI的诞生,第一个神经网络由Rosenblatt在1957年提出;但是由于计算能力的限制,没能使机器完成大规模数据训练和复杂任务,AI进入第一个低谷时期;1986年,BP算法的出现使得大规模神经网络的训练成为可能,带来AI的第二个黄金时代;但是,由于AI计算机DARPA没能实现,政府支持度降低,资金投入缩减,使AI进入了第二个低谷;2006年,Hilton提出“深度学习神经网络”,使AI出现了突破性转折,并先后在机器视觉识别、智能语音等领域取得成功,宣告了AI的爆发式发展,表明人类正式进入智能感知时代,带入第四代工业革命。机器人正以不同的身份出现在人们生活的方方面面,逐渐更新我们的生活方式。机器人出现在在林林总总的服务行业中,改变了传统产品形态与服务模式,比如送餐机器人、青少儿教育机器人、老人陪护与家庭助理型机器人、虚拟/增强现实等交互手段等。
同时,智能机器人具有人力劳动无法比拟的优点,比如精度更高、安全性更高、连续工作,既高效完成任务又减少了人力劳动的成本与危险。智能机器人在工业、农业、交通等方面具有绝佳的研究价值与商业价值。比如智能机器人能够完成工业行业的高强度与高难度的各种作业——核电作业、能源矿山作业、矿山勘探开采、水下考察等任务;%湖南大学研发智能桥梁检测特种机器人
在农业方面常见的有播种机器人、收割机器人等;在安全监测、智慧巡检、特殊运输等方面也出现了很多研究热点,例如深圳一清科技研发的“夸父-I”无人车在重灾区、一线城市郊野和产业园区等场所往返运输生活用品、医疗器械等;
在医疗保健行业也存在广大的应用场景,比如打造智慧医疗、疫情防控、救援抢险、助老助残等智能体,本质上解决医生的长时间工作疲劳问题,同时提高医生工作操作精准度,降低患者痛苦加速伤口愈合。外骨骼机器人——模仿人体骨骼结构设计的一种机电一体化装置,通过感知穿戴者的意图,
帮助残障人士更自主的生活,同时帮助负重等工作人士的身体损伤,提升人体机能。%又如在2020年初新型冠状病毒肺炎大肆感染的同时,湖南大学自主研制了测温-诊断智能机器人、智能消毒机器人、
医药物资搬运机器人等,投入多家医院使用。在军事防御、外太空空间探索、极地科考、深海遨游%、应急安防——在外太空、极地以及深海
等极限环境中降低对人类对挑战,拓展人类能力,利用多模态的感知手段为复杂地形下的自主探测提供了新的解决方案,例如2016年10月26日,美国军方
用3架F/A-18“超级大黄蜂”战斗机搭载并释放了103架“山鹑”(Perdix)微型无人机组成的机群,进行了无人机集群飞行试验。2017年以色列在第二届国际陆战会议上展示了“卡梅尔”自助驾驶装甲车。

智能机器人的发展趋势呈现自主化、协作化、灵巧化。
%    自主化:感知、处理、决策、执行等模块,机器人像人类一样能够灵活的处理出现的一些状况,甚至替人类完成高难度的任务,使人工智能作为人类和智能世界的桥梁。
%    协作化:
%    灵巧化:
%\subsection{研究挑战}
%2019年,《科学:机器人》杂志发表了关于智能机器人目前的十大挑战,包括:新材料和制造方案、仿生机器人、动力和能源、机器人集群、导航和探索、智能机器人、社会交互、脑机接口、医疗机器人、机器人伦理及安全。
    
从以上不同应用领域我们不难发现,移动机器人是第四代工业革命中不可或缺的核心技术,是保证人工智能服务于人类的最常见的机器人外在形态之一。%、多足机器人、人形机器人、手机/电脑、服务机器人、手术机器人、空中机器人、AR/VR机器人
移动机器人的代表之一就是无人驾驶车辆,其是指在没有人工干预的情况下,能够感知环境并进行路径规划、导航避障自主到达目的地的汽车。被科学家普遍认可的无人驾驶汽车出现在1921年,
是美国军队在俄亥俄州空军基地展示的一种三轮拖车,因为这辆拖车上配备了无线电控制系统,所以拖车上无需坐人。通用汽车在1956年的时候,推出的一辆Firebird II概念车,成为了无人驾驶概念车的先驱。
1957年,在内布拉斯加州一条高速公路上,美国一些研究人员,通过埋在地下的探测器检测道路上的障碍物。
1960年,英国研究人员利用埋在道路中的信号电缆网络为提供给汽车道路信息,控制汽车自动转向、加速或制动等。要实现自主导航、运动跟踪、障碍物检测和规避等一系列功能,要求移动机器人必须保证能够获得随着时间的推移获取自身位置的改变。
%为解决移动机器人定位问题科研人员尝试了不同的传感器、方案
%不过值得庆幸的是, 每一个逐梦者的挫折都让无人车的前进之路日渐清晰——这是一条智能进化之路,也是人类社会的变革之路。
%无人车利用各种车载传感器来感知车辆周围环境,根据感知所获得的环境 与车辆状态信息,控制车辆安全、可靠地在各种道路上行驶。无人车作为现代 化战争的新概念、
%汽车技术发展的新方向和科学研究的综合验证平台,一直倍受国防事业、汽车工业和高校与科研机构的关注

近年来为了保障机器人能够获得随着时间的推移其位置的改变,研究人员和工程师们开发了各种用于移动机器人定位的传感器、技术和系统,如轮式里程计IMU、惯性导航系统(INS)、激光或超声波测距法、全球定位系统(GPS)和视觉里程计(VO)等。然而,每种方案都有自己的不可避免的弱点:IMU是最简单的位置估计技术,但由于车轮滑移,它存在位置漂移问题\cite{2005Visual};INS也极易产生漂移,而高精度的INS价格昂贵,对于商业用途来说是不可行的解决方案;
激光测距仪以激光器作为光源,由光电元件以一定的工作频率向目标物体发射并接收目标反射的激光束,由计时器来测定激光束从发射出去到接收的时间差,从而计算出观测者到目标物体的距离,由于不受主动光源的影响,激光测距仪可以在黑暗环境中正常工作。超声波测距原理与激光测距原理相似,不过发射装置发出的是超声波而不是激光;GPS是最常见的定位解决方案,因为它可以提供绝对的位置而不会积累误差,但它只在天空视野清晰的地方有效,它不能在室内、深海、密闭空间等环境中使用\cite{2011Combined}。商用GPS所估计位置的误差也较大,通常其误差是在米级。这种误差被认为对于要求精度以厘米为单位的精确应用来说太大,例如室内服务机器人小面积精准定位和自主停车等。差分式全球定位系统和实时运动式全球定位系统可以提供厘米级精度的位置,但这些技术成本较高价格昂贵。GPS是一项已经深深的融入到了我们老百姓的日常生活中的定位方式,包括车载型、通讯型、便携型、船载型、指挥型等多种用户终端,具体如每日必不可少的车辆导航、手机定位等。%其最早应用可以追溯到冷战时期,真正达到民用是在1993年6月26日,美国空军将第24颗Navstar卫星送入轨道,完成了由24颗卫星组成的网络,
GPS接收器的可以保证以较低的价格就可以立即获取我们在地球上所处的10米级的位置,包括纬度,经度和海拔。根据美国政府有关全球定位系统(GPS)\footnote{https://www.gps.gov/chinese.php}的官方信息,截至2020年5月,有29颗可运行卫星。
%卫星每天以20,200公里(12,550英里)的速度绕地球飞行两次。美国空军监视和管理该系统,并承诺在95%的时间中至少拥有24颗卫星,其中21颗工作卫星,3颗备用卫星。24颗卫星
%运行在6个轨道平面上,运行周期为12个小时。保证在任一时刻、任一地点高度角15度以上都能够观测到4颗以上的卫星。
卫星导航系统虽然最早是美国发明并投入使用的,但其他国家也开发了自己的卫星导航系统,比如我们国家自主研发的北斗卫星导航系统、俄罗斯的格洛纳斯系统等。%、欧盟的伽利略定位系统。除伽利略系统外,其余系统都已投入运营。

本文研究重点之一是增量式单目视觉里程计,其是解决机器人定位的最快捷方便且低成本的方式之一。
\section{国内外研究现状介绍}
\subsection{单目视觉绝对定位研究现状}
在绝对定位方中,机器人对世界的认知须以地图的形式存储,进而与当前的观察结果进行对比,通过GPS等tag完成定位。通过对图像进行特征提取并检索以完成匹配。然而用一个稳定鲁棒的特征来表示一个处于变化的动态场景是一个很大的挑战,如图\ref{fig:Environments}所示。

\subsubsection{图像特征提取相关工作}
根据运算过程中采用的图片帧数,单目视觉定位的方法可分为两类:基于单帧图像的定位方法,基于两帧或多帧的定位方法。基于单帧图像的定位方法包括基于特征点的定位(Perspective-n-Point)、基于直线或平面特征的定位,其关键点在于快速准确地实现投影图像与模板之间的特征匹配。基于两帧或多帧图像的定位方法的关键在于实现多帧投影图像之间的对应特征元素匹配,如SLAM。

%根据图像分类,给出了最先进的手工制作特征和非手工制作特征的总体比较。

%对于手动提取特征,各种各样的最先进的算法被认为是\cite{anni2017handcraft}:局部三元模式、局部相位量化、完成的局部二进制模式、旋转不变共现局部二进制模式、旋转局部二进制模式图像、全局旋转不变多尺度共现局部二进制模式和其他几种算法。

%\subsubsection{局部特征手动提取}
从相机采集的视觉数据中提取特征,即图像特征提取是机器人位置识别的一个基本问题,其中图像特征提取的方式是影响定位性能的关键。局部特征(local features),仍是近年来研究的一个热点。局部特征指一些能够稳定出现并且具有良好的可区分性的稳定特征点。局部特征数量丰富,特征间相关度小,不容易受到部分遮挡、光照等噪声的干扰,因为不会因为部分特征的消失而影响其他特征的检测和匹配,这样如果我们用这些稳定出现的点来代替整幅图像,可以大大降低图像原有携带的大量信息,起到减少计算量的作用,且在物体不完全受到遮挡的情况下,一些局部特征依然稳定存在,以代表这个物体(甚至这幅图像),方便进一步分析与运算。

因此,大量的计算机视觉研究集中在如何手动提取特征上,以发现和描述从图像中提取的特征,如SURF\cite{SURF2006surf}、ORB\cite{rublee2011orb}、BRIEF\cite{calonder2010brief}、SIFT\cite{lowe1999object}、Harris\cite{harris1988combined}、SIFT\cite{lowe1999object}和HOG\cite{Dalal2005Histograms}。

%我们可以看下面这个图,左边一列是完整图像,中间一列是一些角点(就是接下来我们要讲的局部特征),右边一列则是除去角点以外的线段。不知道你会不会也觉得你看中间一列的时候能更敏感地把他们想象成左边一列的原始物品呢?

%\subsubsection{全局特征手动提取}
如果用户对整个图像的整体感兴趣,而不是前景本身感兴趣的话,全局特征用来描述总是比较合适的。但是无法分辨出前景和背景却是全局特征本身就有的劣势,特别是在我们关注的对象受到遮挡等影响的时候,全局特征很有可能就被破坏掉了。

非手工特征采用三种方法:基于卷积神经网络(CNN)的深度转移学习特征、主成分分析网络(PCAN)和紧凑的二进制描述符(CBD)\cite{nanni2017handcraft}。 在我们以前的工作中,我们还尝试用IPCA来减少特征维数\cite{zhang2017Dynamic},我们的结果证明了33维深度特征可以在匹配矩阵中以高精度识别。
%\subsubsection{深度学习特征}
最近,非手工制作的特征能够通过深度卷积神经网络(DCNN)从数百万标记图像中自动学习鉴别特征,这在计算机视觉和机器学习社区的几乎所有重要任务中都取得了最先进的性能\cite{Radford2016Unsuperved}\cite{Chen20143D}\cite{Krizhevsky2012ImageNet}\cite{Simonyan2014Very}\cite{Szegedy2015Going}\cite{He2015Deep}。%一个典型的DCNN由许多卷积层和池化层组成,然后是全连接层\cite{Krizhevsky2012ImageNet}。



最近的文献提出了多种方法来解决这个领域的挑战\cite{milford2012seqslam}\cite{corke2013dealing}\cite{neubert2015superpixel}\cite{mcmanus2015learning}\cite{naseer2014robust}\cite{churchill2012practice}\cite{lowry2014transforming}。众所周知,在2012年的AlexNet大规模视觉识别挑战赛(ILSVRC)上一个新的网络模型CNN获得了令人难以置信的准确率\cite{krizhevsky2012imagenet}。 

文章\cite{donahue2014decaf}\cite{girshick2014rich}\cite{krizhevsky2012imagenet}\cite{sermanet2013overfeat}研究表明,来自神经网络的网络特征优于传统的手动特征 Cite{sharif2014cnn,ORB2011orb,surf2006surf,lowe2004distinctive}。
该网络由5个卷积层,以及3个全连接层和soft-max层组成,它在120万张有标签的图像上进行了预训练。根据从AlexNet中提取的特征对图像进行分类。每个单独层的输出可以作为一个全局的图像描述符。我们还可以根据这些特征对图像进行匹配,然后定位机器人。
\cite{donahue2014decaf}表示,来自CNN中层的特征可以更有效地消除数据集的偏差。\cite{sunderhauf2015performance}比较了不同层特征的性能。他们的结果表明,来自ConvNet层次结构中的中间层的特征对一天中的时间、季节或天气条件引起的外观变化表现出鲁棒性。Conv3层的特征在面对极端的外观变化时表现得相当好。表1列出了AlexNet网络中不同层的向量尺寸。\cite{sunderhauf2015performance}证明了Conv3层的特征在外貌变化方面的表现相当好。此外,fc6和fc7在视角变化方面优于其余层。但是,当外观变化时,fc6和fc7完全失效。Conv3的维度为64896,即一幅图像显示为64896维度的矢量。在线定位将持续接收来自摄像头的图像。毋庸置疑,大量高维度向量数学运算增加了运算时间。它在120万张有标签的图像上进行了预训练。根据从AlexNet中提取的特征对图像进行分类。每个单独层的输出可以作为一个全局的图像描述符。我们还可以根据这些特征对图像进行匹配,然后定位机器人。

%DCNN中包含的不同特征最初是以浮动格式返回的。
为了方便后续的二值化,\cite{arroyo2016fusion}将这些特征投向一个规范化的8位整数格式。然后利用汉明距离对所有二进制特征进行匹配,计算出一个匹配矩阵。他们的研究结果表明,对特征进行压缩可以极大程度上降低其描述符的冗余度,而精度只降低了约2\%。此外,他们对特征的二值化允许使用汉明距离,这也代表了位置匹配的加速。在减少特征集的情况下,改进了地点识别。

\subsubsection{图像特征匹配}
机器人视觉图像匹配是指机器人定位领域的场所识别,是继特征提取之后的另一个挑战。毋庸置疑,在绝对定位方中,机器人对世界的认知须以地图的形式存储,进而与当前的观察结果进行对比,通过GPS等tag完成定位。文章\cite{lowry2016visual}指出,根据视觉传感器的不同,以及识别场景种类的不同,地图框架也有所不同。可以分为纯图像检索、拓扑地图和拓扑-度量地图。纯图像检索只存储环境中每个地方的外观信息,没有相关的位置信息,例如FAB-MAP中使用的Chow-Liu树结构\cite{cummins2008fab}。FAB-MAP\cite{cummins2008fab}描述了一种概率方法来解决匹配图像和地图增强的问题。他们使用了基于向量的描述符,如与bag-of-words联合的SURF特征。FAB-MAP\cite{cummins2008fab}描述了一种概率方法来解决匹配图像和地图增强的问题。他们使用了基于向量的描述符,如SURF与Bag-of-Words联合使用。他们通过构建一个Chow-Liu树结构\cite{chow1968approximating}来捕捉视觉词的共现统计,学习了一个图像深度网络特征的生成模型。Chow-Liu树由节点和边组成。变量之间的相互信息关联度由节点之间边的粗细来显示,图中的每一个节点对应一个由传感器数据转换而来的词袋,图中的每一个节点对应一个由传感器数据转换而来的词袋,变量之间的相互信息通过树的边的粗细来显示。图中的每一个节点对应一个由输入感官数据转换而来的词袋表示。在具有挑战性的户外环境中,FAB-MAP能够成功地检测到了大部分的闭环场景。但\cite{naseer2014robust}的结果显示,在跨季节的数据集中,OpenFABMAP2只找到了少数正确的匹配,原因是,传统手工特征描述符是不可重复的。 论文cite{naseer2014robust}将图像匹配制定为数据关联图中的最小成本流问题,以有效利用序列信息。他们通过最小成本流定位车辆。他们的方法即使在高度变化的动态场景也表现良好。SeqSLAM \cite{milford2012seqslam}将图像识别问题构思为在局部邻域内寻找所有与当前图像最佳匹配的模板。这很容易实现。然而,\cite{milford2012seqslam}很容易受到机器人速度的影响。这种影响限制了机器人进行长时间自我定位。%\cite{Schindler2007City}证明,如果使用每张图像中信息量最大的特征,地方识别性能就会提高。


\subsection{增量式定位--单目视觉里程计研究现状}
里程计,英文是"odmetry",该词源于两个希腊词hodos(意为 "旅程"或者"旅行")和metron(意为 "测量")\cite{2005Visual}。这一推导与估计机器人姿势(平移和方向)随时间的变化有关。移动机器人使用来自运动传感器的数据来估计它们相对于初始位置的位置;这个过程被称为里程测量。里程计是使用来自运动传感器的数据来估算其位置随着时间变化的一种定位技术,一些腿式或轮式机器人在机器人定位技术中使用它来估计其相对于起始点的位置。由于对速度测量值进行了时间积分,因此该方法对误差很敏感,可以给出位置估计值。在大多数情况下,需要快速而准确的数据收集,仪器标定校准和快速处理才能有效使用里程计。

视觉里程计,即VO(visual odometry),是一种视觉定位技术,视觉里程计(VO)是一种仅仅通过从连接到机器人的单个或多个摄像头获取的图像序列来实现机器人定位\cite{2008Monocular}移动机器人的增量式定位过程。这些图像包含足够多的有意义的信息(颜色、纹理、形状等),以估计相机在静态环境中的运动\cite{Rone2013Mapping}。增量定位是连续观察机器人姿态的变化,通过累积运动计算机器人当前姿态。VO是未探索环境中机器人自定位和自主导航的基本模块,因为它不依赖于预先构建的地图\cite{liu2012finding}\cite{salarian2018improved}。单目视觉里程计(MVO)由于配备了最便宜、使用最广泛的传感器,也是最方便校准的传感器,因此引起了机器人界的广泛研究兴趣。同时,它不受固定基线长度的限制,可以在不同的场景中广泛使用。然而,当单目相机将三维(3D)世界投影到二维(2D)平面空间时,它失去了物体的深度信息和绝对尺度。因此,MVO只能获得相对的,而不是机器人运动的绝对距离。这种尺度模糊可以积累尺度误差,称为尺度漂移。尺度模糊和尺度漂移统称为尺度问题。VO已经研究了30多年,最初是由NASA的火星探索项目推动的\cite{scaramuzza2012visual}。与IMU和雷达等其他增量定位系统相比,单目视觉里程法(MVO)的优点是显而易见的。它配备了最便宜和使用最广泛的传感器,也是最方便的校准。 因此,MVO是机器人界的一个活跃的研究领域。

尺度问题严重限制了MVO\cite{scaramuzza2011vo}的精度,相应的解可分为相对尺度修正和绝对尺度估计。前者主要包括光束法平差(BA)\cite{triggs1999bundle}和环路闭合(LC)检测。 虽然它们确实在限制尺度问题上起作用,但它们无法检索机器人的度量信息。 对于MVO系统,只有借助先验知识,引用绝对量度信息才能解决尺度问题。流行的绝对量度信息参考包括基线距离(双目摄像机)、摄像机高度(道路模型)和从其他传感器或离线训练中获得的像素深度等\cite{Costante2015Exploring}。 其中,挂载相机绝对高度是常用的,因为它既不需要其他传感器的帮助,也不需要离线训练,是最方便测量和校准的。此外,在车辆运行过程中,摄像机绝对高度保持稳定。

在已安装摄像机高度的先验知识下,MVO的精度取决于相对尺度下的道路几何估计。 当使用相机高度作为尺度恢复的绝对参考时,有必要获得道路的几何模型。其中包含估计的摄像机高度。许多方法\cite{kitt2011mono}\cite{Song2015MoncularScale}\cite{zhou2016reliable}根据先验知识选择一个感兴趣的区域(ROI),或自动检测的确定固定区域\cite{chen2007automatically}作为道路区域。然而,基于ROI的方法有两个缺点。首先,不能保证所选区域始终为路面。此外,图像信息不能得到充分利用。道路检测解决方案更合理,因为它从整个图像中提取特征。此外,由于深度学习方法在多个领域取得超越性性能,\cite{hoiem2007recovering}提出的分割方法和训练分类器也用于道路检测。但这种方法的计算成本较高且对不熟悉的情况不够鲁棒。此外,所有以前基于分类器的方法都集中在道路的颜色信息上,虽然在深度学习的帮助下,基于道路颜色信息的视觉里程法可以得到很大的改进,此类方法对光照、阴影和材料等因素依然较敏感。因此,我们将颜色信息替换为结合道路几何约束的道路点选择。通过在线更新道路分类器使\cite{Lee2015MoncularScale}中的框架更加鲁棒。

虽然融合传感器测量系统目前在精度、鲁棒性和可靠性方面处于领先地位\cite{ye2019tightly}\cite{zhang2014loam}。单目视觉里程计(MVO)却有可能取代它们。MVO面临的许多挑战\cite{scaramuzza2011visual}主要存在于大规模、动态或无特征的环境中)。




\subsubsection{相对尺度校正}
光束法平差\cite{triggs1999bundle}和环路闭合检测相对尺度校正的两种重要方法。尺度校正被光束法平差描述为一个非线性最小二乘问题,以产生联合最优的三维结构和摄像机姿态估计。Mouragon \textit{et al}.\cite{mouragnon2006real}第一次在实时VO中利用光束法平差,其次是并行跟踪和映射(PTAM),这是定向FAST和旋转BRIEF同时定位和映射(ORB-SLAM)的主要动机。然而,局部和全局光束法平差优化都
存在严重的累积尺度误差。环路闭合检测是一种在先前访问过的地点进行比对的技术,以修正产生的位移偏差;

\cite{nister2006scalable}提出了一种词袋(BoW)方法来表示关键帧。 基于快速外观的映射(FAB-MAP)\cite{cummins2008fab}是一种经典的位置识别方法,它与Chow-Liu树\cite{chow1968approximating}一起构造了BoW模型的视觉词汇表,以表达其特征相似性。 然而,在实际的交通场景中,环路很少出现,关键帧的选择严重影响了环路闭合检测的准确性\cite{piao2019real}。此外,在长距离驾驶中,光束法平差下的尺度漂移也变得严重\cite{mouragnon2006real}\cite{klein2007parallel}。

FAB-MAP\cite{cummins2008fab}是一种经典的位置识别方法。它不仅限于定位任务,而且可以判断一个新的观察是否来自地图上已经存在的地方。FAB-MAP与Chow-Liu树\cite{chow1968approximating}一起构造了BoW模型的视觉词汇表,以表达其特征相似度。该方法也应用于另一个优秀的工作ORB-SLAM2\cite{Mur2016ORB}。 该方法基于面向FAST和旋转的关键帧的BRIEF(ORB)描述符离线训练大量的BoW。当摄像机返回到以前的场景时,它将获得类似的BoW描述符,从而检测环路闭合。

\subsubsection{绝对尺度恢复}
绝对尺度恢复方法可以补偿以已知的度量信息作为参考的相对尺度校正的局限性,例如从深度学习中学到的安装相机高度和图像深度。
\paragraph{相机高度-固定方法}
摄像机高度约束方法的区别主要在于道路平面的检测和建模方法。许多方法\cite{kitt2011mono},\cite{Song2015MoncularScale},\cite{zhou2016reliable}假定ROI为道路。 
来自运动的单目大规模多核结构(MLM-SFM)方法\cite{Song2015MoncularScale}中,扩展了\cite{song2014robust}和\cite{Song2013Parallel},假设图像的下三分
之一的中五分之一为ROI。MLM-SFM提出了一种数据驱动机制,将多个线索组合在一个框架中,该框架反映了它们的每帧相对机密性,这显示了很有前途的性能。然而,当所选区域被
汽车或其他东西遮挡时,基于RIO的方法无法工作,如KITTI数据集的序列07中所发生的那样,因此MLM-SFM不能像我们所预期的那样在它该环境运行。同时图像信息可能无法充分
利用,因为ROI只是图像的一小部分。第二个缺点可以用双向或全向相机\cite{gutierrez2012full},\cite{scaramuzza2008appearance},\cite{scaramuzza2006flexible}弥补。但是考虑到传感器的成本及使用便捷性,、
普通单目相机更具有研究价值。

道路平面估计方法将判断哪些点属于道路进行帧对帧运动估计,会进行特征点筛选并充分利用整张图像信息,因此更合理。 这些方法可根据其特征点大小分为稀疏\cite{engel2017direct},半稠密\cite{Engel2014LSD}\cite{forster2014svo},稠密\cite{newcombe2011dtam}。 描述符也可能是由一些视觉过程\cite{chen2012tag}增强或从卷积神经网络(CNNs)中提取\cite{lin2017hnip}\cite{chadha2017voronoi}。除了特定特征点的方法外,多种描述符相结合的方法也很流行。 该方法将来自稠密和稀疏匹配点的线索结合起来,并使用分类器基于各种特征检测尺度异常值,这确实提高了对各种地面结构的鲁棒性。然而,它依赖于稠密的特性,如果没有GPU的帮助,它就不能很容易地在移动嵌入式系统上实现。 我们的方法在没有稠密特征的任何帮助下取得了有竞争力的结果。

在道路点检测后,一些方法\cite{Gall2015}选择利用三角稀疏地面点计算高度,然后估计绝对尺度。传统上,采用三点RANSAC\cite{choi1997performance}来实现鲁棒平面拟合。
在非传统方法中,\cite{pereira2018novel}用逆向选择代替RANSAC,\cite{Geiger2011IV}用一种快速匹配方法对一组选定的点进行三角剖分,该方法称为有效的大规模立体声\cite{geiger2010efficient}。

\paragraph{基于图像深度的方法}
最近,MVO有一个与深度学习相结合的流行趋势,其中包括从图像与CNN估计的深度。对于MVO的训练数据,\cite{rukhovich2019estimation}中的结果表明,从合成训练输入中获得的尺度估计精度与从实际数据中获得的估计精度相似。在\cite{saxena2006learning}中,使用深度CNN与CRF相结合的方法来估计在小规模和大规模环境中拍摄的单目图像的深度。Luo \textit{et al.}将在线自适应深度与直接单目SLAM相结合\cite{luo2018real},提高了不同场景的深度预测精度。它有望解决两个核心挑战:地图完整性低和尺度模糊。然而,单帧图像的深度估计\cite{karsch2014depth,ranftl2016dense,yang2019bayesian,eigen2014depth,saxena2006learning}比来自连续帧的深度估计更复杂\cite{yang2019bayesian}。\cite{yin2017scale}提出了一种新颖的监督系统从估计的深度图计算平移的尺度,将条件随机场与CNN网络相结合以优化深度图,这是通过考虑两个连续图像和运动约束来改进的。

深度预测的准确性对单目SLAM中的特征跟踪误差有巨大影响。CNN-SLAM\cite{tateno2017cnn}扩展了大规模直接SLAM(LSD-SLAM)\cite{engel2014lsd},通过部署深度神经网络的预测深度图来产生密集的3D地图。它在室内数据集\cite{Sturm2012A,2014A}中取得了很好的效果,但在多个关键帧重叠时,预测的深度图无法优化,从而使得重建和映射的精度降低。DVSO\cite{yang2018deep}使用与\cite{godard2017unsupervised}类似的虚拟立体视图,将深度预测纳入几何单目测绘流水线。Luo等人\cite{luo2018real}将在线适应深度与直接单眼SLAM相结合,提高不同场景的深度预测精度。这些方法都有希望解决地图完整性低和比例尺模糊性这两个核心难题。

从连续帧中进行深度估计比从单幅图像中进行深度估计更容易\cite{yang2019bayesian,eigen2015predicting}。\cite{Costante2015Exploring}的方法从连续图像中提取密集的光流,并训练一个基于深度CNN的估计器来进行自运动估计。本研究中的新型监督系统通过考虑两幅连续图像和运动约束,从估计的深度图中计算出转换的尺度,并对其进行了改进。他们的网络是通过并发CNN和条件随机场来构建的,以完善深度图。除了估计单视角深度,\cite{zhan2020visual}还尝试估计双视角光流作为另一个中间输出。最近,Xue等人\cite{xue2020toward}提出了一种利用密集法线进行道路检测的新方法,在几何约束方面与我们的方法类似。这些基于端到端深度学习的SLAM系统已经取得了令人印象深刻的性能,然而,CNN的深度预测不准确会严重导致单目SLAM中的特征跟踪误差,且它们都需要进行离线训练,增加了时间成本与计算成本。此外,也不能保证它们能泛化到新的环境中。我们的系统不仅可以在新的环境中工作,而且可以用低成本的硬件达到较好的性能。





视觉里程计的实现方法?按照传感器数目的不同可以分为:多目相机、双目相机、单目相机,其中相机的类型包括普通相机、鱼眼相机和全景相机等。双目里程计与单目里程计是视觉里程计的研究热点,双目VO的优势在于,对机器人的运动轨迹估计更加精确,且具备明确的距离单位。而在单目VO中,如果没有已知量度大小作为参考我们只能知道物体在x/y方向上移动了1个或多个单位,却不知道具体的单位大小。但是,单目视觉里程计比双目里程计更有研究价值,原因如下:单目相机比双目相机更加便于标定校准、当物体距离机器人很远的,双目系统又退化为单目系统。而且当机器人形体很小时,对单目相机被占据更小空间,即使安装了双目相机,但是因为两个相机距离及其近,又可被近似为单目系统。

视觉里程计的主要工作就是计算从图像$I_t$到图像$I_{t+1}$位置变换$T_k$ ,然后集成所有的姿态变换恢复出相机相对于初始位置坐标$\mathbf{P}_0$的位姿$\mathbf{P}_t$,这意味着VO是一种增量式轨迹重建方法。

\section{本文内容与贡献}
本文提出了哪几种方案,主要贡献是:
绝对定位之单目深度特征提取与压缩
增量定位之单目视觉里程计的绝对尺度运算
增量定位之单目视觉里程计求解新框架简化
\begin{enumerate}
	\item 基于固定相机高度,基于路面几何约束的单目视觉里程计尺度恢复。
	\item 基于单目模型的单目尺度计算:从手动建模到自主学习
	\item 传统视觉位姿估计与深度学习尺度恢复的结合
	\item 视觉定位及其与视觉里程计的结合
\end{enumerate}

\section{全文架构}
       %3分   
\chapter{数学原理及研究热点数学建模}
\label{ch:principle}
\section{单目视觉绝对定位数学原理}
%准确的视觉定位是自主导航的关键技术,顾名思义,单目视觉定位就是在仅仅利用一台摄像机的条件下完成机器人定位工作。
大规模绝对视觉定位的最先进技术包括基于2D图像的图像检索法和基于3D结构的2D-3D匹配方法。基于2D图像的图像检索法根据当前图像特征,从具有地理标记图像的数据库中调取出与当前图像最相似的模版图像,并以该模版图像已知的地理标记(常用全球定位系统,即GPS信息)作为自身的地理位置,并返回最相关的数据库图像的姿势。基于3D结构的方法采用场景的3D模型,使用2D-3D匹配与3D模型进行相机姿势估计来非常精确地估计相机的6-DOF姿势。然而,构建大规模的3D模型仍然是一个巨大的挑战。在\cite{2017Are}中通过大量实验证明,当检索到足够多的相关数据库图像时,基于二维图像检索的单目视觉绝对定位可以恢复出精确的相机姿势,大规模的3D模型并不是准确的视觉定位所必需的。相比之下,基于二维图像检索的方法只需要一个地理标记图像的数据库,节省了大量的地图构建和维护成本。


\section{单目视觉增量式定位数学原理}
在本章我们首先介绍了该方法的背景和表示法,然后详细介绍了我们在道路模型计算中的道路点选择算法。 最后,我们使用RANSAC计算相机的初始高度,并采用中值滤波器来减少噪声干扰。

在我们提出的方法中,在摄像机高度保持不变,地面局部平面的假设下,摄像机到路面平面的绝对高度$h_0$被认为是一个参考。所提出的MVO尺度恢复算法的结构如图\ref{fig:structure_mvosr}所示。初始自我运动($R$和$t$
在相对尺度上)和匹配特征由初始VO过程给出。道路模型估计模块计算安装摄像机的相对高度,即摄像机的光学中心到地面的距离,并有经过验证的道路点。详细来讲,图像首先被Delaunay三角剖分进行分割,%每个三角形都会通过两个约束————深度一致性和路面模型约束进行筛选以判断其是否属于路面。
每个三角形首先通过考虑深度一致性来确定是否属于道路区域。其余点再通过Delaunay三角剖分,并通过考虑道路模型一致性进行选择。通过筛选的路面特征点被用于计算路面模型$\mathbf{n}^T_i\mathbf{x_i}-\bar{h}_{i}=0$,以求得摄像机相对高度$\bar{h}_i$,通过与给定相机绝对高度比较恢复相机运动绝对尺度$s$,以求得绝对运动估计${R}$和${t}$。

MVO旨在求解相机相对于初始位置坐标$\mathbf{P}_0$的位姿$\mathbf{P}_t$。两帧图像之间$I_t$和$I_{t-1}$之间的相机运动$\mathbf{R}$与$\mathbf{t}$可通过累积求取:
%(\mathbf{i}=0,1,2,...,n)
%\begin{equation}
$\mathbf{P}_t = \mathbf{P}_{t-1}\mathbf{T}$,这里估计运动$\mathbf{T}=[\mathbf{R},\mathbf{t};\mathbf{0},\mathbf{1}]$. 
两帧图像$I_{t-1}$和$I_t$之间的匹配特征可以分别表示为$\mathbf{M}_{t-1}$和$\mathbf{M}_t$。
对于最初的两个帧,最常用的解决方案是求解基本矩阵,因为特征点的三维坐标是未知的\cite{luong1996fundamental}:
\begin{equation}
    {\mathbf{M}_{t-1}}^T\mathbf{F}\mathbf{M}_{t}=0
\label{eq:initial}   
\end{equation}
这里$\mathbf{F}={\mathbf{K}^{-1}}^{T}{[\mathbf{t}]_\times}\mathbf{R}\mathbf{K}^{-1}$是基础矩阵。
$\mathbf{K}=[f_x,0,c_x;0,f_y,c_y;0,0,1]$是标定得到的相机内参,其中$c_x$和$c_y$是光心像素坐标(主光轴在物理成像平面上的角点),$f_x$, $f_y$是焦距,即左右投影中心(光心)到物理成像平面的距离。$\mathbf{[\mathbf{t}]_\times} = [0, -t_3, t_2; t_3, 0, -t_1; -t_2, t_1, 0]$是位移矩阵的斜对称矩阵。在公式\eqref{eq:initial}中, 基础矩阵
$\mathbf{F}$首先被解决,然后通过$\mathbf{F}$的矩阵分解得到旋转矩阵$\mathbf{R}$和位移矩阵$\mathbf{t}$\cite{Nister2004An}。我们通过观察得知如果对位移矩阵
$\mathbf{t}$乘以一个系数时$s \in \mathbb{R}$, 公式\eqref{eq:initial}仍然成立。

\begin{equation}
    {\mathbf{M}_{t-1}}^T{\mathbf{K}^{-1}}^{T}{s}{[\mathbf{t}]_\times}\mathbf{R}\mathbf{K}^{-1}\mathbf{M}_{t}=0.
\end{equation}
我们可以看到,图像$I_i$中的度量信息是在同一尺度上的,但MVO在没有先验知识的帮助下,不能直接计算出位移向量$\mathbf{t}$在帧$I_i$中的绝对尺度$s_i$,单纯MVO无法获得绝对尺度$\mathbf{\bar{t}}$。
这意味着MVO可以保证在不同时间计算的相对转换向量$\bar{\mathbf{t}}$在同一尺度上,但转换向量$\mathbf{t}$的绝对尺度无法通过分解基本矩阵$\mathbf{F}$实现。所以强调了绝对先验知识的重要性。

对于接下来的帧,在获取初始运动后,三角测量法计算出特征点$\mathbf{\bar{x}}_i$的三维坐标,该坐标与$\bar{\mathbf{t}}$的比例相同。下一个相机姿势是由3D地图和当前帧通过透视-n-point(PnP)方法计算出来的\cite{lepetit2009epnp},通过求解 
\begin{equation}
    \mathbf{R,\bar{t}}=\mathop{\argmin}_{\mathbf{R,\bar{t}}}\sum_{\mathbf{\bar{x}}_i, \mathbf{\bar{u}}_i}|\frac{\mathbf{K}\left(\mathbf{R}\mathbf{\bar{x}}_i+\mathbf{\bar{t}}\right)}{\mathbf{\bar{x}}_{i3}}-\mathbf{\bar{u}_i}|
\end{equation}
其中$\mathbf{\bar{x}}_{i3}$是向量$\mathbf{\bar{x}_i}$的第三个元素。特征点$i$在帧$I_t$中的2D像素坐标表示为$\mathbf{u}_i=(u_i,v_i)$。这种方法可以保持尺度,但误差会累积。
大多数方法,如直接稀疏odmetry(DSO)\cite{engel2017direct},大规模直接单目SLAM(LSD-SLAM)\cite{engel2014lsd},ORB-SLAM \cite{raul2015orb},
以及半间接视觉odmetry(SVO)\cite{forster2014svo},试图通过光束法平差和环路闭合检测技术来对抗尺度漂移,而不是考虑绝对尺度计算。在不与IMU和GPS等其他传感器融合的情况下,
利用周围环境中已知的绝对比例尺来还原比例尺是一种便捷的方法。借助环境中的度量信息$l$,我们根据其相对尺度计算出尺寸$\bar{l}$,并通过$\frac{l}{\bar{l}}$计算出尺度系数。
$s=\left.{l}\middle/ \bar{l} \right.$。位移矩阵是根据公式$\mathbf{t}=s\mathbf{\bar{t}}$计算恢复的绝对位移$\mathbf{t}$。

在本文中,所有的标量、向量和矩阵分别用纯字母(如$s$)、粗体小写(如$\mathbf{t}$)和粗体大写(如$\mathbf{R}$)表示。默认情况下,向量是列式的。
矩阵$\mathbf{R}$的$i_{th}$行和$j_{th}$列中的元素用$R_{ij}$表示。上面带有条形的变量为相对尺度(例如,$\mathbf{\bar{t}}$)。特别是,我们把一个向量的斜
对称矩阵表示为$[*]_\times$(例如,$[\mathbf{t}]_\times$)。数学集用希腊大写字母表示。例如,$\nabla$表示通过Delaunay三角测量分割的三角形集,$\Theta$表示属于道路的验证三角形。
$Omega$表示初始特征点集,算法\ref{alg:flat_selection}中验证的道路点集用$\Gamma$表示。这些三角形$\nabla$的内部区域和顶点分别表示为$\widetilde{\nabla}$和$\widehat{\nabla}$。               %deep
\chapter{动态环境中绝对定位之单目深度特征提取与压缩}
\label{sec:extract}

\begin{figure}[t]
  \centering
  \includegraphics[width=0.95\columnwidth]{introduction/ImageRetrieval.pdf}
  \caption{基于图像检索的单目绝对定位示意图}  
  \label{fig:ImageRetrieval}
\end{figure}

如何在动态环境中快速准确地自主定位机器人是机器人路径规划、导航、避障等一些问题的保障。与深度学习相结合的单目视觉定位已经获得了令人难以置信的结果。然而,从深度学习中提取的特征维度巨大,匹配算法也很复杂。如果自动驾驶汽车只能在单一场景中进行训练,将很难满足复杂多变的现实场景。如何减少尺寸与精确的定位是困难之一。本章提出了一种新的方法,通过对动态环境中的大尺度图像训练,来探索满足一定精度要求的深度特征维度。我们从AlexNet网络中提取特征并通过IPCA减少了特征的维度,更重要的是,我们用核化方法、归一化和形态学等方法处理得到匹配矩阵,消除了图像的冗余匹配与歧义。最后,我们在跨季节的动态环境数据集Norland中在线检测最佳匹配序列,证明了经过特征降维后该深度特征仍然能够表达图像信息,仍可以快速定位机器人。

在过去的几年里,为了找到不受大幅度变化影响又能表达场景信息的特征,人们已经研究了各种类型的特征用于本地化\cite{cummins2008fab,sunderhauf2011brief,milford2012seqslam,arroyo2014fast}。图像描述符可分为基于特征的局部特征和整体的图像全局特征。基于特征的描述符在计算机视觉中起着重要的作用。到目前为止,一些手工制作的特征已经获得了一定的成功\cite{lowe2004distinctive,tola2008fast,SURF2006surf,ORB2011orb}。然而,机器人在动态环境中往往无法通过这些手工制作的特征来进行定位。
\begin{figure}[htbp]
 \centering
 \begin{subfigure}[h]{0.23\textwidth}
 \includegraphics[width=\textwidth]{ICVS/RainNight.jpg}
 \label{fig:RainNight}
  \caption{Rainy night}
 \end{subfigure}
 \begin{subfigure}[h]{0.23\textwidth}
 \includegraphics[width=\textwidth]{ICVS/RainDay.jpg}
  \caption{Rainy daytime}
  \end{subfigure}
 \begin{subfigure}[h]{0.23\textwidth}
 \includegraphics[width=\textwidth]{ICVS/Shadow.jpg}
  \caption{Shadows of trees and buildings and others}
 \end{subfigure}
 \begin{subfigure}[h]{0.23\textwidth}
 \includegraphics[width=\textwidth]{ICVS/Dynamic.jpg}
  \caption{Moving objects}
  \end{subfigure}
 \begin{subfigure}[h]{0.23\textwidth}
 \includegraphics[width=\textwidth]{ICVS/Dusk.jpg}
  \caption{Dusk}
 \end{subfigure}
 \begin{subfigure}[h]{0.23\textwidth}
 \includegraphics[width=\textwidth]{ICVS/Light.jpg}
 \caption{Light outsides}
 \end{subfigure}
 \label{fig:Environments}
  \caption{复杂多变的动态环境。}
\end{figure}

\begin{figure}[t]
  \centering
  \includegraphics[width=0.95\columnwidth]{ICVS/FrameWork1.pdf}
  \caption{基于AlexNet的深度特征单目图像匹配定位框架。}  
  \label{fig:Framework}
\end{figure}
图像全局描述符根据图像不变特征表达整幅图像信息。最近的结果表明,从卷积神经网络中提取的通用描述符非常强大\cite{sharif2014cnn}。2012年,CNN在AlexNet大规模视觉识别挑战赛(ILSVRC)上获得了令人难以置信的准确性\cite{krizhevsky2012imagenet}。这表明,从CNN中提取的特征在分类上明显优于手工制作的特征。他们在120万张标注的图像上训练了一个名为AlexNet的大型CNN。对于图像根据AlexNet提取的特征进行分类,我们也可以根据这些特征来定位机器人。\cite{donahue2014decaf}研究表明来自CNN中层的特征可以更有效地消除数据集偏差。\cite{sunderhauf2015performance}比较了来自不同层的特征的性能。他们的结果表明,来自ConvNet层次结构中的中间层的特征对一天中的时间、季节或天气条件引起的外观变化表现出鲁棒性。来自Conv3层的特征在面对极端外观变化时表现得相当好。然而,CNNs特征的主要障碍是昂贵的计算成本和内存资源,这对实时性能是一个很大的挑战。如果我们将巨大维度的图像特征与记录的数据集逐一在线比较,将耗费大量的时间,因此有必要降低CNNs特征的计算成本和内存资源。所以有必要降低这些向量的维度。\cite{arroyo2016fusion}将CNN特征的冗余数据压缩成一个可控的比特数。通过应用简单的压缩和二值化技术来减少最终的描述符,以便使用汉明距离进行快速匹配。压缩意味着丢失一定量的信息。但是,我们可以尽可能地保留数据之间的重要关系。我们通过在数据分析中广泛使用的增量PCA(Principal Component Analysis)来实现这一目的。在本章中,我们提出了一种新型的机器人跨季节动态环境定位算法。

本章的主要贡献有:
1)我们通过深度学习特征的维度减少,提出了一种新型的动态环境下的定位系统。
2)我们减少了从AlexNet中提取的特征的维度。它不仅可以加快计算速度,而且可以减少从数据集引起的图像与大多数在线图像匹配的混乱匹配。
3)代替复杂的数据关联图,通过对匹配矩阵进行形态学处理,在线找到最佳匹配序列。
\section{单目视觉绝对定位深度特征提取方法}
在本文中,我们提出一个新的视觉定位图像特征提取方法,它结合CNNs网络特征表示的优势,在不同季节等环境条件下执行基于单目视觉的鲁棒定位,正如方法框架图(图\ref{fig:Framework})所示,我们的工作过程如下:
1)从AlexNet的Conv3中提取特征,并通过IPCA进行图像特征降维。
2) 将在线图像的向量与已存数据集的向量通过余弦距离逐一匹配。通过核化方法对匹配矩阵进行归一化,以减少因大部分在线图像匹配的数据集混乱造成的歧义。
3)对匹配图像进行图像处理,包括图像二值化、图像侵蚀等。
4)设置参数,通过RANSAC(随机样本共识)在线寻找最佳匹配序列。
本文的研究过程如下。 在第3节中,我们描述了我们的方法的细节。在第4节中,我们在Norland数据集上做了一个动态环境下的在线定位实验。在第5节中,我们对结果和未来的工作进行了讨论。
\subsection{基于AlexNet网络的深度学习特征提取}
\begin{algorithm} 
 \caption{视觉定位算法}
 \KwIn{视觉地图$\{[\mathbf{f},\mathbf{l}]\}_{i=1}^{n}$, where $\mathbf{f}$ is the feature vector of image on location extracted from AlexNet;  $\mathbf{l}$ and $n$ is the size of visual map; Current image sequences $\{\mathbf{I}\}_{j=t-m+1}^{t}$, where $m$ is the sequence size; last robot location $\hat{l}_{t-1}$}
 \textbf{Initialize}: $\hat{l}_t = \hat{l}_{t-1}$
 \KwOut{机器人当前位置$l_t$}
  \For{$t=2$ to n}
  { 计算图像$\{\mathbf{I}\}_{j=t-m+1}^{t}$的特征表示$\{\mathbf{\hat{f}}\}_{j=t-m+1}^{t}$  \\
    计算匹配矩阵$\mathbf{M}$,其中每个元素$\mathbf{M}_{ij}=\mathcal{F}(\mathbf{f}_i,\mathbf{\hat{f}}_j)$   \\
    矩阵归一化$\mathbf{M}_{ij}=\frac{255\left(\mathbf{M}_{ij}-\mathbf{M}_{min}\right)}{\mathbf{M}_{max}-\mathbf{M}_{min}}$ take $\mathbf{M}$ as a gray image $\mathbf{I}_g$\\
    用合适的阈值对矩阵$\mathbf{I}_g$进行二值化,得到$\mathbf{I}_b$\\
    对匹配图像$\mathbf{I}_b$ 进行处理得到$\mathbf{I}_m$\\
    使用RANSAC法得到最佳匹配曲线$y=kx+b$ on $\mathbf{I}_m$ \\
    在视觉地图中当前图像的最佳匹配特征是$\mathbf{f}_{km+b}$\\
    所以当前位置是$\hat{l}_t = l_{km+b} $
 }
 return $\hat{l}_t$
  \label{alg:Algorithm}
\end{algorithm}

本文算法框架如算法\label{fig:Framework}所示。关于地图框架,我们采用的是纯图像检索,但数据集是按照图像传入时间的先后顺序存储的。
不仅可以保证精度,同时保证了计算效率。我们选择AlexNet的Conv3中的特征作为我们的整体图像描述符。Conv3的维度是64896,也就是说,一张图像显示为64896维的向量$\mathbf{f}$。
我们用每张图像的位置建立可视化地图$\{[\mathbf{f},\mathbf{l}]\}_{i=1}^{n}$。所以当前的图像序列表示为$\{\mathbf{I}\}_{j=t-m+1}^{t}$。
为了减少高维度向量运算耗时,我们通过PCA(主成分分析)来减少深度特征维度。虽然图像特征描述在一定程度上丢失了信息,但同时减少了因天空、地面和树木等数据集中因背景信息而导致的模糊匹配。
在线图像的向量将通过余弦距离与数据集向量进行逐一比较,然后然后得到匹配矩阵$\mathbf{S}$,其组成元素是位于(0,1)之间的浮点数。通过核化方法对匹配矩阵进行归一化处理,以减少因与大多数在线图像匹配的数据集混淆而造成的歧义。
将匹配矩阵保存为灰色图像,然后通过合适的阈值将其转换为二进制图像。此外,我们对匹配的灰度图像进行了核化与侵蚀,以消除接近最佳匹配的相似匹配的影响。我们尝试调整参数,然后通过RANSAC在线寻找最佳匹配序列。
匹配矩阵中当前图像的最佳匹配特征为$\mathbf{f}_{km+b}$。那么当前图像在视觉图中的最佳匹配图像为$\mathbf{l}_{km+b}$。

%我们的实验电脑配置是英特尔Core i7,CPU处理器。
我们采用Tensorflow深度学习框架,从AlexNet的Conv3中提取特征,作为每张图像的全局特征描述符。Conv3层的向量维度为64896,也就是说每一张图像的深度特征由64896维的向量来表示。AlexNet的不同层特征适用于不同的机器视觉任务。AlexNet ConvNets中不同层的向量尺寸见表\ref{tab:layer}。\cite{krizhevsky2012imagenet}。层次较高的层在语义上更有意义\cite{sunderhauf2015performance},但同时失去了过多的语义信息,而对同类型场景无法区分。所以使用网络的哪一层作为图像特征表示具有重要意义。来自Conv3层的特征在剧烈的外观变化条件下仍表现较好。此外,fc6和fc7在视角变化方面优于其余层。然而,当外观变化时,fc6和fc7完全失败。基于以上研究,我们用AlexNet的Conv3的特征来表达图像。

\begin{table}[!htbp] 
    \caption{AlexNet不同层特征维度} 
    \label{tab:layer}
    \begin{center} 
        \begin{tabular}{c c c c} 
              \toprule
             Layer & dimensions & Layer & dimensions\\
              \midrule 
             Conv1 & 96$\times$55$\times$55 & Conv4 & 384$\times$13$\times$13\\
             pool1 & 96$\times$27$\times$27 & Conv5 & 256$\times$13$\times$13\\
             Conv2 & 256$\times$27$\times$27 & fc6 & 4096$\times$1$\times$1\\
             pool2 & 256$\times$13$\times$13 & fc7 & 4096$\times$1$\times$1\\
             Conv3 & 384$\times$13$\times$13 & fc8 & 1000$\times$1$\times$1\\
             \bottomrule 
         \end{tabular} 
     \end{center} 
 \end{table}
\subsection{基于AlexNet网络的单目视觉绝对定位深度特征降维}
\begin{table}[!htbp] 
    \caption{信息保有率与参数n\_components的关系。} 
    \label{tab:component}
    \begin{center} 
        \begin{tabular}{cccc} 
              \toprule
             n\_components&Ratio& n\_components&Ratio\\
              \midrule 
              316 & 99\%  & 51  & 93\%\\
              187 & 98\%  & 44  & 92\%\\
              136 & 97\%  & 38  & 91\%\\ 
              99  & 96\%\ & 33  & 90\%\\ 
              76  & 95\%\ & 29  & 89\%\\          
              62  & 94\%\ & 25  & 88\%\\ 
                         \bottomrule 
         \end{tabular} 
     \end{center} 
 \end{table}{}


\begin{figure}[h]
  \centering
  \begin{subfigure}[h]{0.23\textwidth}
    \includegraphics[width=\textwidth]{ICVS/5.png}
    \caption{图像特征保留5维时的匹配矩阵}
  \end{subfigure}
  \begin{subfigure}[h]{0.23\textwidth}
    \includegraphics[width=\textwidth]{ICVS/10.png}
    \caption{图像特征保留5维时的匹配矩阵}
  \end{subfigure}
  \begin{subfigure}[h]{0.23\textwidth}
    \includegraphics[width=\textwidth]{ICVS/20.png}
    \caption{图像特征保留5维时的匹配矩阵}
  \end{subfigure}
  \begin{subfigure}[h]{0.23\textwidth}
    \includegraphics[width=\textwidth]{ICVS/33.png}
    \caption{图像特征保留5维时的匹配矩阵}
  \end{subfigure}
  \begin{subfigure}[h]{0.23\textwidth}
    \includegraphics[width=\textwidth]{ICVS/51.png}
    \caption{图像特征保留5维时的匹配矩阵}
  \end{subfigure}
  \begin{subfigure}[h]{0.23\textwidth}
    \includegraphics[width=\textwidth]{ICVS/99.png}
    \caption{图像特征保留5维时的匹配矩阵}
  \end{subfigure}
 \caption{深度特征分别保留5,10,20,33,51,99维时的匹配矩阵}
 \label{fig:MatchingComparision}
\end{figure}

我们在Norland数据集上进行测试,以选择平衡计算消耗和准确性的图像特征维度。
Norland数据集是:*******
我们选择300张春季的图像序列进行深度特征训练,500张秋季的图像序列作为测试。我们在scikit-learn中使用IPCA进行大量的图像匹配。PCA是高维数据分析的重要手段之一。PCA通过线性变换将高维数据转化为低维数据。AlexNet各层的维度如表1********所示。从表中可以看出,我们保留的维度越多,获得的信息就越多,但也很耗时,所以首要任务是由参数n\_components为参照以确定每个向量保留多少维度,该参数与主信息Ratio之间的关系列在表\ref{tab:component}中。一般来说,在保持一定精度的情况下,我们最好保持至少90%的主信息比。我们还对不同维度的匹配结果进行了比较,比较结果如图 \ref{fig:MatchingComporision}所示。随着特征维度的降低,匹配曲线变得模糊,当图像特征维度降低至20以下时无法检测出最佳匹配线。而保留33个维度时匹配矩阵足够清晰,同时也节省了计算量。综上所述,我们选择33个维度的向量作为图像描述符。
\begin{figure}[h]
  \centering
  \includegraphics[width=0.6\textwidth]{ICVS/cos-crop.pdf}
  \caption{图片特征描述子降维匹配矩阵示意图}
  \label{fig:cos}
\end{figure}

\subsection{单目视觉绝对定位匹配矩阵核化处理与归一化}
\begin{figure*}[t]
  \centering
  \includegraphics[width=0.85\textwidth]{ICVS/1.pdf}
  \caption{矩阵核化运算前后函数曲线对比。}
  \label{fig:Kernel}
\end{figure*}

我们的任务是准确地找到匹配矩阵中的最佳匹配线。我们利用数学变换使这条线更加清晰比如核化方法,包括对匹配矩阵的元素进行反演和指数化。选择这种方法的原因如下。
1)2幅图像之间的余弦距离,即匹配矩阵元素与图像相似度不是完全正相关关系。
2)核化方法会扩大假阴性与真阳性之间的距离。
图\ref{fig:Kernel}是由余弦距离计算出的函数曲线比较,如公式(\ref{tab:cosine})所示,核化距离如公式(\ref{tab:normalization})所示。红色圈线代表两个图像向量的余弦距离。绿色星线代表的是核法距离 右图是两个图像向量的核法距离。我们可以看到,核化法可以增强两个不同地点之间的差异。最佳匹配图像的颜色会显示为黑色,不同的地方会显示为白色,如图\ref{fig:KernelMatrix}所示。更重要的是,通过内核法对匹配矩阵进行归一化处理,可以减少大部分在线图片匹配数据集混乱造成的歧义。将匹配矩阵保存为灰度图像,以便进行后续的处理,包括形态变换和二值化。我们对匹配矩阵进行了归一化处理,范围为0~255,公式为(\ref{tab:normalization}),经过核方法后,匹配矩阵就变得更加明显了。


我们在Norland数据集中选取了3000张在同一个地方拍摄的春季和冬季图像,测试了核化方法。由于两个图像序列起点是同一地点,因此,在对角线上出现的匹配线即最佳匹配序列。匹配结果如图\ref{fig:KernelMatrix}所示,我们通过余弦距离$cos <\mathbf f_i,\mathbf f_j>$将在线图像与记录的数据集图像逐一进行匹配。因此,一条线出现在对角线上,为最佳匹配序列。匹配结果如图\ref{fig:KernelAfter}所示。
我们通过余弦距离$cos <\mathbf f_i,\mathbf f_j>$将在线图像与地图数据集图像逐一匹配。然而,匹配的图像显示为图\ref{fig: KernelBefore},这意味着错误匹配和最佳匹配之间产生了混淆。然而,通过核方法和归一化,对角线变得很明显,如图\ref{fig:KernelAfter}所示。最后,将匹配矩阵保存为灰度图像,通过适当的阈值将其转换为二进制图像。

\begin{equation}
\label{tab:cosine} 
\cos <\mathbf f_i,\mathbf f_j> = \frac{\displaystyle\sum_{i=1}^{33} a_i b_i}{\displaystyle\sum_{j=1}^{33} a_j^2 \displaystyle\sum_{k=1}^{33}b_k^2}
\end{equation}
$\mathbf f_i=\{\mathbf{{a}_1\ {a}_2\ ... \ {a}_{33}}\}$,  $i\in D$, D is set of datasets images
$\mathbf f_j=\{\mathbf{{b}_1\ {b}_2\ ... \ {b}_{33}}\}$,  $j\in O$, O is set of online images
\begin{equation}
\label{tab:normalization}
\mathbf{M}_{ij}=\frac{255\left(\mathbf{M}_{ij}-\mathbf{M}_{min}\right)}{\mathbf{M}_{max}-\mathbf{M}_{min}}
\end{equation}
\begin{figure}[h]
\centering
\begin{subfigure}[t]{0.23\textwidth} 
  \includegraphics[width=\textwidth]{ICVS/MatchMatr_spring_winter_3000_3300_KITTI.png}
  \caption{欧式距离匹配矩阵}
  \label{fig:KernelBefore}
\end{subfigure}
\begin{subfigure}[t]{0.23\textwidth} 
  \includegraphics[width=\textwidth]{ICVS/KernelMethodMatchMatr_spring_winter_3000_3300_e_255.png}
  \caption{核化处理后的匹配矩阵}
  \label{fig:KernelAfter}
\end{subfigure}
\caption{Norland数据集春-冬跨季节匹配矩阵}
\label{fig:KernelMatrix}
\end{figure}


\section{基于AlexNet的单目视觉绝对定位实验}
我们的实验旨在展示我们的方法对图像进行深度特征降维以及匹配矩阵核化处理后对定位效果的影响。我们的方法能够(i)在跨季节的场景中进行定位,忽略动态物体、不同天气和季节变化。(ii)节省时间和计算成本。我们在SeqSLAM中使用的公开数据集Norland上进行了评估\cite{milford2012seqslam}。采用该数据集中64x32的灰色图像图像频率为1帧/秒,如果我们的方法在这种不清晰和微小的低信息含量图像中仍然有效,那么它可以节省大量的时间和计算消耗。我们可以看到在图\ref{fig:KernelAfter}中,最佳匹配线已经很明显了。我们通过经典的RANSAC算法找到它的数学模型,在数据集中找到相应的索引。
\begin{figure}[h]
\centering
\begin{subfigure}[h]{0.23\textwidth} 
  \includegraphics[width=\textwidth]{ICVS/33.png}
  \label{fig:MatchingMatrix}
    \caption{核化及标准化后的匹配矩阵}
\end{subfigure}
\begin{subfigure}[h]{0.23\textwidth} 
  \includegraphics[width=0.4\textwidth]{ICVS/Binary_im2bw_301p_600p_201p_700p-crop.pdf}
  \label{fig:Binary}
  \caption{二值化匹配矩阵}
\end{subfigure}
\begin{subfigure}[h]{0.23\textwidth} 
  \includegraphics[width=\textwidth]{ICVS/RANSAC_Binary_im2bw_2-crop.pdf}
  \label{fig:RANSAC}
    \caption{匹配矩阵中的最佳匹配查找}
\end{subfigure}
\label{fig:ImageProcessing}
\caption{Norland数据集秋-春跨季节匹配矩阵}
\end{figure}
我们选择了300张秋季图片作为训练地图,对春季进行视觉定位,如图\ref{fig:ImageProcessing}所示,
我们选择了一个300张秋季图片序列作为地图,由春季图像进行在线定位。我们可以看到,从AlexNet的Conv3中提取的特征并没有影响匹配结果。相反,如图\ref{fig:MatchingMatrix}所示,减少了背景信息的影响。图\ref{fig:Binary}是匹配图像的二值化结果。你可以看到,大部分干扰信息已经被擦掉了。%The capacity of restraining distractor has more important effect during robots localization. 
我们可以看到,在图\ref{fig:ransac}中,绿色的线只是这段时间的最佳匹配。当前图像在匹配矩阵中的最佳匹配特征为$\mathbf{f}_{km+b}$。那么当前图像在视觉图中的最佳匹配图像是$\mathbf{l}_{km+b}$。
在图中\ref{fig:ExperimentResult},我们绘制了3条线来评估我们方法的在3000张匹配图像上的实验结果。其中蓝线是图像索引Ground Truth,红线是匹配图像索引,黄线是匹配索引误差。


\begin{figure}[htbp]
 \centering
 \begin{subfigure}[h]{0.23\textwidth} 
  \includegraphics[width=\textwidth]{ICVS/spr_images-00285.png}
  \caption{场景1春天图像}
  \label{fig:RainNight}
 \end{subfigure}
 \begin{subfigure}[h]{0.23\textwidth} 
  \includegraphics[width=\textwidth]{ICVS/spr_images-00234.png}
  \caption{场景2春天图像}
 \end{subfigure}
 \begin{subfigure}[h]{0.23\textwidth} 
  \caption{场景3春天图像}
  \includegraphics[width=\textwidth]{ICVS/spr_images-00226.png}
 \end{subfigure}
 \begin{subfigure}[h]{0.23\textwidth} 
  \includegraphics[width=\textwidth]{ICVS/spr_images-00005.png}
  \caption{场景4春天图像}
\end{subfigure}
 \begin{subfigure}[h]{0.23\textwidth} 
  \includegraphics[width=\textwidth]{ICVS/images-00285.png}
  \caption{场景1秋天图像}
 \end{subfigure}
 \begin{subfigure}[h]{0.23\textwidth} 
  \caption{场景2秋天图像}
  \includegraphics[width=\textwidth]{ICVS/images-00234.png}
\end{subfigure}
 \begin{subfigure}[h]{0.23\textwidth} 
    \includegraphics[width=\textwidth]{ICVS/images-00226.png}
    \caption{场景3秋天图像}
  \end{subfigure}
 \begin{subfigure}[h]{0.23\textwidth} 
 \includegraphics[width=\textwidth]{ICVS/images-00005.png}
 \caption{场景4秋天图像}
 \end{subfigure}
 \caption{匹配易失败场景春秋对比图}
\end{figure}

\begin{figure*}[h]
 \centering
 \begin{subfigure}[h]{0.23\textwidth}
 \includegraphics[width=\textwidth]{ICVS/spring_8101_8400_fall_8001_8500_transform_33d_0_255.png}
 \caption{Spring images 8101-8400 with fall 8001-8500}
 \end{subfigure}
 \begin{subfigure}[h]{0.23\textwidth}
 \includegraphics[width=\textwidth]{ICVS/spring_6601_6900_fall_6501_7000_transform_33d_0_255.png}
 \caption{春季6601-6900序列-秋季6501-7000序列匹配矩阵}
 \end{subfigure}
 \begin{subfigure}[h]{0.23\textwidth}
 \includegraphics[width=\textwidth]{ICVS/spring_9301_9600_fall_9201_9700_transform_33d_0_255.png}
 \caption{春季9301-9600序列-秋季9201-9700序列匹配矩阵}
 \end{subfigure}
 \begin{subfigure}[h]{0.23\textwidth}
 \includegraphics[width=\textwidth]{ICVS/spring_9601_9900_fall_9501_10000_transform_33d_0_255.png}
 \caption{春季9601-9900序列-秋季9501-10000序列匹配矩阵}
 \end{subfigure}
 \caption{部分序列匹配矩阵}
\end{figure*}

\begin{figure*}[t]
  \centering
  \includegraphics[width=2.0\columnwidth]{ICVS/resultlines-crop.pdf}
  \caption{3000张图片匹配结果}  
  \label{fig:ExperimentResult}
\end{figure*}

\begin{figure}[htbp]
 \centering
 \begin{subfigure}[h]{0.23\textwidth}
    \includegraphics[width=\textwidth]{ICVS/images-02024.png}
 \end{subfigure}
 \begin{subfigure}[h]{0.23\textwidth}
    \includegraphics[width=\textwidth]{ICVS/images-02025.png}
 \end{subfigure}
 \begin{subfigure}[h]{0.23\textwidth}
    \includegraphics[width=\textwidth]{ICVS/images-02026.png}
 \end{subfigure}
 \begin{subfigure}[h]{0.23\textwidth}
    \includegraphics[width=\textwidth]{ICVS/images-02027.png}
 \end{subfigure}
 \caption{Norland数据集中隧道图像采集样例}
\end{figure}

\begin{figure}[t]
  \centering
  \includegraphics[width=0.3\textwidth]{ICVS/spring_1801_2100_fall_1701_2200_transform_33d_0_255.png} 
  \label{fig:Dark}
  \caption{春季数据集中包含隧道片段的匹配矩阵}
\end{figure}


\section{本章小结}

我们的论文提出了一种在动态环境中对机器人进行跨季节定位的新型算法。我们从AlexNet的Conv3中提取特征,该深度特征的匹配精度优于传统手工提取特征,通过PCA减少维度是一个新的尝试。

事实证明,Conv3是机器人本地化的最佳选择。相对于更底层的深度特征它具有更快的计算速度,并保留了一定语义信息,减少了图像混乱匹配。我们通过核化法将在线图像向量与地图向量逐一进行比较。我们从AlexNet的Conv3中提取特征,通过PCA减少维度特征的匹配精度仍优于传统手工提取特征。

这个过程扩大了正确匹配和错误匹配之间的差异。此外,通过适当阈值核化处理、图像放大和侵蚀,将复杂的数据关联图转化为简单的图像处理。在序列匹配方面,我们采用经典的RANSAC算法,在短时间内找到最佳匹配线。我们的实验结果表明,深度特征降维是加快计算速度和减少混乱匹配的好主意,且该算法对季节变换、动态环境、天气变化等都有很强的适应性。
本章算法的局限性主要受限制于图像采集设备。在完全黑暗的环境下,由于没有主动光源,图像信息很难表现出来,如图\ref{fig:Dark}所示。实际上在第1872至2016张图像中没有对应的匹配线。在AlexNet的Conv3中,对于难以表达深度特征的图像,匹配图像显示为黑色区域,我们将考虑加入激光的辅助。此外,我们还将研究特征维度与定位精度之间的更具体的影响关系。我们希望通过训练一个针对性的网络结构,得到不受季节变化、天气变化、动态环境等因素的影响的全局图像特征。





\begin{figure*}[h]
    \centering
    \begin{subfigure}[t]{0.23\textwidth}
    \includegraphics[width=\textwidth]{mvosr/fig_scale/orb_00_crop.pdf}
    \caption{00 ORB NO LC}
    \label{fig:orb_path_00}
    \vspace{4pt}
    \end{subfigure}
    \begin{subfigure}[t]{0.23\textwidth}
        \includegraphics[width=\textwidth]{mvosr/fig_scale/orb_00_17_crop.pdf}
        \caption{00 ORB NO LCx17}
        \label{fig:orb_path_00_17}
        \vspace{4pt}
        \end{subfigure}
        \begin{subfigure}[t]{0.23\textwidth}
            \includegraphics[width=\textwidth]{mvosr/fig_scale/scale_00_crop.pdf}
            \caption{00 Scale Parameter}
            \label{fig:scale_00}
            \vspace{4pt}
            \end{subfigure}
            \begin{subfigure}[t]{0.23\textwidth}
                \includegraphics[width=\textwidth]{mvosr/fig_scale/path_00_crop.pdf}
                \caption{00 Scale Recovery Path}
                \label{fig:scaled_path_00}
                \vspace{4pt}
                \end{subfigure}
    \begin{subfigure}[t]{0.23\textwidth}
    \includegraphics[width=\textwidth]{mvosr/fig_scale/orb_05_crop.pdf}
    \label{fig:orb_path_05}
    \caption{ 05 ORB NO LC}
    \vspace{4pt}
    \end{subfigure}
    \begin{subfigure}[t]{0.23\textwidth}
        \includegraphics[width=\textwidth]{mvosr/fig_scale/orb_05_17_5_crop.pdf}
        \label{fig:orb_path_05_175}
        \caption{05 ORB NO LC*17.5}
        \vspace{4pt}
        \end{subfigure}
        \begin{subfigure}[t]{0.23\textwidth}
            \includegraphics[width=\textwidth]{mvosr/fig_scale/scale_05_crop.pdf}
            \label{fig:scale_05}
            \caption{05 Scale Parameter}
            \vspace{4pt}
            \end{subfigure}
            \begin{subfigure}[t]{0.23\textwidth}
                \includegraphics[width=\textwidth]{mvosr/fig_scale/path_05_crop.pdf}
                \label{fig:scaled_path_05}
                \caption{05 Scale Recovery Path}
                \vspace{4pt}
                \end{subfigure}   
    \begin{subfigure}[t]{0.23\textwidth}
        \includegraphics[width=\textwidth]{mvosr/fig_scale/orb_08_crop.pdf}
        \label{fig:orb_path_08}
        \caption{08 ORB NO LC}
        \end{subfigure}
        \begin{subfigure}[t]{0.23\textwidth}
            \includegraphics[width=1.005\textwidth]{mvosr/fig_scale/orb_08_25_crop.pdf}
            \label{fig:orb_path_08_25}
            \caption{08 ORB NO LC x25}
            \end{subfigure}
            \begin{subfigure}[t]{0.23\textwidth}
                \includegraphics[width=\textwidth]{mvosr/fig_scale/scale_08_crop.pdf}
                \label{fig:scale_08}
                \caption{08 Scale Parameter}
                \end{subfigure}
                \begin{subfigure}[t]{0.23\textwidth}
                    \includegraphics[width=\textwidth]{mvosr/fig_scale/path_08_crop.pdf}
                    \label{fig:scaled_path_08}
                    \caption{08 Scale Recovery Path}
                    \end{subfigure}
    \caption{在KITTI数据集序列00、05、08上与无LC的ORB-SLAM2的尺度恢复性能比较。第一列中的三个数字是没有环路闭合的单目ORB-SLAM2轨迹,显然尺度发生了明显的错误。第二列为通过对应序列的前100帧尺度校正乘以17.0、17.5、25.0三个固定尺度参数得到的轨迹。第四列是第三列中乘以我们估计的尺度参数得到的轨迹。}
    \label{fig:scale_recovery}
\end{figure*}

\chapter{增量式定位--单目视觉里程计几何尺度估计}

近年来,人们提出了多种视觉里程表恢复方法。从\textbf{尺度scale}概念来看,解决思路大致可以分为两类:相对尺度校正和绝对尺度恢复。前者致力于将机器人自我运动保持在同一尺度下,以保持全局一致性;后者借助给定的绝对度量参考,计算每一帧的真实尺度。
%TMM摘要:
我们将固定在机器人上的相机至路面高度作为绝对尺度参考,提出了一种基于路面几何模型的单目视觉里程计尺度恢复算法。在从未探索过的环境种,单目视觉里程计是机器人实现自我定位和自主导航的核心模块,而尺度恢复是弥补单目视觉里程计不可或缺的功能,因为它弥补了相机所丢失的度量信息造成的尺度漂移等问题。当将相机高度视为绝对尺度参考系时,尺度恢复的精度取决于路面特征点的筛选和路面模型的建立。大多已有方法将这两个问题独立解决:他们的路面特征点筛选是基于路面颜色信息或者已知的图像固定区域,并没有利用两个方法的优势,将其进行有利结合。

如图\ref{fig:structure_mvosr}所示:单目图像序列、图像相对运动位姿($\mathbf{R}$和$\mathbf{\bar{t}}$)、匹配特征点作为输入,由VO初始化操作提供。每一帧都会被Delaunay三角剖分进行分割,相邻图像的匹配特征点即三角形的顶点,每个三角形都会通过深度一致性约束进行筛选以判断其是否属于路面。最终筛选得到的路面特征点会用来帮助恢复相机运动$s$。最终,相机绝对尺度下的运动位姿估计$\mathbf{R}$和$\mathbf{s\bar{t}}$得到了解决。在第一次Delaunay三角剖分后的结果图中(左中),所有的点都是通过初始视觉里程测量过程得到的匹配特征点,蓝色的点是选择的满足深度约束的特征点(如III-B1所述),红色的点是不满足深度约束的特征点;在第二次Delaunay三角剖分后的结果图中(左下),蓝色的点是满足道路模型约束的特征点(如III-B1所述),初步通过筛选的路面特征点被用于计算路面模型$\mathbf{n}^T_i\mathbf{x_i}-\bar{h}_{i}=0$,该模型又翻过来结合路面模型约束进行再一次的路面特征点筛选。

我们提出迭代求解道路点选择和道路几何模型计算:我们考虑估计的道路几何模型来检测道路点;考虑检测到的道路点在线更新道路几何模型。筛选出的路面特征点会用于估计路面几何模型,路面几何模型又是路面特征点筛选的一个几何约束;同时,我们采用路面几何信息代替路面颜色信息进行特征点筛选,使得系统更加鲁棒,这两个问题可以互相受益。 此外,对于道路点的选择,也采用了新的解决方案处理这个关键任务。 详细地,我们利用Delaunay三角剖分将图像分割成一组以匹配特征点为顶点的三角形。 每个三角形通过考虑深度和道路模型一致性两个约束条件来确定是否属于道路区域。 此外,我们通过随机样本共识(RANSAC)\cite{fischler1981random}估计具有验证道路点的几何道路模型,并通过中值滤波器去除尺度噪声。 对于道路建模,我们用不同的RANSAC参数检验了翻译误差。研究了中值滤波器和均值滤波器的比较。此外,在VO中,我们根据深度一致性和道路模型约束两个规则来选择道路点。综上所述,我们工作的主要贡献如下:
\begin{enumerate}
    \item {本章我们提出了一种基于路面几何信息的鲁棒单目视觉里程计尺度估计方法,将道路点选择和道路几何模型计算结合为一个问题————基于道路几何模型检测道路点,并根据检测到的道路点更新道路模型。}
    \item {本章提出了一种新的道路点选择策略,该策略受深度一致性和道路模型一致性的约束,并结合Delaunay三角剖分方法提出了该策略。该方法提高了MVO的精度,实现起来很简单。我们公开了可用的源代码\footnote{https://github.com/TimingSpace/MVOScaleRecovery}。}
\end{enumerate}

\begin{figure*}
    \centering
    \includegraphics[width=1.1\textwidth]{mvosr/method/structure_mvosr.pdf}
    \caption{基于路面几何模型的单目尺度恢复算法框架}
    \label{fig:structure_mvosr}
\end{figure*}

\begin{figure*}%[t]
    \centering
    \includegraphics[width=0.5\textwidth]{mvosr/method/RoadModel.pdf}
    \caption{路面模型法向量$n$与车辆平移矩阵$t$方向垂直。道路俯仰角是道路几何模型的一个中心单元,可以从自我运动来估计。 通过考虑道路几何模型。$\Theta_t$,$\Theta_n$ $h_0$ $n$ $\bar{t}$}
    \label{fig:structure}
\end{figure*}

\section{道路几何模型计算}
\label{sec:road_detection}
毫无疑问,图像特征点集中只有一部分属于道路区域,所以需要对属于路面的特征点进行筛选。道路区域检测的目的是计算包含计算高度摄像头的道路模型公式。本论文以道路几何结构的约束,而非顏色信息来侦测道路区域,因为几何结构比颜色信息更为鲁棒且几何模型在每一帧都会更新。

道路模型估计模块如图所示,\ref{fig:structure}。我们的道路点选择和道路几何模型计算是迭代进行的。它们可以相互受益。通过考虑道路几何模型来检测道路点,然后通过选择
的三维道路特征点来更新道路几何模型,经过验证的道路点用绿色标记。详细来说,我们提出利用初始运动来粗略估计道路特征的初始选择中的摄像机俯仰角。利用Delaunay三角剖分\cite{Shewchuk1996Triangle}
将已知三维坐标的特征点划分为多个三角形区域。然后,我们根据深度一致性(算法\ref{alg:depth_selection_1}和\ref{alg:depth_selection_4})剔除道路异常点,
被验证的道路点用蓝色标记,如图\ref{fig:structure}所示。然后对剩余的点再次使用Delaunay三角测量,我们根据道路模型的一致性继续剔除道路异常点(算法\ref{alg:flat_selection})。
\subsection{基于深度一致性的路面特征点筛选}
\label{sec:depth}
\subsubsection{直接剔除法}
\begin{algorithm}
    \caption{基于深度一致性的特征点筛除 (直接剔除法)}
    \KwIn{准路面特征点集$\Omega=\{f_0,f_1,....,f_n\}$,每个点像素坐标 $(u_i,v_i)$及其深度$\bar{d}_i$}
    \KwOut{认证路面特征点集$\Lambda \subset \Omega$}
    依据其像素位置,对特征点$\Omega$进行三角剖分,得到三角形集合$\nabla =\{\nabla_0,\nabla_1,...,\nabla_m\},%\quad 
    \nabla_{i}=\{f_i\in\Omega,\ f_j\in\Omega,\ f_k\in\Omega\}$\\
    设置$\Lambda = \Omega$\\
     \For{$t_i=0$ to $m$}
     {   
         \For {$\forall  \{f_i,f_j\} \subset \nabla_{i}$}
         {
             \If {$(v_i-v_j)(d_i-d_j)>0$}
             {
                 $\Lambda = \Lambda - \{f_i,f_j\}$
             }
         }
     }
  \label{alg:depth_selection_1}
\end{algorithm}
匹配的特征点的三维坐标在初始VO处理后就可以得到。它们与实测尺度的坐标保持相同的几何结构。
第$I_t$帧图像的初始特征点表示为$\Omega=\{f_0,f_1...f_{n}\}$。每个点的二维像素坐标为$\textbf{u}_i$=$(u_i,v_i)$,
相对尺度下的深度和三维坐标分别表示为$\bar{d}_i$和$\mathbf{\bar{x}}_i$。路点选择方法以$\textbf{u}_i$和$\bar{d}_i$为基础。
%\newcommand{\mysubfigure}[3][name]{\begin{subfigure}[h]{#2\textwidth}#3 \caption{#1}\end{subfigure}}
首先,根据特征点在帧$I_t$中的二维投影坐标$(u_i,v_i)$,用Delaunay三角测量法将特征点划分成一系列三角形区域$\nabla$,其顶点就是特征点。
如果一个特征点$f_i$满足路面几何模型,那么它的深度为
\begin{equation}
    \bar{d}_i = \frac{{\bar{h}}_i f_y} {v_i-c_y},
\end{equation}
所以我们可以得出结论:$\bar{d}_i\propto\frac{1} {v_i}$。此外,对于路面上的任意两个特征点
$f_i=(u_i, v_i, \bar{d}_i)$和$f_j=(u_j, v_j, \bar{d}_j)$,有以下关系必然成立:
\begin{equation}
    \sigma(i,j)= (v_i - v_j)(\bar{d}_i-\bar{d}_j)\leq 0. 
\label{eq:depth}
\end{equation}
如果$\sigma>0$,则至少有一个特征不属于道路,或者其深度$\bar{d}_i$不正确。对于这两种情况中的任何一种,我们选择将其排除。然而$f_i$和$f_j$中哪一个点应该
被删除是不确定的。我们提出了两种选择机制:一种是算法\ref{alg:depth_selection_1}中所示的直接删除法,另一种是算法\ref{alg:depth_selection_4}中所示的最大团和集成学习法。
\subsubsection{最大团和集成学习剔除法}
\begin{algorithm}
    \caption{基于深度一致性的路面特征点筛选(最大团筛选)}
    \KwIn{准路面特征点集$\Omega=\{f_0,f_1,....,f_n\}$, 每个点的像素坐标$(u_i,v_i)$及其深度$\bar{d}_i$}
    \KwOut{认证路面特征点集$\Lambda\subset\Omega$}
    设置准路面特征点属于路面的阈值$p_s$,根据像素坐标对准路面特征点集$\Omega$进行三角剖分
    得到三角形集合$\nabla =\{\nabla_0,\nabla_1,...,\nabla_m\}, \nabla_{i}=\{f_i\in\Omega,\ f_j\in\Omega,\ f_k\in\Omega\}$
    以$Omega$为节点$V$,即$V=\Omega$,根据$\nabla$生成无向$G={V, E}$,图中的最大团为$\nabla$ 
%    Set a threshold that the feature point belongs to the road $p_s$, triangulate features $\Omega$ according to its pixel coordinates $(u_i,v_i)$ 
%    to get triangles set $\nabla =\{\nabla_0,\nabla_1,...,\nabla_m\}, \nabla_{i}=\{f_i\in\Omega,\ f_j\in\Omega,\ f_k\in\Omega\}$
%    Generate undirected graph $G={V, E}$ according to $\nabla$, nodes are $V=\Omega$, and the largest clique set in the graph is $\nabla$ 
%    Set the side potential function $P_e$ as a $4\times2$ matrix
%    Set the maximum clique function $P_m$ as a $8\times8$ matrix
    \For{$\forall \nabla_i \in \nabla$}     %Set the vote of each point as 0 $vote(f_i)=0$, 
    {设置每个特征点的投票初始值为0: $\text{vote}(f_i)=0, \forall f_i \in \Omega$
        \For {$\forall\{f_i,f_j\} \subset \nabla_{i}$}
        {根据\eqref{eq:feature_poss}计算特征点属于最大团的概率
            \If {$p(f_i|\nabla_{i})\ge p_s$}
              {$\text{vote}(f_i)$=$\text{vote}(f_i)+1$}
            \Else
              {$\text{vote}(f_i)$=$\text{vote}(f_i)-1$}
        } 
    $\Lambda = \{f_i\}, \forall f_i \in \Omega$ and $p(\check{f_i})>0.5$
    }
    \label{alg:depth_selection_4}
\end{algorithm}
\begin{table}{|c|c|c|c|}
    \caption{两点之间(四种情形)的势函数定义。在第1-2列中,0表示该点被移除,1表示该点被保留。}
    \label{tab:max_clique}
    \centering
    \begin{tabular}{c c c c}
    \toprule
    \multirow{2}{*}{A} & \multirow{2}{*}{B}  &$P_e$&$P_e$ \\
     &&($\sigma\ge0$)&($\sigma\leq0$)\\
    \midrule
   0&0&3&1\\
   0&1&2&2\\
   1&0&2&2\\
   1&1&0&4\\
   \bottomrule
   \end{tabular} 
   \label{tab:Pe}
 \end{table}
算法\ref{alg:depth_selection_1}是一种最简单直接的算法,它将不满足方程\eqref{eq:depth}的点$f_i$和$f_j$都删除。此外,一个特征点可能存在于多个三角形中,
这可能会导致$\sigma$被重复计算,如图\ref{fig:max_clique}所示。为了避免冲突,我们提出了另一种基于最大团和集成学习法的特征点选择方法。此外,一个特征点可能存在于多个三角形中,这可能导致$\sigma$被重复计算。为了避免冲突,我们提出了另一种基于最大团和集成学习法的特征点选择方法。如图 \ref{fig:max_clique}所示,将每个三角形视为一个最大团,一个点可能存在于三个最大团中。这个点是否从路面特征点集合中删除由所有最大团的投票决定。每个最大团内的计算细节在算法\ref{alg:depth_selection_4}中描述,基于非定向图$G$,最大团$\nabla_{i}$中的特征点$f_i$属于道路区域的概率$p(f_i|\nabla_{i})$可以估计为条件概率分布。

\begin{equation}
    p(f_i|\nabla_{i}) =\frac{\sum_{\check{f_i},f_j,f_k}P_m(f_i,f_j,f_k|\sigma_{ij},\sigma_{jk},\sigma_{ik})}{\sum_{f_i,f_j,f_k}P_m(f_i,f_j,f_k|\sigma_{ij},\sigma_{jk},\sigma_{ik})}
    \label{eq:feature_poss}
\end{equation}
其中$\check{f_i}$表示某点属于道路。$P_m$为各最大组的势函数,其计算方法为:
\begin{equation}%\small %、scriptsize
    \begin{split}
        P_m(f_i,f_j,f_k|\sigma_{ij},\sigma_{jk},\sigma_{ik}) = &{P_e(f_i,f_j|\sigma_{ij})}   \\
        & \cdot{P_e(f_j,f_k|\sigma_{jk})} \cdot{P_e(f_i,f_k|\sigma_{ik})}
        \label{eq:clique_potential}
    \end{split}
\end{equation}
其中$P_e(f_i,f_j|\sigma_{ij})$表示当观测值为$sigma_{ij}$时,$P_e(f_i,f_j|\sigma_{ij})$的有效性概率,当观测值是$\sigma_{ij}$时,由公式\eqref{eq:depth}计算出来的。我们为图中的每一条边定义势函数$P_e(f_i,f_j|\sigma_{ij})$,如图\ref{fig:max_clique}中右侧表格所示。两点之间有四个删除或保留的决策,1表示倾向于保持,反之为0。在集成学习的思想下,每个最大组根据概率$p(f_i|\nabla_{i})$对特征点进行投票。只有在$\text{vote}(f_i)$中有一半以上的赞同率,$f_i$才会被验证为道路特征点。

\begin{figure}
    \centering
    \includegraphics[width=0.8\columnwidth]{mvosr/max_clique.pdf}
    \caption{Illustration of maximum clique.}
    \label{fig:max_clique}
\end{figure}
    
\subsection{基于路面模型一致性的特征点筛选}
\label{sec:RoadNormal}
我们根据道路模型的一致性,包括角度和距离的一致性,继续选择属于道路的三角形。由于道路模型约束选择是深度一致性选择之后的步骤,我们选择算法\ref{alg:depth_selection_4}
的输出作为算法\ref{alg:flat_selection}的输入。我们用Delaunay三角测量法再次对剩余的点进行三角测量,即Delaunay三角测量法\cite{Shewchuk1996Triangle}。每个
三角形的平面三维几何模型可以通过以下方式求解:
\begin{equation}
    \mathbf{n_i}^T[\mathbf{x}_{i1},\mathbf{x}_{i2},\mathbf{x}_{i3}] -[\bar{h_i},\bar{h_i},\bar{h_i}]^{T}=[0,0,0]^{T},
    \label{eq:road_model_1}
\end{equation}
上述方程的解是:
\begin{equation}
    \mathbf{\bar{n_i}} = [\mathbf{x}_{i1}\\,\mathbf{x}_{i2}\\,\mathbf{x}_{i3}]^{-1}[1,1,1]^{T}
    \label{eq:road_model_2}    
\end{equation}

\begin{equation}
   \begin{cases}
       \mathbf{n}_{i} =\frac{\mathbf{\bar{n}}_{i}}{\|\mathbf{\bar{n}}_{i}\|},\\
       \bar{h}_{i} =\frac{1}{\|\mathbf{\bar{n}}_{i}\|}.\\
   \end{cases}
   \label{eq:road_model_3}
\end{equation}

我们在方程\eqref{eq:road_model_1}中再增加两个约束条件,$||n_{i}||=1$和$n_{i_{y}}>0$,得到唯一解。在得到每个三角形区域的几何结构
$\mathbf{n}_{i}^T\mathbf{x}-h_{i}=0$后,我们先根据角度约束选择三角形,再根据距离约束选择。初始,我们假设摄像头安装时是向前看的,三角形应该位于摄像头下方,
所以直接删除$bar{h}_i<0$的点。

首先,考虑到移动的连续性,连续帧的法向量应相似,所以只保留法向量与前一帧路面法向量接近的三角形。,我们将三角形区域的几何模型与前一帧的道路模型进行比较,
由摄像机运动$\mathbf{R}$和$\mathbf{bar{t}}$计算出的法线应该接近估计的道路法线,$n_t$。
%begin{quation} .
%label{eq:d_normal}.
% d_{ij}=\arccos{\mathbf{n}_{i}\cdot\mathbf{n}_{j}}.%|\arccos{\mathbf{n}\cdot\mathbf{n}_{i}}|
%$end{equation}
每个三角形区域法线的俯仰角可以通过以下方式计算出来:
\begin{equation}
    \theta_i =\arcsin{(-\frac{n_{i2}}{|\mathbf{n}_{i}|})}
    \label{eq:pitch_n_triangle}
\end{equation}
其中$n_{i2}$表示$\mathbf{n}_i$的第二个元素。然后将$\theta_{i}$与道路正常向量的角度$\theta_{r}$进行比较。一般来说,$\theta_{r}$可以直接从最后一帧的道路
正常向量中估算出来。但是,对于最初的几帧,道路的法向量是未知的,我们必须从运动的$\mathbf{\bar{t}}$来计算。
严格来说,$\theta_\mathbf{\bar{t}}$只有在摄像机俯角$\theta_{\mathbf{R}}$小于某个阈值时才有效,所以这两个变量都需要计算。
\begin{equation}
    \theta_\mathbf{\bar{t}} =
    \begin{cases}
        \arcsin{(-\frac{t_{2}}{|\mathbf{\bar{t}}|}}) &\text{if}\quad |\mathbf{\bar{t}}|\neq 0,\\
        \text{NaN} &\text{if}\quad |\mathbf{\bar{t}}| = 0.\\
    \end{cases}
    \label{eq:pitch_t}
 \end{equation}
当$|t|=0$时, 说明机器人无运动,无需恢复尺度。$\theta_{\mathbf{R}}$绝对值是
\begin{equation}
    \label{eq:pitch_r}
     |\theta_{\mathbf{R}}| = 
    \begin{cases}
         |\arctan{(-\frac{\mathbf{R}_{32}}{\mathbf{R}_{33}})}| & \text{if}\quad\mathbf{R}_{33}\neq 0,\\
        \frac{\pi}{2}&\text{if}\quad\mathbf{R}_{33} = 0.\\
    \end{cases}
 \end{equation}
%两个法向量之间的关系可以根据它们的俯仰角来分析。
%两个法向量$\mathbf{n}_{i}$和$\mathbf{n}_{j}$之间的角差也可以通过俯仰角来测量。
如果$\theta_{\mathbf{R}}$足够小,那么道路法线的俯仰角被估计为$\theta_r=\theta_\mathbf{\bar{t}}-\frac{\pi}{2}$,因为当车辆在道路上行驶时,运动矢量$\mathbf{\bar{t}}$与路面相切,与道路法线正交,如图所示。只有满足以下条件:
\begin{equation}
    || \theta_r- \theta_i || < \theta_0 %threshold
\end{equation}
的点将被保留。这里$\theta_0$是角隙阈值,我们在实验中设置为10度。
\begin{algorithm}
    \caption{基于路面模型的特征点选择}
    \KwIn{准路面特征点集$\Omega=\{f_0,f_1,....,f_n\}$,每个点的像素坐标$(u_i,v_i)$,相机坐标系下的点坐标$(\bar{x}_i,\bar{y}_i,\bar{z}_i$),路面模型$\mathbf{n}^T_{i}\mathbf{x}_{i}-\bar{h}_{i}=0$,运动向量$\mathbf{\bar{t}}$\\ }
    \KwOut{认证路面特征点集$\Gamma \subset \Omega$} %$\Gamma \subset \Lambda$}
    根据像素坐标对准路面特征点集$\Omega$进行三角剖分, 得到三角形集合$\nabla=\{\nabla_0,\nabla_1,...,\nabla_m\},\nabla_{i}=\{ f_i \in \Omega,\ f_j \in \Omega,\ f_k \in \Omega\}$\\
%    \If {the last road model is unknown} 
    \If {上一帧的路面模型未知} 
        {$\theta_r=\theta_\mathbf{\bar{t}}-\frac{\pi}{2}$,$\theta_\mathbf{\bar{t}}$是$\mathbf{\bar{t}}$的仰角初始值\\}
        将认证路面三角形集初始化为空集$\Theta =\varnothing$\\
    \For{$\forall\nabla_i\in\nabla$}
     {   
        计算相机高度$\bar{h}_i$,三角形$\nabla_i$的法向量$n_i$和仰角$\theta_i$\\
         %Calculate the normal $n_i$ and distance of the triangle $\nabla_i $ and height $\bar{h}_i $\\
         %Calculate triangle pitch angle $\theta_i $\\
        \If {$||\theta_r- \theta_i|| < 10$}%{$\theta_i <\theta_r-80$}
         {
             $\Theta =\Theta + {\nabla_i}$
         }
     }
         %$Calculate the upper half of the effective triangle distance $h_t$
         
         Calculate the median of the distance between effective triangles and camera optical center $h_t$\\
    \For{$\forall\nabla_i\in\Theta$}
     {   
        \If {$\bar{h}_{i} <h_t$}
         {
             $\Theta =\Theta - {\nabla_i}$
         }
     }
     $\Gamma =\{f_k\}, \forall f_k\in \nabla_i, \forall\nabla_i\in\Theta$
\label{alg:flat_selection}
\end{algorithm}

其次,我们继续以摄像机光学中心与三角形区域的距离为约束进行选择。根据道路点相对较低的约束条件,
选择三角形有效距离的中值作为阈值,将高于该阈值的点视为道路点。利用所选三角形区域$\Gamma$的顶点更新路面几何模型。具体过程如算法\ref{alg:flat_selection}所示。

\subsection{路面模型与绝对尺度计算}
\label{sec:road_model}
%todo{添加关于RANSAC和过滤器的公式}。
在本节中,只对第\ref{sec:road_detection}节中两次选取后存活下来的特征点进行验证,估计道路几何模型和尺度。我们主要利用平面拟合的方法来确定相对尺度的相机高度,
并利用中值滤波来降低尺度噪声。我们假设道路是一个平面,在帧$I_t$中的几何模型可以表示为:
\begin{equation}
    \mathbf{n}^T_i\mathbf{x_i}-\bar{h}_{i}=0,
    \label{eq:road_model}
\end{equation}
其中$\mathbf{n}_i$为道路平面法线,$\bar{h}_{i}$为计算高度。比例尺可以用以下方法恢复:
\begin{equation}
    s_i = h_0/\bar{h}_i,
    \label{eq:scale}
\end{equation}
其中$h_0$是给定的摄像机安装高度。应用RANSAC方法\cite{FISCHLER1981Random}对经过验证的道路点进行道路平面估计。如果选择的特征点数量小于一个阈值,我们跳过
RANSAC步骤,并保持道路几何模型与上次验证的帧相同。我们在实验中设置这个阈值为12。根据车辆速度不会发生剧烈变化的假设,我们在时间维度上对尺度进行过滤,以削弱尺度
噪声的影响。直接采用独立于噪声模型的中值滤波器来消除噪声,即用前$q$帧估计的尺度系数的中值作为尺度系数。
RANSAC不同参数下所获得的性能以及不同滤波器大小的影响将在第\ref{sec:parameter_select}节中进行分析。

\begin{equation}
    s_i = median({s_{i-q+1},s_{i-q+2}...,s_{q}})
    \label{eq:scale_median}.
\end{equation}
由此得到绝对尺度下的位移矩阵是:
\begin{equation}
    \mathbf{t} = s_i\bar{\mathbf{t}}.
    \label{eq:absolute_t}
\end{equation}
最后,计算出机器人的绝对运动估计得到解决$\mathbf{R}$和$\mathbf{t}$。

\section{KITTI数据集单目视觉里程计实验}
我们在常用的公开测评数据集KITTI\cite{geiger2012we}上评估了我们的视觉里程计尺度恢复方法,可用于评估视觉里程计算方法的准确性,是目前应用最广泛的测试环境之一。它由22个序列组成,覆盖了城市、村庄、高速公路等环境,运行长度从数百米到数公里不等。其中,前11个序列提供了真实的运动轨迹。但我们忽略序列01,因为大多数VO方法在这种高速场景下无法提供满意的初始结果。
此外,我们的主要评估标准是相对姿态误差(RPE)\cite{geiger2012we}和绝对轨迹误差(ATE)\cite{sturm2012benchmark}。RPE测量每个序列中每个固定距离段的$\mathbf{R}$和$\mathbf{t}$的平均相对误差。ATE计算$\mathbf{t}$的绝对误差,这些指标可以通过预处理相似性转换\cite{raul2015orb}来评估尺度漂移消除性能。我们用Python实现了我们的算法,源代码是公开的。所有的实验都是在英特尔酷睿i5,2.7GHz,使用单线程进行的。

我们的实验由三部分组成。首先,我们将我们的单目尺度恢复方法与其他简单开源的VO/SLAM方法结合起来,定量和定性地测试对它们性能改进。其次,设置MonoVO\footnote{https://github.com/uoip/monoVO-python}和ORB-SLAM2 \cite{raul2015orb}作为我们的初始自我运动估计,将我们的方法与四个最先进的VO算法进行比较。最后,给出了每个模块的性能分析和参数探索。

\subsection{对现有单目视觉里程计开源算法的改进}
为了展示比例尺恢复方法的性能,我们将我们的比例尺校正方法移植到ORB-SLAM2、LibVISO \cite{Geiger2011IV}和MonoVO等三种基于特征的开源定位算法上,并将它们的性能分别与原方法进行比较。

\subsubsection{对ORB-SLAM2视觉里程计性能的提升}
\label{sec:eva_scale_recovery}

\begin{figure*}[t]
    \centering
    \begin{subfigure}[b]{0.23\textwidth}
    \includegraphics[width=\textwidth]{figures/mvosr/fig_scale/orb_00.pdf}
    \caption{00 ORB NO LC}
    \vspace*{1mm}
    \label{fig:orb_path_00}
    \end{subfigure}
    \begin{subfigure}[b]{0.23\textwidth}
        \includegraphics[width=\textwidth]{figures/mvosr/fig_scale/orb_00_align.pdf}
        \caption{00 ORB NO LC Aligened}
        \label{fig:orb_path_00_17}
        \vspace*{1mm}
        \end{subfigure}
        \begin{subfigure}[b]{0.23\textwidth}
            \includegraphics[width=\textwidth]{figures/mvosr/fig_scale/00_scale.pdf}
            \caption{00 Scale Parameter}
            \label{fig:scale_00}
            \vspace*{1mm}
            \end{subfigure}
            \begin{subfigure}[b]{0.23\textwidth}
                \includegraphics[width=\textwidth]{figures/mvosr/fig_scale/orb_00_rescale.pdf}
                \caption{00 Scale Recovery Path}
                \label{fig:scaled_path_00}
                \vspace*{1mm}
                \vspace*{1mm}
                \end{subfigure}
    \begin{subfigure}[b]{0.23\textwidth}
        \includegraphics[width=\textwidth]{figures/mvosr/fig_scale/orb_02.pdf}
        \caption{02 ORB NO LC}
        \vspace*{1mm}
        \label{fig:orb_path_02}
        \end{subfigure}
        \begin{subfigure}[b]{0.23\textwidth}
            \includegraphics[width=\textwidth]{figures/mvosr/fig_scale/orb_02_align.pdf}
            \caption{02 ORB NO LC Aligned}
            \label{fig:orb_path_02_aligned}
            \vspace*{1mm}
            \end{subfigure}
            \begin{subfigure}[b]{0.23\textwidth}
                \includegraphics[width=\textwidth]{figures/mvosr/fig_scale/02_scale.pdf}
                \caption{02 Scale Parameter}
                \label{fig:scale_02}
                \vspace*{1mm}
                \end{subfigure}
                \begin{subfigure}[b]{0.23\textwidth}
                    \includegraphics[width=\textwidth]{figures/mvosr/fig_scale/orb_02_rescale.pdf}
                    \caption{02 Scale Recovery Path}
                    \label{fig:scaled_path_02}
                    \vspace*{1mm}
                    \vspace*{1mm}
                    \end{subfigure}
    \begin{subfigure}[b]{0.23\textwidth}
    \includegraphics[width=\textwidth]{figures/mvosr/fig_scale/orb_05.pdf}
    \label{fig:orb_path_05}
    \caption{ 05 ORB NO LC}
    \vspace*{1mm}
    \end{subfigure}
    \begin{subfigure}[b]{0.23\textwidth}
        \includegraphics[width=\textwidth]{figures/mvosr/fig_scale/orb_05_align.pdf}
        \label{fig:orb_path_05_175}
        \caption{05 ORB NO LC Aligned}
        \end{subfigure}
        \begin{subfigure}[b]{0.23\textwidth}
            \includegraphics[width=\textwidth]{figures/mvosr/fig_scale/05_scale.pdf}
            \label{fig:scale_05}
            \caption{05 Scale Parameter}
            \end{subfigure}
            \begin{subfigure}[b]{0.23\textwidth}
                \includegraphics[width=\textwidth]{figures/mvosr/fig_scale/orb_05_rescale.pdf}
                \label{fig:scaled_path_05}
                \caption{05 Scale Recovery Path}
                \vspace*{1mm}
                \end{subfigure}   
    \begin{subfigure}[b]{0.23\textwidth}
        \includegraphics[width=\textwidth]{figures/mvosr/fig_scale/orb_08.pdf}
        \label{fig:orb_path_08}
        \caption{08 ORB NO LC}
        \end{subfigure}
        \begin{subfigure}[b]{0.23\textwidth}
            \includegraphics[width=1.005\textwidth]{figures/mvosr/fig_scale/orb_08_align.pdf}
            \label{fig:orb_path_08_25}
            \caption{08 ORB NO LC Aligned}
            \end{subfigure}
            \begin{subfigure}[b]{0.23\textwidth}
                \includegraphics[width=\textwidth]{figures/mvosr/fig_scale/08_scale.pdf}
                \label{fig:scale_08}
                \caption{08 Scale Parameter}
                \end{subfigure}
                \begin{subfigure}[b]{0.23\textwidth}
                    \includegraphics[width=\textwidth]{figures/mvosr/fig_scale/orb_08_rescale.pdf}
                    \label{fig:scaled_path_08}
                    \caption{08 Scale Recovery Path}
                    \vspace*{1mm}
                    \end{subfigure}
    \caption{在KITTI数据集序列00、02、05、08上与无LC的ORB-SLAM2的尺度恢复性能比较。第一列中三个图像是没有环路闭合检测的单目ORB-SLAM2轨迹,显然尺度发生了明显的错误。第二列为对应序列的前100帧通过7-dof尺度校正得到的轨迹。第四列是第一列图像与我们估计的尺度参数(第三列)相运算得到的轨迹。}
    {\label{fig:scale_recovery}}
\end{figure*}

我们将我们的方法与ORB-SLAM2(无LC)进行定性比较,如图\ref{fig:scale_recovery}和图\ref{fig:orb_noLC_scale_recovery}所示,并与ORB-SLAM2(有LC)进行定量比较,如表\ref{tab:orb_scale_drift_LC}所示。

首先,我们将我们的算法添加到单目ORB-SLAM2(无LC)中,以比较固定和计算尺度的轨迹。在计算ORB-SLAM2中摄像机的初始运动之前,我们关闭全局优化和环路闭合检测,以避免额外的尺度漂移优化。

%考虑到车辆运行时间较长,面临较明显的尺度漂移问题,我们选择了KITTI数据集的00(3.7 km)、05(2.2 km)和08(2.8 km)三个长距离序列。如图:\ref{fig:scale_recovery}b、\ref{fig:scale_recovery}f和\ref{fig:scale_recovery}j所示,虽然我们将三个固定的尺度参数与第一列相乘,但它们仍然存在严重的尺度漂移问题,尤其是在后期。因此,用固定比例尺恢复轨迹是不可行的。这三个比例系数17.0、17.5、25.0,分别与序列00、05、08的前100帧对齐计算。图中第三列\ref{fig:scale_recovery}显示了我们估计的尺度参数和地面真实尺度参数的对比。地面真实参数的计算方法是:

考虑到较长的车辆运行面临较明显的尺度漂移问题,选择了KITTI数据集中的00(3.7公里)、02(5.1公里)、05(2.2公里)和08(2.8公里)四个长距离序列。从图\ref{fig:scale_recovery}b,\ref{fig:scale_recovery}f,\ref{fig:scale_recovery}j和\ref{fig:scale_recovery}n中可以看出,虽然我们对第一列的初始路径进行了对齐,但仍然存在严重的尺度漂移问题,尤其是在后期。因此,用固定比例尺恢复轨迹是不可行的。图中第三列\ref{fig:scale_recovery}显示了我们估计的尺度参数和地面真实尺度参数的对比。地面真值尺度参数的计算方法为
\begin{equation}
    s_g = \frac{||\mathbf{t_g}||}{||\mathbf{\bar{t}}_i||}
    %    s_g = \frac{\sum_{i=1}^{3}\frac{t_i}{\bar{t}_i}t_i}{\sum_{i=1}^{3}{t_i}},
\end{equation}
其中$\mathbf{t_g}$是以自我运动估计位移的ground-truth。估计的尺度参数由中值滤波器过滤。通过将我们估计的尺度参数与初始的相对位移矩阵相乘,我们得到了绝对尺度下的自我运动估计结果,如图\ref{fig:scale_recovery}d,\ref{fig:scale_recovery}h,\ref{fig:scale_recovery}l和\ref{fig:scale_recovery}所示。加入我们的尺度漂移消除模块后,定性比较来看机器人移动轨迹与ground-truth更接近。
%图\ref{fig:orb_noLC_scale_recovery}中收集了更多由ORB-SLAM2(不含LC)和我们的尺度恢复算法组合得到的轨迹。考虑到ORB-SLAM2在序列02中的第2005帧(共4661帧)(图\ref{fig:orb_noLC_path_02})、08中的第3539帧(共4071帧)(图\ref{fig:orb_noLC_path_08})和09中的第761帧(共1591帧)(图\ref{fig:orb_noLC_scale_recovery}i)发生丢失,图示这些轨迹只是丢失前的部分。

\begin{figure*}{h}
    \centering
    \begin{subfigure}[c]{0.22\textwidth}
        \includegraphics[width=\textwidth]{mvosr/path_fig/00.pdf}
        \caption{00}
        \vspace*{2mm}
        \label{fig:orb_noLC_path_00}
    \end{subfigure}
    \begin{subfigure}[c]{0.22\textwidth}
        \includegraphics[width=\textwidth]{mvosr/path_fig/02.pdf}
        \caption{02}
        \label{fig:orb_noLC_path_02}
        \vspace*{2mm}
    \end{subfigure}
    \begin{subfigure}[c]{0.22\textwidth}
            \includegraphics[width=\textwidth]{mvosr/path_fig/03.pdf}
        \caption{03}
        \label{fig:orb_noLC_path_03}
        \vspace*{2mm}
    \end{subfigure}
    \begin{subfigure}[c]{0.22\textwidth}
        \includegraphics[width=\textwidth]{mvosr/path_fig/04.pdf}
        \caption{04}
        \label{fig:orb_noLC_path_04}
        \vspace*{2mm}
    \end{subfigure}
    \begin{subfigure}[c]{0.22\textwidth}
        \includegraphics[width=\textwidth]{mvosr/path_fig/05.pdf}
        \caption{05}
        \label{fig:orb_noLC_path_05}
        \vspace*{2mm}
    \end{subfigure}
    \begin{subfigure}[c]{0.22\textwidth}
        \includegraphics[width=\textwidth]{mvosr/path_fig/06.pdf}
        \caption{06}
        \label{fig:orb_noLC_path_06}
        \vspace*{2mm}
        \end{subfigure}
    \begin{subfigure}[c]{0.22\textwidth}
        \includegraphics[width=\textwidth]{mvosr/path_fig/07.pdf}
        \caption{07}
        \label{fig:orb_noLC_path_07}
        \vspace*{2mm}
    \end{subfigure}
    \begin{subfigure}[c]{0.22\textwidth}
        \includegraphics[width=\textwidth]{mvosr/path_fig/08.pdf}
        \caption{08}
        \label{fig:orb_noLC_path_08}
        \vspace*{2mm}
    \end{subfigure}
    \begin{subfigure}[c]{0.22\textwidth}
        \includegraphics[width=\textwidth]{mvosr/path_fig/09.pdf}
        \caption{09}
        \label{fig:orb_noLC_path_09}
        \vspace*{2mm}
    \end{subfigure}
    \begin{subfigure}[c]{0.22\textwidth}
        \includegraphics[width=\textwidth]{mvosr/path_fig/10.pdf}
        \caption{10}
        \label{fig:orb_noLC_path_10}
        \vspace*{2mm}
    \end{subfigure}   
    \caption{}
\label{fig:orb_noLC_scale_recovery}
\end{figure*}

以ORB-SLAM2(无LC)的初始运动作为基准,在KITTI数据集序列00和02-10上得到的视觉里程计轨迹。每个子图中的红线是轨迹ground-truth,蓝线是经过我们的绝对尺度恢复方法校准后的轨迹。

其次,我们对ORB-SLAM2(带LC)和我们的方法进行了性能量化比较,说明我们的尺度恢复算法在消除尺度漂移方面效果更好。评价标准是ATE,如表\ref{tab:orb_scale_drift_LC}所示。使用ATE的原因是ORB-SLAM2的单目版不提供尺度计算,不能直接计算RPE。%ATE在评估前将轨迹对准绝对尺度,以评估尺度模糊的影响。
具有7-自由度对齐(平移、旋转和尺度)的ATE误差可以在评估前在绝对尺度上对齐轨迹,以评估尺度模糊性的影响。

在大多数序列中,我们的VO效果优于ORB-SLAM2(有LC模块)的效果。对于小尺度的轨迹,如04、06和07,我们的方法和ORB-SLAM2与LC之间的误差差距很小。但是,由于我们的算法可以恢复绝对尺度,消除尺度漂移,所以在所有长轨迹上都可以达到满意的误差。

\begin{table}
    \caption{ORB-SLAM2中,采用尺度漂移与环路闭合检测两种方法时对绝对平移误差(ATE)的影响。}
    \label{tab:orb_scale_drift_LC}
    \centering
    \begin{tabular}{l c c c c}
    \toprule
    \multirow{3}*{Seq}&Running distance&ORB+LC&ORBnoLC+Our SR  \\
        &   & \cite{raul2015orb} &\\
     &(m$\times$m)&ATE(m)&ATE(m)\\
 \midrule
 00&3724&6.68 &\textbf{5.56}\\
 02&5067&21.75&\textbf{4.36}\\
 03&561&1.59&\textbf{1.36}\\
 04&394&\textbf{1.79}&2.36\\
 05&2206&8.23&\textbf{8.03}\\
 06&1233&\textbf{14.68}&18.36\\
 07&695&\textbf{3.36}&5.16\\
 08&3223&46.58&\textbf{5.64}\\
 09&1705&7.62&\textbf{2.53}\\
 10&920&8.68&\textbf{2.33}\\
 \midrule
 Average&1972.80&12.10&\textbf{5.56}\\
 \bottomrule
 \end{tabular}
 \end{table}
 
\subsubsection{对LIBVISO2性能的提升}
LibVISO2是一个开源的C++库,用于测量视觉里程计。它通过双核和非最大抑制的响应提取特征。根据这三个点,计算出它们的矩阵旋转矩阵$R$和位移矩阵$t$来表示两帧之间的姿态变换。它提供了一种和我们一样基于局部平面地面和摄像机高度的简单比例计算方法,
我们用本文提出的比例恢复算法来代替它,以测试对LibVISO2单目版本算法的改进。为方便起见,本文将LibVISO2使用pybind11封装成Python库\footnote{https://github.com/SummerHuiZhang/Libviso-Python}。
如表\ref{tab:orb_scale_drift_LibVISO}所示,通过与我们的尺度恢复算法相结合,LibVISO2的性能得到了极大的提升。以序列00为例,加入我们的尺度恢复算法后,RPE从9.77\%下降到6.51\%。在序列02、03、04和10中,
这种改进尤为明显。平均RPE从14.18\%下降到7.11\%。且当车头通常被其他车辆挡住时,LibVISO2很难运行,如序列07中出现的情况。
\begin{table}
    \caption{Improvement on LibVISO2.}
    \label{tab:orb_scale_drift_LibVISO}
    \centering
\begin{tabular}{l c c c c c}
\toprule
\multirow{3}*{Seq}&Running distance&Rotation error&LibVISO2&LibVISO2+Our SR\\
       &&\cite{raul2015orb}&\cite{Geiger2011IV}\\
       &(m$\times$m)&RPE(deg/m)&RPE(\%)&RPE(\%)\\
\midrule
00&3724&0.0267&9.77 &\textbf{6.51}\\
02&5067&0.0136&16.40&\textbf{3.89}\\
03&561&0.0200&22.85&\textbf{4.81}\\
04&394&0.0292&18.79&\textbf{6.70}\\
05&2206&0.0382&12.22&\textbf{12.0}\\
06&1233&0.0255&9.42&\textbf{9.19}\\
%07&694&3.36&\textbf{5.16}\\
08&3223&0.0223&9.60&\textbf{7.83}\\
09&1705&0.0143&9.82&\textbf{5.20}\\
10&920&0.0378&18.70&\textbf{7.86}\\
\midrule
Average&1903.30&0.0253&14.18&\textbf{7.11}\\
\bottomrule
\end{tabular}
\end{table}

\subsubsection{对MonoVO性能的提升}
MonoVO是一个简单的基于OpenCV的开源MVO项目,它用FAST描述符\cite{Rosten2006Machine}提取特征,并用光流跟踪它们。MonoVO确实提供了一个方便的五点运动估计,但它缺乏尺度计算。
因此,我们将我们的尺度恢复算法与原始MonoVO和带有特征稀疏化的MonoVO结合起来,测试我们的方法在这种简单VO方法上的性能。

\begin{table}[h]
    \caption{Improvement on MonoVO}
    \label{tab:orb_scale_drift_MonoVO}
\begin{center}
\setlength{\tabcolsep}{2mm}{
\begin{tabular}{l c c c c c c}
\toprule
\multirow{3}*{Seq}&Rotation error & MonoVO &MonoVO-ROI &MonoVO-SR\\
       &\cite{raul2015orb}&&&\\
       &RPE(deg/m)&RPE(\%) & RPE(\%)&RPE(\%)\\
\midrule
00&0.0046&36.44&20.32&\textbf{2.51}\\
02&0.0059&46.19&7.28&\textbf{2.33}\\
03&0.0048&46.19&12.06&\textbf{5.65}\\
04&0.0036&55.92&14.12&\textbf{2.40}\\
05&0.0316&35.43&27.93&\textbf{8.48}\\
06&0.0057&22.94&17.14&\textbf{2.32}\\
%07&694&30.92&27.20&\textbf{18.19}\\
08&0.0072&32.47&16.27&\textbf{3.05}\\
09&0.0062&33.50&12.26&\textbf{2.15}\\
10&0.0162&28.32&20.00&\textbf{4.92}\\
\midrule
Average&0.0096&37.50&16.38&\textbf{3.76}\\
\bottomrule
\end{tabular}
}
\end{center}
\end{table}

结合我们的尺度恢复算法(称为MonoVO-SR),对原始MonoVO的改进如表\ref{tab:orb_scale_drift_MonoVO}\ref{tab:orb_scale_drift_MonoVO_M}所示。MonoVO-ROI和MonoVO-SR都是MonoVO和本文提出的方法的组合,不同的是MonoVO-SR的性能比原来的MonoVO和MonoVO-ROI都要好,MonoVO-ROI假设ROI是固定的道路。以序列00为例,MonoVO和MonoVO-ROI的RPE分别为36.44%和20.32%。加入我们的规模恢复算法后,降低到2.51\%。改善明显,平均RPE从37.75\%急剧下降到3.76\%。

此外,我们观察到,分散的特征点可以进一步帮助提高MonoVO的性能。我们将原始MonoVO的特征点分散开来,并在每个切割的$30\times30$像素区域随机选择两个点。然后我们在分散的MonoVO上加入我们的尺度恢复方法(称为ST-MonoVO-SR),
结果如表\ref{tab:orb_scale_drift_MonoVO_M}所示。我们可以看到,ST-MonoVO-SR在所有测试序列上都优于特征稀疏化的MonoVO,平均RPE从10.30%提高到2.13%。此外,从表\ref{tab:orb_scale_drift_MonoVO}和
表\ref{tab:orb_scale_drift_MonoVO_M}的比较中,我们也可以得出结论,分散的特征点避免了很多模糊性,确实有助于提高MonoVO的性能。ST-MonoVO-SR将是我们与其他机器人自我运动估计算法比较时的系统之一。

%\subsection{与其他VO算法的比较}
\label{sec:overall_evaluation}
\begin{figure}[t]
    \centering
    \includegraphics[width=0.98\columnwidth,height=0.4\columnwidth]{mvosr/method/feature_point_vert_dis_3.pdf}
    \caption{算法1-3对KITTI序列00的前500帧特征点分布的初步测试。绿线表示深度一致性选择(算法1和2)后的点分布。黄线表示深度和道路模型一致性选择后的点分布(算法1-3)。}
    \label{fig:GlobalDistribution}
\end{figure}
我们将我们的比例尺恢复方法与散射的MonoVO(ST-MonoVO-SR),以及没有LC的ORB-SLAM2(ORBSLAM2-SR)结合起来,并与四种最先进的视觉里程计方法,在KITTI数据集的序列00和02-10上进行比较,他们分别来自\cite{Song2015MoncularScale}、\cite{Geiger2011IV}、
\cite{Lee2015MoncularScale}和\cite{zhou2016reliable}。如表\ref{tab:kitti_compare}所示,我们算法的平均误差低于其他单目尺度恢复算法,与LibVISO2-stereo算法相当。我们的方法也优于在大多数序列上失败的\cite{zhou2016reliable}的方法。
虽然Lee\cite{Lee2015MoncularScale}提出的方法在序列01(一个高速场景)上有效,但它在其他序列上的表现比我们的差。\cite{Song2015MoncularScale}的算法假设一个ROI作为路面,所以当车前被其他物体挡住时,它很难计算出路面的几何模型,就像序列07中发生的那样。但是,当使用ORB-SLAM2作为我们的前端时,我们仍然可以在序列07中通过在线识别路面面积获得相对稳定的结果1.73\%。

值得强调的是,虽然ORB-SLAM2-SR(ORB-SLAM2和本方法的组合)的性能更好,但ST-MonoVO-SR(MonoVO和本方法的组合)更能证明本方法的效果。MonoVO原本方法比ORB-SLAM2简单得多且精度较差,我们的尺度恢复方法大大提高了它的精度。

\begin{figure}[t]
    \centering
    \begin{subfigure}[c]{0.95\textwidth}
    \includegraphics[width=\textwidth]{mvosr/method/feature_sample_dis_good.pdf}
    \caption{}
    \end{subfigure}
    \begin{subfigure}[c]{0.95\textwidth}
    \includegraphics[width=\textwidth]{mvosr/method/feature_sample_dis_bad.pdf}
    \caption{}
    \end{subfigure}
    \caption{在KITTI序列00的两帧上对算法1至3进行初步测试。标签"Depth-Correct Features"代表算法1和2之后的分布,标签"Selected Features"代表算法1至3的分布。(a)第10帧。(b)第20帧。}  
\end{figure}

\begin{table}
    \centering
    \setlength{\tabcolsep}{1mm}{
    \begin{tabular}{l c c c c c c c c c c c c c}
    \toprule  
    \multirow{4}*{Seq} & \multicolumn{2}{c}{Zhou \textit{et al}.} & \multicolumn{2}{c}{LibVISO2-stereo} & \multicolumn{2}{c}{Lee \textit{et al}.}& \multicolumn{2}{c}{Song \textit{et al}.} & \multicolumn{2}{c}{\multirow{2}*{ST-MonoVO-SR}} & \multicolumn{2}{c}{\multirow{2}*{ORBSLAM2-SR}} \\
    & \multicolumn{2}{c}{(from \cite{zhou2016reliable})} & \multicolumn{2}{c}{(from \cite{Geiger2011IV})}&   \multicolumn{2}{c}{(from \cite{Lee2015MoncularScale})}&\multicolumn{2}{c}{(from \cite{Song2015MoncularScale})}   & &\\
        \cline{2-3}  \cline{4-5}  \cline{6-7} \cline{8-9} \cline{10-11} \cline{12-13}
    & Trans & Rot  & Trans & Rot &Trans & Rot& Trans & Rot&Trans & Rot& Trans & Rot\\
    & (\%) & (deg/m)  & (\%) & (deg/m)& (\%) & (deg/m) & (\%) & (deg/m)&(\%) & (deg/m) &(\%) & (deg/m)\\
        \midrule
        00&2.17&0.0039&2.32&0.0109&4.42&0.0150&2.04 &0.0048& 2.17&0.0053&\textbf{1.01}& \textbf{0.0014} \\
        01&-&-&-&-&6.91&0.0140&- &&-&- &-&-  \\
        02&-&-&2.01&0.0074&4.77&0.0168&1.50 &0.0035  &1.81&0.0041  &\textbf{0.93}&\textbf{0.0018}\\
        03&2.70&0.0044&2.32&0.0107&8.49&0.0192&3.37&0.0021     &1.45&0.0035 &\textbf{0.52} &\textbf{0.0010}\\
        04&-&-&\textbf{0.99}&0.0081&6.21&0.0069&1.43 &\textbf{0.0023} &2.21&0.0049 &1.16&\textbf{0.0023} \\
        05&-&-&1.78&0.0098&5.44&0.0248&2.19 &0.0038     &1.51&0.0041 &\textbf{1.45}&\textbf{0.0014}\\
        06&-&-&\textbf{1.17}&0.0072&6.51&0.0222&2.09 &0.0081 &2.91&0.0060 &2.92&\textbf{0.0027}\\
        07&-&-&-&-&6.23&0.0292&- &-      &-&-& \textbf{1.73}&\textbf{0.0023}  \\
        08 & - & - & 2.35 & 0.0104 & 8.23& 0.0243&2.37 & 0.0044  &2.34 &0.0035 &\textbf{1.18}&\textbf{0.0017}\\
        09  & -& - & 2.36 & 0.0094 & 9.08& 0.0286&1.76 & 0.0047   &1.85&0.0032  &\textbf{1.17}&\textbf{0.0020}\\
        10 & 2.09 & 0.0054 & 1.37 & 0.0086 & 9.11& 0.0322&2.12 & 0.0085   &1.83&0.0048 &\textbf{0.93}&\textbf{0.0029}\\
        \midrule
        % \textbf{Average.} & \textbf{84.0}\\
        AVG & 2.32 & 0.045 & 2.02 & 0.0095& 6.86 &0.0212&2.03&0.0045  &2.01 &0.0049&\textbf{1.25}&\textbf{0.0020}\\
        % \hline
        \bottomrule
        \end{tabular}}
        \label{tab:kitti_compare}
        \caption{与其他最先进的VO/SLAM算法的比较}
\end{table}

\subsection{算法不同模块效果分析}
\label{sec:ablation}。
开源代码包括离线版和在线版。离线版需要保存初始运动和特征点,然后运行比例恢复代码;在线版的代码与初始VO代码融合,同时运行。本章给出了各模块的消融研究,包括特征选择模块(基于深度和道路模型的一致性)和参数探索模块(中值滤波和RANSAC)。使用经过特征稀疏化的的MonoVO和ORB-SLAM2提供初始运动估计。
\subsubsection{路面特征点选择测试实验}
\label{sec:FilterNecessrity}
\begin{figure}[t]
%\setlength{\abovecaptionskip}{0cm}
%\setlength{\belowcaptionskip}{-0cm}
 \centering
% \mysubfigure[]{0.44}{              %\label{fig:mvosr_method_dis_good}
% \includegraphics[width=\textwidth,height=0.69\columnwidth]{figures/mvosr/method/feature_sample_dis_good.pdf}}
% \label{FeaturesDistribution_a}
\begin{subfigure}[b]{0.44\textwidth}
    \includegraphics[width=\textwidth]{figures/mvosr/method/feature_sample_dis_good.pdf}
\caption{10}
\label{fig:FeaturesDistribution_a}
\end{subfigure}  
\begin{subfigure}[b]{0.44\textwidth}
    \includegraphics[width=\textwidth]{figures/mvosr/method/feature_sample_dis_bad.pdf}
\caption{10}
\label{fig:FeaturesDistribution_b}
\end{subfigure}  
% \mysubfigure[]{0.44}{               %\label{fig:mvosr_method_dis_bad}
% \includegraphics[width=\textwidth,height=0.69\columnwidth]{figures/mvosr/method/feature_sample_dis_bad.pdf}}
\caption{算法1-3在两帧图像上测试结果} % (c) Frame 15.}
%\vspace{-2\baselineskip}
\label{fig:FeaturesDistribution}
\end{figure}

选取的道路点受到两个规则的约束:深度一致性和道路模型一致性。我们算法的性能如图\ref{fig:GlobalDistribution}和\ref{fig:FeaturesDistribution}所示,图\ref{fig:GlobalDistribution}中表示在KITTI数据集中序列00上第10帧与第20帧两帧图像上测试算法1-3的结果。红色折线表示经过两帧图像匹配后可能的路面特征点,绿色折线"Depth-Correct Features"表示经过算法1和算法2后的特征点分布,黄色折线"Selected Features"表示经过算法1-3后的特征点分布。

表\ref{tab:depth_filter}-\ref{tab:feature_filter}所示,证明了本文提出的特征点选择算法确实可以有效地拒绝道路异常值,提高轨迹估计的精度。我们先在一些帧上进行定性测试,再在KITTI数据集的10个序列上进行定量测试。在此基础上,本文提出的轨迹估计算法确实能够有效地剔除道路异常值,以提高轨迹估计的精度。

首先,我们用我们的特征选择方法对KITTI数据集序列00上的前500帧进行定性分析,并计算其在纵轴上的分布,如图\ref{fig:GlobalDistribution}所示,特征过滤后,靠近道路的点的比例增加。然后,选取个别帧来观察道路点选择的效果。我们比较KITTI序列00中第10帧(图\ref{fig:FeaturesDistribution}a)和第20帧(图\ref{fig:FeaturesDistribution}b)在道路点选择前后的特征分布。红色曲线是所有特征点的分布。特征点属于路面的概率在特征选择后变得突出。

然后,选取KITTI序列00中的个别帧第10帧(图\ref{fig:FeaturesDistribution}a)和第20帧(图\ref{fig:FeaturesDistribution}b),查看道路点选择前后的特征分布。可以看出,第20帧上的特征点数量较少,不足以计算道路模型。
特征点选择后,道路点的百分比变得很突出。此外,我们将剩余的点用不同的颜色可视化,我们可以看到大部分被选择的点(用黄色标记)位于道路上。深度一致性选择后,在随机选择的帧上进行两次Delaunay三角测量的结果。
图\ref{fig:outlier_selection}直观地显示了深度一致性选择(图\ref{fig:outlier_selection}a)和深度与道路模型一致性选择(图\ref{fig:outlier_selection}b)后,在随机选择的帧上进行两次Delaunay三角测量的结果。二维特征点集$(u_i,v_i)$被分割成一组三角形。
一些由移动障碍物或不良特征跟踪产生的错误匹配特征被移除,如图\ref{fig:outlier_selection}所示。

\begin{table}
    \caption{采用同一种路面模型(RM)特征点筛选的情形下,加入直接踢除法(DD)或最大团模型(MC)特征点筛选对旋转矩阵$R$的影响。}
    \setlength{\tabcolsep}{3.3mm}
    \begin{center}
    \begin{tabular}{lccccc} 
    \toprule
    \multirow{2}{*}{Seq}  &Rotation error & RM & RM+DD  &RM+MC \\
    &(deg/m) &(\%) & (\%) & (\%)    \\
    \midrule
    \text{00} & 0.0053 & 2.87 & 2.25 & \textbf{2.17}  \\
    \text{02} & 0.0041 & 2.03 & 1.87 & \textbf{1.81}  \\
    \text{03}  & 0.0035 & 1.20 & \textbf{1.19} & 1.45  \\
    \text{04} & 0.0049 & 2.08 & \textbf{1.77} & 2.21  \\
    \text{05}  & 0.0041 & 2.97 & 1.81  & \textbf{1.51}  \\
    \text{06}  & 0.0060 & 4.51 & 3.10  & \textbf{2.91}  \\
    \text{07} & 0.0092 & 4.07 & 3.72  & \textbf{3.21}  \\
    \text{08}  & 0.0035 & 3.28 & 2.47 & \textbf{2.34}  \\
    \text{09}  & 0.0032 & 1.86 & 1.87  & \textbf{1.85}  \\
    \text{10} & 0.0048 & 3.04 & 2.45  & \textbf{1.83}  \\
    \midrule
    \text{Average} & 0.0049 & 2.80 & 2.25  & \textbf{2.13}\\ 
    \bottomrule
    \end{tabular}
    \end{center}
    \label{tab:depth_filter}
\end{table}


其次,我们在KITTI数据集的序列00、02-10上量化了我们的路面特征点选择策略的性能。我们分别测试基于深度一致性(\ref{tab:depth_filter})和模型一致性(\ref{tab:flat_filter})的路面点选择,
然后在表\ref{tab:feature_filter}中进行比较,实验重复十次,记录相对位移误差的平均百分比。旋转误差是由MonoVO算法获得的经过稀疏化的特征的RPE\cite{geiger2012we}。\%表示相对位移误差的百分比。

为了评估基于深度一致性的路点选择的性能,我们用相同的道路模型一致性进行测试,如表\ref{tab:depth_filter}所示。我们可以得出结论,直接删除算法\ref{alg:depth_selection_1}和最大小团筛选
算法\ref{alg:depth_selection_4}在道路异常值剔除中都能发挥作用,算法\ref{alg:depth_selection_4}在8个序列上的表现略优于算法\ref{alg:depth_selection_1}。

为了评估基于道路模型一致性的道路点选择的性能(在\ref{sec:RoadNormal}中的算法\ref{alg:flat_selection}),我们设计了另一个类似的对比实验,结果显示在\ref{tab:flat_filter}中。使用相同的最大团深度一致性,MC-ROI显示了基于ROI的最大团深度一致性选择的性能。对于基于ROI的方法将一个固定的区域视为道路而不进行选择,其性能是最差的。第5列中的结果是将最大团和道路模型方法结合起来得到的,其性能优于仅有最大团一致性约束的性能(第4列)。
我们还可以看到,使用直接删除后,道路模型一致性可以进一步拒绝更多的道路离群值,相对位移误差的比例从3.77\%下降到2.13\%。


\begin{table}
    \caption{采用同一种最大团模型(MC)特征点筛选的情形下,加入路面模型(RM)特征点筛选对旋转矩阵$R$的影响。MC-ROI是采用固定区域作为路面而未进行特种筛选的情形。}
    \centering
        \begin{tabular}{lccccc}
        \toprule
        \multirow{2}{*}{Seq} &Rotation error &MC-ROI &MC &MC+RM\\
        &  (deg/m) & (\%) &  (\%)& (\%) \\
        \midrule
        \text{00}  & 0.0053  & 4.05 &4.09  & \textbf{2.17}  \\
        \text{02}  & 0.0041& 4.08 &2.75  & \textbf{1.81}  \\
        \text{03} & 0.0035 & 5.05&2.73   & \textbf{1.45}  \\
        \text{04} & 0.0049 & 6.32 &\textbf{1.93}  & 2.21  \\
        \text{05}  & 0.0041& 3.72 &3.73   & \textbf{1.51}  \\
        \text{06} & 0.0060 & 2.99&3.71   & \textbf{2.91}  \\
        \text{07} & 0.0092 & 3.67&5.61   & \textbf{3.21}  \\
        \text{08}  & 0.0035 & 3.59&3.55  & \textbf{2.34}  \\
        \text{09}  & 0.0032 & 4.25&4.54   & \textbf{1.85}  \\
        \text{10} & 0.0048 & 1.97 &5.02   & \textbf{1.83}  \\
        \midrule
        \text{Average} & 0.0049 & 3.97  &3.77 & \textbf{2.13}\\ 
        \bottomrule
        \end{tabular}
        \label{tab:flat_filter}
\end{table}

在对深度和道路模型一致性性能单独分析后,我们收集的结果如表\ref{tab:feature_filter}所示。我们可以看到,它们的组合性能(第6列)确实优于单个算法。算法\ref{alg:depth_selection_4}和\ref{alg:flat_selection}的组合比单独使用效果更好。
我们还可以推断,道路模型一致性选择的效果比深度一致性选择的效果好,道路模型的约束性更强。这是有道理的,因为深度一致性约束的主要作用只是消除一些匹配错误的点,但道路模型一致性约束可以拒绝所有角度或高度与道路模型矛盾的分割三角形。

\begin{table}
    \caption{基于最大团模型(MC)特征点筛选和路面模型一致性(RM)特征点筛选下的不同旋转矩阵$R$的变化。}
    \centering
    \begin{tabular}{lcccccc}%{spreadtab}{{tabular}{cccccc}}
    \toprule
    \multirow{2}{*}{Seq}  &Rotation error &No selection & MC &RM  &MC+RM {}\\ {}
    & (deg/m) & (\%) & (\%)&(\%) & (\%)    \\
    \midrule
    \text{00} & 0.0053 & 13.28  &4.09 & 2.87&  \textbf{2.17}  \\
    \text{02}  & 0.0041 & 11.27  &2.75 & 2.03&  \textbf{1.81}  \\
    \text{03} & 0.0035 & 7.070 &2.73 & \textbf{1.20}&  1.45 \\
    \text{04} & 0.0049 & 9.200 &1.93 & \textbf{2.08}&  2.21  \\
    \text{05}  & 0.0041 & 9.860 &3.73 & 2.97&  \textbf{1.51}  \\
    \text{06}  & 0.0060 & 3.000 &3.71 & 4.51&  \textbf{2.91}  \\
    \text{07} & 0.0092 & 12.30  &5.61 & 4.07&  \textbf{3.21}  \\
    \text{08} & 0.0035 & 10.82  &3.55 & 3.28&  \textbf{2.34}  \\
    \text{09}  & 0.0032 & 16.77  &4.54 & 1.86&  \textbf{1.85}  \\
    \text{10} & 0.0048 & 10.50  &5.02 & 3.04&  \textbf{1.83}  \\
    \midrule
    \text{Average} & 0.0049 & 10.41  & 3.77  & 2.80 & \textbf{2.13} \\
    \bottomrule
    \end{tabular}
    \label{tab:feature_filter}
\end{table}

\subsubsection{RANSAC与Median滤波参数性能比较}
\label{sec:parameter_select}

我们使用RANSAC方法计算几何道路模型,并使用中值滤波法去除尺度噪声,如\ref{sec:road_model}节所述。我们评估了不同的RANSAC(最大迭代)和过滤器参数(过滤器大小)下视觉里程测量的位移误差。我们用不同的RANSAC和过滤器参数评估视觉里程测量的位移误��,以决定最合适的参数。初始运动由ORB\-SLAM2提供。

如果选择的特征点数量小于12个,我们跳过RANSAC步骤,保持道路几何模型与最后一帧验证的模型相同。为了确定最合适的RANSAC参数,我们设计了三个实验,结果如图所示\ref{fig:filter_method}。

首先,我们在KITTI数据集的序列00上用10次不同的最大迭代和17个滤波器大小进行测试。我们对每一种情况运行10次,平均位移误差如图\ref{fig:filter_method}a所示。我们可以看到,虽然这十条线代表着不同的最大迭代次数,但在过滤器大小为6之前,它们都会急剧减少,然后随着过滤器大小的增加而变得稳定。

所以我们暂时将滤波器大小设置为6,记录不同最大迭代下翻译误差的变化趋势,如图\ref{fig:filter_method}b所示。随着最大迭代次数的增加,性能不断提高,在15次之后变得稳定。考虑到计算成本和系统的鲁棒性,我们在实验中留有余地,将RANSAC的最大迭代次数设置为20次。

滤波器的大小决定了考虑多少张之前的图像来获得尺度$s$的中值。同样,我们将RANSAC的最大迭代次数设置为20,并记录不同滤波器大小下的翻译误差方差,如图\ref{fig:filter_method}c所示。我们可以观察到,在一定的阈值下,随着滤波器大小的增加,更多的路面特征点异常值被去除。
当滤波器大小达到6左右时,我们的方法实现了比较小的误差。这意味着所选取的特征点大多在道路平面上。但是,当超过一定的阈值后,过滤器尺寸过大,无法跟随尺度漂移,从而降低了尺度恢复方法的性能。最后,我们在实验中采用6作为最佳滤波大小。位移误差的平均值和中值分别约为1.0207\%和1.0008\%。此外,位移误差的标准误差约为0.0712\%。

在确认了中值滤波的最佳滤波尺寸后,我们对KITTI数据集中带中值滤波的序列00上的尺度和轨迹进行了展示,如图所示,\ref{fig:filter_path}。我们可以观察到,在这个序列中,没有尺度滤波器的原始轨迹发生了严重的尺度漂移。然而,在中值滤波器的帮助下,大部分尺度噪声被去除,我们的尺度恢复方法估计的VO轨迹非常接近真实路线。


\begin{figure}\centering
    \vspace{4mm}
    \begin{subfigure}[h]{0.8\textwidth}
    \includegraphics[width=\textwidth]{mvosr/fig_filter/filter_all_ransac_all_crop.pdf}
    \caption{}
    \vspace{3mm}
    \end{subfigure}
    \begin{subfigure}[h]{0.8\textwidth}
    \includegraphics[width=\textwidth]{mvosr/fig_ransac/errors_box_crop.pdf}
    \caption{}
    \vspace{3mm}
    \end{subfigure}
    \begin{subfigure}[h]{0.8\textwidth}
    \includegraphics[width=\textwidth]{mvosr/fig_filter/median_box_crop.pdf}
    \caption{}
    \vspace{3mm}
    \end{subfigure}
    \caption{RANSAC和中指滤波不同参数下的旋转误差。(a)RANSAC最大迭代次数的影响。(b) 中值滤波不同滤波尺寸的影响。}
    \label{fig:filter_method}
\end{figure}
    
\begin{figure}\centering
    \begin{subfigure}[h]{0.4\textwidth}
    \includegraphics[width=\textwidth]{mvosr/fig_filter/scale_f1_crop.pdf}
    \caption{}
    \end{subfigure}
    \vspace{2mm}
    \begin{subfigure}[h]{0.4\textwidth}
    \includegraphics[width=\textwidth]{mvosr/fig_filter/path_f1_crop.pdf}
    \caption{}
    \end{subfigure}
    \begin{subfigure}[h]{0.39\textwidth}
    \includegraphics[width=\textwidth]{mvosr/fig_filter/scale_f6_crop.pdf}
    \caption{}
    \end{subfigure}
    \begin{subfigure}[h]{0.4\textwidth}
    \includegraphics[width=\textwidth]{mvosr/fig_filter/path_f6_crop.pdf}
    \caption{}
    \end{subfigure}
    \caption{值滤波加入前后尺度系数和轨迹比较。(a)加中值滤波前的尺度系数。(b)加中值滤波前的轨迹。(c)加中值滤波后的尺度系数。(d)加中值滤波后的轨迹。}
    \label{fig:filter_path}
\end{figure}

\section{本章小结}
\label{sec:Conclusion}。
我们提出了一种以摄像机高度为绝对参考,基于道路几何约束的实时单目视觉里程计尺度恢复方法。我们的方法的新颖之处在于,道路特征点的选择和道路模型估计是迭代计算的——的估计道路几何模型被认为是选择道路点的反馈。考虑检测到的道路点,对几何道路模型进行在线更新。选取的路点用于估计道路模型,进而限制路点的选择。通过Delaunay三角测量法\cite{shewchuk1996triangle}将每一帧图像分割成一组三角形。通过考虑深度一致性和道路模型一致性,检查每个三角形是否属于道路区域。
在道路点的选择上,我们用Delaunay三角测量法对点位进行分割,根据深度一致性和道路模型一致性选择道路点。
因此,对于经过深度一致筛选的剩余三角形,我们将三角形区域的几何模型与前一帧的道路模型进行比较,只有在后续帧中法线$n_i$相似的三角形以及高度符合要求的才会被认证为路面特征点。摄像机运动$\mathbf{R}$和$\mathbf{bar{t}}$计算出的法线应该接近估计的道路法线$n_t$ 。

对于经过验证的道路点,我们采用RANSAC来估计道路模型,并采用中值滤波器来去除尺度噪声。实验结果表明,我们的路面特征点选取策略能够有效地剔除道路异常值,RANSAC和中值滤波器的参数探索有助于提高系统的精度和鲁棒性。实验结果还表明,现有的开源VO或SLAM方法,包括ORB-SLAM2(有LC和无LC)、LibVISO和MonoVO,通过与我们的尺度恢复方法相结合,得到了显著的改进。将ORB-SLAM2(无LC)和MonoVO作为我们的初始自我运动估计,我们的方法在四种最先进的VO方法中取得了最好的性能。因此,通过简单的单相机校准和安装的相机高度测量,我们的方法可以帮助机器人在未探索的环境中进行精准的自我定位。

在未来,我们计划探索与IMU等廉价传感器融合的绝对尺度估计,因为所有基于地面平面的视觉方法在地面平面被严重遮挡或不能作为平面消耗时都会失败。此外,考虑到我们基于点的算法在低纹理环境下可能会因为缺乏足够的特征而失效,我们也会关注更丰富的线或点线结合的特征。
                  %5       基于几何方法的尺度计算  子空间建模与辨识
\chapter{单目视觉里程计尺度计算:从手动建模到自主学习}
\section{引言}
前一章中介绍了路面几何模型的单目视觉里程计绝对尺度计算方法,该方 法将相对稳定的路面几何模型作为先验信息,本章将介绍一种场景中稳定区域的自主建模方法。
\section{基于场景建模的尺度计算方法} 
本文提出了一种基于场景建模的尺度计算方法,本节中将依次介绍场景建模方法以及基于场景模型的尺度计算方法。
\subsection{场景建模} 
\subsubsection{场景模型表征}
绝对尺度计算直接依赖于场景中特征点的绝对深度,故本文使用像素深度 概率模型来表征机器人所在的场景。该方法将机器人视野纵横分为多个栅格区 域$G_{ij}$,并使用高斯分布建模每个栅格区域的深度模型.
\begin{equation}
D(G_{ij} \mid S^k) N(\mu_{ij}^{k}-\sigma_{ij}^{k})>0
\end{equation}
其中$\mu_{ij}^{k}$ 和$\sigma_{ij}^{k}$分别为场景$S_k$中栅格$G_{ij}$所服从的高斯分布中的均值和标准差。

\subsubsection{场景建模方法}
场景建模的训练集需要包扩在特定场景中拍摄的连续图像序列${\textbf{I}_t}$和相邻图片拍摄位置的绝对距离$l_{t-1}^t$。首先依据相邻帧
图像$\textbf{I}_{t-1}$和$\textbf{I}_t$进行特征检测匹配和相对运动估计以及相对深度估计,具体操作如下
在图像$\textbf{I}_{t-1}$过量提取特征点,得到初始特征点集合$\Omega_{\tilde{f}}$和$u_{f^{t}}$可计算出机器人的相对运动$\textbf{R}$ $\bar{\textbf{t}}$,由于单目的尺度歧义性,无法准确获取$\bar{\textbf{t}}$的绝对大小,但可以
根据此运动通过三角测量的方式获取每个像素点的像素深度$\bar{d}_f$
在获取特征点的相对深度后,本文对特征点不同区域的深度分布进行统计学建模。首先根据相邻图片的绝对距离$l_{t-1}^t$,确定绝对尺度与相对尺度之前的系数
\begin{equation}
    s = \frac{l_{t-1}^t}{||\bar{\textbf{t}}_{t-1}^t||}
\end{equation}
进而可以获取,每个像素点在真实世界中的绝对深度$d =\textit{s}\bar{d}$。同时根据像素位置,将特征点归属于不同的栅格$G_{ij}$。在执行完场景中全部帧之后,使用高斯分布建模每个栅格内的特征点深度均值$\mu_{ij}$和标准差$\sigma_{ij}$
考虑到训练数据中可能存在多种分布不同的场景,简单的将所有场景用统一中模型表征会是偏差较大。本文提出使用聚类方法,将场景按照深度分为$K$个场景,对于每一个单独进行建模。场景聚类方法简述如下:首先我们将每一帧图像$\textbf{I}^t$按照特征点的分布位置和深度进行编码表示为$C_t \in \mathbb{R}^{h×w×2}$
\begin{equation}
    C_t[i_w,i_h] = (N(f),\mu(d))
\end{equation}
其中$N(f)$为特征点数量,$μ(d)$为特征点深度的均值。即我们使用各个栅格的特征
点数量和深度均值作为编码基本单元来表征每帧图像的结构信息(查查参考文献)。我们定义图像结构的距离为
\begin{equation}
    ||C^{t1}-C^{t2}|| = \sum(|N(f)^{t1}-N(f)^{t2}|+N(f)^{t1}N(f)^{t2}(\mu(d)^{t1}-\mu(d)^t2))    
\end{equation}
依据如上编码方式和编码距离的定义,可以完成聚类。
\subsection{尺度计算}
获取场景绝对深度模型之后,使用绝对深度与相对深度的比值即可计算尺度系数,在计算过程中,首先根据相对深度计算场景结构编码,
然后选取与当前结构最近的场景用于计算尺度
\begin{equation}
    S_k = argmin S_k
\end{equation}
在计算过程中,以栅格深度的均值作为该位置区域的绝对深度,通过栅格深度的标准差评价改绝对深度的可靠性,并根据如下公式计算尺度的最优值
\begin{equation}
    s = \frac{\sum f_i(\frac{1}{\sigma_{fi}^2})\frac{u_i}{d_{f_i}}}{\sum f_i(\frac{1}{\sigma_{f_i}^2})}
\end{equation}
获取尺度系数之后,即可恢复机器人相对运动的绝对尺度。

\subsection{基于场景建模的尺度计算验证实验} 
\subsubsection{模型参数测试实验}
\subsubsection{算法有效性消冗实验}
\subsubsection{与其它方法性能对比}
\section{本章小结}        %2     
\chapter{传统单目位姿估计与深度学习尺度恢复}
\section{引言}     %0  20201010 至 20201020     11天的工作
                                    %   20201011 DeepScale跑出来 1012融合
\chapter{路面驾驶机器人单目视觉里程计简化}
% Problem

人类的日常生活越来越多地涉及到移动机器人,包括自主地面车辆(AGV)、无人驾驶飞行器(UAV)和服务机器人。
移动机器人在复杂的环境中进行导航和执行任务时,必须对自己进行定位。
在环境未开发的情况下,既没有全球定位系统(GPS),也没有环境地图,无法进行绝对状态估计,只能采取相对状态估计,又称为增量式定位方法,如视觉测距。
该方法是通过累积式增量计算机器人相对于到它的起始坐标系中的增量自我运动(包括平移和旋转运动),与构建的地图无关。
大多数VO利用基于几何学的方法来估计自我运动$(\mathbf{R},\mathbf{t})$,
通过最小化重投影误差\cite{raul2015orb}(基于特征的方法)或最小化光学误差\cite{Engel-et-al-pami2018}(直接法)。
然而,这些方法需要准确的传感器校准和手动参数调整,以便在不同的环境中良好地工作\cite{roberts2008memory}。
但是,参数调整过程需要工程师根据状态反馈不断进行调整,是一项消耗精神与时间的繁琐工作。


为了减少人为调整参数的努力,很多研究工作都投入到基于学习的端到端方法中。
用基于学习的方法进行自我运动估计是由Roberts等人开始的\cite{roberts2008memory},他们尝试用K-Nearest Neighbors模型学习从光流到二维运动的映射。 
其他许多开创性的方法也在探索建立从光流到小我运动的映射模型\cite{guizilini2012semi,costante2016exploring,pillai2017towards,costante2018ls}。
Wang等人\cite{wang2017deepvo,wang2018end}首先提出了从原始图像到自我运动的端到端模型。除此之外,Wang等人还通过使用递归神经网络来处理图像序列信息。
为了减少标签数据的依赖性,Zhou等人/\cite{zhou2017unsupervised}提出了两个无监督的网络结构,与图像深度预测与自我运动估计相结合,将重投影图像的残差作为损失函数。
在此基础上,许多研究人员设计了不同损失函数进行网络改进:增加额外的3D几何损失函数\cite{mahjourian2018unsupervised},双目损失函数\cite{li2018undeepvo},深度特征重建损失函数\cite{zhan2018unsupervised},动态和光流损失函数\cite{yin2018geonet}或相对损失函数\cite{almalioglu2019ganvo}来等。
为了实现系统的鲁棒性,Klodt等人\cite{klodt2018supervising}和Yang等人\cite{yang2020d3vo}进一步提出测评自我运动和深度的不确定性。而Clement等人\cite{clement2018train}则试图用图像转换模型来增加算法对光照的鲁棒性。


H然而,基于端到端深度学习方法的表现仍然不容乐观。我们发现其中一个基本问题是基于学习的方法总是依赖于一个庞大而多样化的数据集来训练一个性能良好的模型。如Imagenet数据集\cite{deng2009imagenet}用于物体检测,Cityscape数据集\cite{Cordts2016Cityscapes}用于语义分割。然而,对于地面车辆的自我运动估计,
最显著的数据集如KITTI\cite{geiger2012kitti}和Robotcar\cite{RobotCarDatasetIJRR},而用于移动机器人视觉定位的开源训练数据集在图片数量和多样性上仍十分有限。
Zhou等人的自监督方法\cite{zhou2017unsupervised}可以减弱对地面车辆运动ground truth的依赖性,降低了训练数据采集难度,但他们并没有解决对训练数据数量的要求,因为他们仍然依赖于图像序列。
为了应对训练数据集的局限性,Slinko等人\cite{slinko2019training}提出基于RGB-D图像的随机重投影来生成训练集,增加的训练数据的可利用性。
此外,Wang等人\cite{wang2020tartanair}通过模拟不同的环境和具有挑战性的光照条件,收集了一个更大的具有复杂运动模式的数据集。


为了降低网络对数据集的依赖,我们提出通过简化学习目标,在有限的KITTI数据集上学习视觉里程计深度模型。

我们观察发现,以往的路面机器人在实现自我定位时,学习的是六自由度的运动$\textbf{R}$和$\textbf{t}$。然而,在车辆的行驶过程中,其主要运动(如图\ref{fig:car_simplify}所示)。由于地面车辆的运动受到其机械结构和动力学的限制,所以如果我们的深度网络结构只专注于主要运动时并不会导致过多的姿势偏移。
另外,我们观察发现由于噪声的存在,观察到的次要运动总是具有较低的信噪比。基于以上观察我们得出结论,当深度学习网络聚焦于学习地面车辆的
主要运动时,可以减少对训练数据量的依赖,简化学习问题的同时可以降低其余自由度带来的冗余误差,是一个可观的深度学习视觉定位方案。

地面车辆的约束运动模型已被广泛采用在基于几何学的视觉里程测量方法中。%许多方法都是利用安装摄像头的固定高度作为绝对参考来恢复单目视觉里程计的尺度。
Scaramuzza等人\cite{scaramuzza2009real}基于运动约束和阿克曼转向定律\cite{siegwart2011introduction}提出1-point-RANSAC用于自我动作估计,
以提高实时性能。Choi等人\cite{choi2015simplified}考虑了突然的颠簸或相机振动,并放宽了平面假设。Scaramuzza等人 \cite{4625958}利用全向相机提
供的地面特征点根据单应性矩阵法计算机器人自我运动位姿。

\begin{figure}[h]
    \centering
    \includegraphics[width=0.7\textwidth]{datavo/car_simplify.pdf}
    \caption{动机:将网络学习目标集中在路面车辆的主要运动上以简化深度学习目标并降低冗余误差。}
    \label{fig:car_simplify}
\end{figure}

本文尝试了一种全新的路面车辆运动模型,我们将神经网络的学习目标集中在主要运动的自由度上(命名为运动聚焦),并定量评估了当忽略某个次要运动自由度时所引起的姿态位移,
并探讨了通过考虑地面车辆运动模型(命名为运动解耦)来最小化姿态位移。

此外,我们构建了仅有4个卷积层的轻型卷积神经网络来学习地面车辆的显著运动,并进行了实验,表明运动聚焦与运动解耦可以提高自我运动估计性能。整个系统的结构如图所示\ref{fig:system_structure}。本章的主要贡献是:

\begin{enumerate}
    \item {通过对运动聚焦引起的地面真实姿势位移进行定量评估,发现其位移相对较小,可以通过实验证明运动聚焦的可行性;}
    \item {分析了X轴意外平移的原因,建立了X轴平移和Y轴旋转的关系模型,并利用该模型来减少运动解耦引起的姿势位移;}
    \item {我们在KITTI数据集上进行了对比实验,表明所提出的运动聚焦和解耦可以减少训练时间,提高学习性能;}
    \item {构建了轻型卷积神经网络来模拟地面车辆的主要运动,模型足够轻,可以在GPU上用2G左右的内存进行训练,并在CPU上实时运行(每秒超过200帧)。
    为了促进视觉里程计的发展,我们公开了源代码\footnote{https://github.com/TimingSpace/DMVOGV}。}
\end{enumerate}

本文的其余部分组织如下。第一节\ref{sec:IEEEAccess_approach}描述了我们的方法,包括运动聚焦和训练细节。在第\ref{sec:IEEEAccess_experiments}节中,我们的方法在KITTI数据集\cite{geiger2012kitti}上进行了评估。我们在\ref{sec:IEEEAccess_conclusion}中对本文进行了总结并讨论了未来的工作。
\section{方法}
\label{sec:IEEEAccess_approach}

\begin{figure*}[t]
    \centering
    \includegraphics[width=0.9\textwidth]{datavo/system_structure.pdf}
    \caption{运动聚焦与解耦算法系统结构。}
    \label{fig:system_structure}
\end{figure*}
本节在\ref{sec:motion}首先介绍了运动聚焦和运动解耦的数据处理方法;然后在章节\ref{sec:model}中介绍了关于地面车辆视觉里程模型的网络结构和训练细节。

%\subsection{Data Processing}
\subsection{运动聚焦和运动解耦}
\label{sec:motion}
运动聚焦是通过将注意力集中在主要运动上来简化学习目标的一种思想,这在\ref{sec:motion_focus}中有详细描述,运动聚焦引起的姿势位移在节\ref{sec:info_loss}中进行了评估。运动解耦减少了运动聚焦引起的姿势位移,
这在\ref{sec:motion_decouple}中进行了描述,在\ref{sec:info_decouple}中进行了定量评价。运动聚焦和运动解耦引起的性能提升在\ref{sec:ego_improvement}节中进行评估。
\subsubsection{运动聚焦}
\label{sec:motion_focus}
\begin{figure}[ht]
    \centering
    \begin{subfigure}[b]{0.65\textwidth}
        \includegraphics[width=\textwidth]{datavo/motion_dis.png}
        \caption{}
        \label{fig:motion_dis} 
        \vspace{4pt}
    \end{subfigure}
    \begin{subfigure}[b]{0.65\textwidth}
        \includegraphics[width=\textwidth]{datavo/rotation_corr.pdf}
        \caption{}
        \label{fig:rotation_corr}
    \end{subfigure}
        \caption{运动模式分析。(a)沿着或围绕不同轴的运动比较;(b)X轴平移和Y轴旋转的相关性。}
\end{figure}

%The common method to deal with motion amplitude difference is data normalization or using different loss scale parameters \cite{wang2017deepvo}. 
%However, because the signal-noise ratio is small for the constraint motion axis of a ground vehicle,
%the normalization will introduce a lot of noise. 
运动聚焦就是忽略地面车辆的微不足道的运动,集中精力对大部分运动进行建模。我们在分解运动时采用常规的相机坐标系作为参考系(如图\ref{fig:car_simplify}所示),
该坐标系为右手系,定义如下:原点为相机的光学中心,z轴定义为前进光轴,x轴水平向右,y轴垂直向下。
围绕x轴、y轴和z轴的旋转运动分别用Euler角$\psi$、$\varphi$和$\theta$表示。沿不同轴的平移运动分别用$x$、$y$和$z$表示。

\begin{figure}[h]
    \centering
    \begin{subfigure}[b]{0.65\textwidth}
        \centering
        \includegraphics[width=\textwidth]{datavo/r_t_2d_hist.pdf}
        \caption{}
        \label{fig:rt_2d} 
        \vspace{4pt}
    \end{subfigure}
    \begin{subfigure}[b]{0.65\textwidth}
        \centering
        \includegraphics[width=\textwidth]{datavo/r_t_1d_hist.pdf}
        \caption{}
        \label{fig:rt_1d}
    \end{subfigure}
    \caption{旋转角和平移角的关系。(a)平移角和旋转角的二维直方图;(b)平移角和旋转角的一维直方图。}
    \label{fig:rotation_corr_analysis}
\end{figure}
我们在一个典型的地面车辆运动估计数据集——KITTI视觉里程数据集\cite{geiger2012kitti}上对地面车辆的运动模式进行了定量评估。
我们计算了KITTI序列00中关于不同轴的平移运动和旋转运动的方差,如图\ref{fig:motion_dis}所示,更多的方差可视化与图\ref{fig:motion_dis}类似,详见附录。
图\ref{fig:motion_dis}显示了地面车辆的大部分运动是z轴方向的平移运动和y轴方向的旋转运动。
所以我们建议简化运动估计目标,只关注大多数运动,我们称这个建议为运动聚焦。由运动聚焦引起的姿势位移在第\ref{sec:info_loss}节中进行了评估。

\subsubsection{运动解耦}
\label{sec:motion_decouple}

然而,如图\ref{fig:motion_dis}所示,沿x轴仍有不可忽略的平移运动,具体分析见表\ref{tab:info_loss_1}。
由此可知,如果直接全部忽略X轴的运动,会造成较大的漂移姿势(10\%)。然而,考虑到动力学约束,地面车辆不能沿X轴移动太多,为什么
存在高达10%的X轴平移呢?当深入研究地面车辆的运动模式时,我们发现X轴运动是由运动表示方法产生的。
\begin{equation}
    \begin{pmatrix} \mathbf{R} & \mathbf{t}\\ 0 & 1  \end{pmatrix} = \begin{pmatrix} \mathbf{I}& \mathbf{t}\\ 0 & 1  \end{pmatrix}\begin{pmatrix} \mathbf{R}& \mathbf{0}\\ 0 & 1  \end{pmatrix}
    \label{eq:ftlr}
\end{equation}
这里$\mathbf{I}$是一个3x3的单位矩阵。
在这种表示方式中,平移运动$mathbf{t}$先于旋转运动$mathbf{R}$。因此,当车辆有旋转运动时,参考坐标系统已经发生了变化,前向运动$z'$
被映射成较小的前向运动$z$与侧向运动(x轴平移{$x$}),如图\ref{fig:vehicel_rotation_model}所示。它引起的平移角$\alpha$,定义为: 
\begin{equation}
    \alpha = \arctan\left(\frac{x}{z}\right)
\end{equation}
这里$x$和$z$分别表示x-轴与z-轴的平移。
当我们将x轴平移和y轴旋转可视化后,就可以得到验证,如图\ref{fig:rotation_corr}所示,因为这两个运动是高度相关的。图\ref{fig:rotation_corr}只能可视化一个子序列的局部相关性。我们利用图\ref{fig:rotation_corr_analysis}中的两个直方图来可视化KITTI数据集序列00-10中,所有Y轴旋转角$\theta$和平移角$\alpha$的全局关系。
我们利用图中的两个直方图(\ref{fig:rotation_corr_analysis})来可视化所有KITTI序列00-10中所有Y轴旋转角$theta$和平移角$alpha$的全局关系。图\ref{fig:rt_1d}中的1d直方图显示了$\alpha/\theta$的分布,图\ref{fig:rt_2d}中的2d直方图可视化了$\alpha$和$\theta$的联合分布。这两个分布都表明y轴的旋转角$\theta$和平移角$\alpha$是相关的。
那么如何重新制定运动表示法来减少运动的相关性呢?一个简单的方法是将平移重写为:
\begin{equation}
    \begin{pmatrix} \mathbf{R} & \mathbf{t}\\ 0 & 1  \end{pmatrix} = \begin{pmatrix} \mathbf{R'}& \mathbf{0}\\ 0 & 1  \end{pmatrix}\begin{pmatrix} \mathbf{I}& \mathbf{t'}\\ 0 & 1  \end{pmatrix}
    \label{eq:frlt}
    \end{equation}
在这个公式中,车辆的旋转是先于平移的,所以平移运动是相对于旋转运动后的参考系而言的,不会被重新映射。可以得出的关系是:
$\mathbf{R'} = \mathbf{R}$, $\mathbf{t'} = \mathbf{R}^{-1}\mathbf{t}$.
\begin{figure}[ht]
    \centering
    \begin{subfigure}[b]{0.65\textwidth}
        \includegraphics[width=\textwidth]{datavo/vehicle_rotation_1-crop.pdf}
        \caption{}
        \label{fig:vehicel_rotation_model} 
        \vspace{4pt}
    \end{subfigure}
    \begin{subfigure}[b]{0.6\textwidth}
        \includegraphics[width=\textwidth]{datavo/vehicle_rotation_2-crop.pdf}
        \caption{}
        \label{fig:vehicel_rotation_model_s} 
    \end{subfigure}
    \caption{地面车辆旋转模型。(a) 旋转模型;(b) 简化的旋转模型。}
    \label{fig:rotation_model}
\end{figure}
然而,如图\ref{fig:vehicleel_rotation_model}所示,绕y轴的旋转角度$\theta$不等于由旋转产生的平移角$\alpha$。
我们需要找出$\alpha$和$\theta$之间的关系$\alpha=f(\theta)$。
然后,我们就只需保持y轴的旋转$\theta$和汽车的前移$z$,使用平移角$\alpha$,用公式\eqref{eq:car_angle}来恢复车辆的运动。
\begin{equation}
    (x,z) = z'(sin(f(\theta)),cos(f(\theta)))
    \label{eq:car_angle}
\end{equation}
在图\ref{fig:vehicleicel_rotation_model}中,A点是车辆后轴的中心,标记B是安装摄像头的位置,$l$表示A和B之间的距离。通过视觉里程测量法估算出的
车辆平移距离等于$B$和$B'$之间的距离,用{$z'$}表示。我们将图\ref{fig:vehicleel_rotation_model}简化为图\ref{fig:vehicleel_rotation_model_s}。
根据阿克曼转向定律\cite{siegwart2011introduction},$OA \bot AB$且$OA' \bot A'B'$,所以$\phi = 0.5 \beta = 0.5 \theta$。
在三角形$CBB'$中,根据正弦定律可知:
{
\begin{equation}
   \frac{\sin(\gamma)}{\sin(\beta)}  = \frac{l-\frac{\hat{z}}{2} / \cos(\frac{\theta}{2})}{z'} 
\end{equation}}
因为$\theta$趋近于0,所以$\cos(\frac{\theta}{2}) \approx 1$,且$ \frac{\gamma}{\beta} \approx  \frac{\sin(\gamma)}{\sin(\beta)} $,因此
{
\begin{equation}
    \frac{\gamma}{\beta}  \approx \frac{l-\frac{\hat{z}}{2}}{z'} 
\end{equation}}
又根据三角形$CBB'$中的余弦定律,{$d = |AC| \approx 0.5|AA'| =0.5\hat{z}$}
{
\begin{equation}
    \begin{split}
        z'^2 &= (l+d)^2 + (l-d)^2- 2(l+d)(l-d)\cos(\beta) \\
        &= 2l^2+2d^2 - 2(l^2-d^2)\cos(\beta)\approx 4d^2
    \end{split}
\end{equation}
}
因此$z'\approx \hat{z}$,可知平移角度$\alpha$和旋转角度$\theta$
{
\begin{equation}
    \alpha = \beta + \gamma \approx (\frac{l}{z'}+0.5)\beta =(\frac{l}{z'}+0.5)\theta
    \label{eq:r_t_ratio}
\end{equation}}
我们用位移角度$a$构建旋转矩阵$\mathbf{R}_\alpha$,
\begin{equation}
    \mathbf{R}_\alpha = \begin{pmatrix}
        \cos(\alpha)& 0 & \sin(\alpha)\\ 
        0 & 1 & 0\\ 
        -\sin(\alpha)& 0 & \cos(\alpha)\\ 
    \end{pmatrix} 
    \label{eq:r_alpha}
\end{equation}
然后得到$\mathbf{t}'$变形为
\begin{equation}
    \mathbf{t}' = \mathbf{R}_\alpha^{-1}\mathbf{t}
    \label{eq:decouple_z}
\end{equation}
所需的车辆前行运动{$z'$}是$\mathbf{t}'$的第三个元素。到目前为止,地面车辆的规划者运动可以由两个变量来表示:旋转角$\theta$和重映射前向运动$z'$。
我们专注于学习两维运动,以简化学习目标。模型和学习细节将在下一节介绍,运动聚焦和解耦引起的性能提升将在\ref{sec:ego_improvement}中进行评估。

\subsection{模型与训练}
\label{sec:model}
\begin{figure*}[t]
    \centering
    \includegraphics[width=\textwidth]{datavo/network_structure_2-crop.pdf}
    \caption{自我运动估计的轻型卷积神经网络}
    \label{fig:nerwork_structure}
\end{figure*}
我们构建一个轻型网络结构来学习地面车辆的主要运动。如图\ref{fig:nerwork_structure}所示。模型主要由卷积层组成,除了最后一层外,每个卷积层后面都有一个
组归一化层\cite{wu2018group}和ReLU层。与Zhou等人相同的是\cite{zhou2017unsupervised},我们使用全局平均池化层\cite{lin2013network}而不是全连接层
作为最后一层,以减少过拟合。我们观察到,地面车辆拍摄的图像的光流主要是水平的,特别是当车辆转弯时,如图\ref{fig:optical_flow}所示,所以我们利用卷积层与非
正方形核来实现更大的水平感受野。此外,我们还采用了扩张卷积层\cite{yu2015multi},以较少的参数获得更大的感知场。
\begin{figure}[ht]
    \centering
    \begin{subfigure}[b]{0.45\textwidth}
        \includegraphics[width=\textwidth]{datavo/flow_61.png}
        \caption{}
        \label{fig:optical_flow_f} 
        \vspace{4pt}
    \end{subfigure}
    \begin{subfigure}[b]{0.45\textwidth}
        \includegraphics[width=\textwidth]{datavo/flow_196.png}
        \caption{}
        \label{fig:optical_flow_l} 
        \vspace{4pt}
    \end{subfigure}
    \begin{subfigure}[b]{0.45\textwidth}
        \includegraphics[width=\textwidth]{datavo/flow_96.png}
        \caption{}
        \label{fig:optical_flow_r} 
    \end{subfigure}
    \caption{地面车辆的光流。(a) 前进时的光流;(b) 左转时的光流;(c) 右转时的光流。}    
    \label{fig:optical_flow}
\end{figure}

模型的输入是堆叠的灰色图像对,我们不仅使用序列图像来构造图像对{(帧间隔为0)},而且在{[-4,4]}中为每个样本随机选择一个{帧间隔},作为一个数据{增量}过程。 

输出是由y轴旋转$\theta $和重映射前向平移$z'$所代表的相应主要相机运动。  {$z'$是根据\eqref{eq:decouple_z}}得到的。
L2损失被用作监督函数:
\begin{equation}
    L_2 = \|\theta_g -\theta_p\|_2 +\|z'_g -z'_p \|_2
\end{equation}
下标为$p$的变量$\theta_p$和{$z'_p$}代表预测的{结果},下标为$g$的变量$\theta_g$和{$z'_g$}代表{the}地面真值。
我们使用ADAM \cite{kingma2014adam}来优化模型参数,学习率设置为0.001,50个epoch后线性衰减。%我们将权重衰减参数设置为0.001,以增加正则化。

模型训练完成后,输入一个新图像序列后,从模型输出中可以得到旋转角$\theta$和前向运动{$z'$}。
我们首先计算平移角$\alpha$,并假设关于其他轴的旋转为零,并构造旋转矩阵$\mathbf{R}_\theta$和$\mathbf{R}_\alpha$,则车辆转换向量{$\mathbf{t}_\alpha=\mathbf{R}_\alpha(0,0,z')^T$},该方程等同于\eqref{eq:car_angle},称为运动恢复。
路面车辆整体运动矩阵可以表示为: 
\begin{equation}
    \mathbf{T}_i =\begin{pmatrix} \mathbf{R}_\theta & \mathbf{t}_\alpha\\ 0 & 1  \end{pmatrix} 
    \label{eq:rt_final}
\end{equation}
然后,通过累积可以得到车辆位姿为: 
\begin{equation}
    \mathbf{P}_i = \mathbf{P}_{i-1}\mathbf{T}_i
    \label{eq:pose}
\end{equation}
\section{实验}
\label{sec:experiments}
我们在KITTI数据集\cite{geiger2012kitti}上进行了四个实验来评估所提出的方法的性能。首先,我们对KITTI数据集和实验平台进行介绍说明。
其次,我们详细介绍了四个实验:姿势位移评估、运动解耦性能、自我运动估计改进以及与其他方法的比较。最后,我们对实验结果进行讨论和分析。

\subsection{数据集和实验平台}

KITTI基准提供了22个测试序列,其中前11个序列为地面真实姿态评估。每个测试序列中都提供了{RGB图像、灰色图像和激光雷达点云}。


训练和测试我们的模型时,我们只利用单眼灰色图像与地面真实姿势{(KITTI数据集序列00-10)}。

{我们把训练数据集分成了四个不同的训练-评估模型,用于定量评估所提出的运动聚焦和解耦对自我运动估计的改进,详细内容见章节\ref{sec:ego_improvement};为了与其他相对方法进行比较,在章节\ref{sec:compare}中,我们使用KITTI 00-08进行训练,09-10进行评估,这与其他基于学习的方法是一样的,以便进行公平比较。} 

{我们使用RPE(相对姿势误差的简称)的平均值包括
相对旋转误差和相对位移误差 \cite{geiger2012kitti}作为评估指标。}

{我们的算法是基于PyTorch用Python实现的,PyTorch是一个成熟的深度学习框架,具有方便的Python接口}。

我们的代码已在github网站公开,该算法在个人笔记本电脑上进行测试,其配置内存为16GB,CPU为Intel Core i7-7700@2.80GHz,主频为2.80GHz,Nvidia 1060 GPU,6GB图形内存。测试环境为Ubuntu 18.04,
使用CUDA 10.0和Python 3.6.9。模型训练只需要2.0G的GPU内存{当批量大小设置为30时},且即使只用CPU进行测试的情况下,测试频率也可达到200FPS(帧/秒)以上。

\subsection{实验结果}

首先,我们在第\ref{sec:info_loss}节中通过RPE评估运动聚焦所造成的姿势位移。第二,在第\ref{sec:info_decouple}节中,我们评估了运动解耦后姿势位移的缓解(mitigation)。第三,在第\ref{sec:ego_improvement}节中,我们评估了所提出的运动聚焦和解耦对自我运动估计的改进。最后,我们在章节\ref{sec:compare}中比较了我们与其他基于学习和基于几何的方法的结果。
\subsubsection{Motion Displacement by Motion Focusing}
\label{sec:info_loss}

%由于地面飞行器的运动受其动力学和机械结构的限制,其大部分运动是沿z轴和绕y轴的。

为了说明运动聚焦的可行性,我们定量地评估了当忽略部分或所有其他无关紧要的运动维度时,运动聚焦的程度对姿态漂移的影响。

我们重建了运动减少后的姿势,并利用RPE\cite{geiger2012kitti}来评估姿势位移。KITTI数据集序列00-10的平均RPE记录在表\ref{tab:info_loss_1}中。
\begin{table}[h]
    \caption{Average RPE When Only Keeping Part of Vehicle Motion}
    \label{tab:info_loss_1}
    \begin{center}
    \begin{tabular}{c c c c c}
    \toprule
    % \hline
    \multirow{2}*{R / t} &{z}&{c}{xz}&{yz}&{zyz}\\
    & RPE(\%) /NID& RPE(\%) /NID& RPE(\%) /NID& RPE(\%) /NID\\
    %  % \hline%\hline
    \midrule
     y   &2.20  /4 & 2.06 /3 & 2.45 /3 & 2.34 /2 \\
     xy  &1.92  /3 & 1.77 /2 & 1.76 /2 & 1.56 /1 \\
     zy  &2.05  /3 & 1.91 /2 & 1.47 /2 & 1.27 /1 \\
     xyz &1.92  /2 & 1.81 /1 & 0.49 /1 & 0    /0   \\
    % \hline
    \bottomrule
    \end{tabular}
    \end{center}
 \end{table}
 \iffalse
 \begin{table}[t]
    \caption{Information Loss by Focusing only on translation along $z$ and roation about $y$}
    \label{tab:info_loss_2}
    \begin{center}
    \begin{tabular}{c c c c c c c c c c c c }
    \toprule
    % \hline
    seq & 00 & 01 & 02 & 03 & 04 & 05 & 06 & 07 & 08 & 09 & 10\\
    %  % \hline%\hline
    \midrule
     RPE(\%) &1.31 & 1.93 & 3.05 & 2.80& 2.22&1.30&1.34&1.19&1.44&3.79&3.85 \\
     ATE(m) &1.31 & 1.93 & 3.05 & 2.80& 2.22&1.30&1.34&1.19&1.44&3.79&3.85 \\
    % \hline
    \bottomrule
    \end{tabular}
    \end{center}
 \end{table}
\fi
\begin{figure}[ht]
    \centering
    \includegraphics[width=0.9\textwidth]{datavo/info_loss.pdf}
    \caption{当忽略不同数量的维度时RPE平均值比较。}
    \label{fig:info_loss}
\end{figure}
在表\ref{tab:info_loss_1}中,列和行的名称分别代表保留的旋转轴和位移轴,NID表示忽略的维度数量。
当我们只保留z轴平移和y轴旋转时,忽略四个维度(NID=4)时,重建路径的RPE为2.20/%。我们将运动聚焦后的一些重构路径进行了可视化,如图\ref{fig:path_recon}所示,由此可以看出重构后的水平路径位移误差很小,误差在可接受范围。
位移主要积累在z轴上,z轴上的位移也受运行环境的影响,因为当路面有高低起伏时,位移会比较大(如图\ref{fig:conc_10}中表示的序列10),而当路面几乎平坦,位移就比较小(如图\ref{fig:conc_07}中的序列07)。更多可视化的重构路径详见附录。

\begin{figure}[ht]
    \centering
    \begin{subfigure}[b]{0.7\textwidth}
        \includegraphics[width=\textwidth]{datavo/path_recon_07.pdf}
        \caption{}
        \label{fig:recon_07}
        \vspace{4pt}
    \end{subfigure}
    \begin{subfigure}[b]{0.7\textwidth}
        \includegraphics[width=\textwidth]{datavo/path_recon_10.pdf}
        \caption{}
        \label{fig:recon_10}
    \end{subfigure}
  \caption{重建路径的可视化。(a)重建的KITTI 07;(b)重建的KITTI 10。}     
  \label{fig:path_recon}
\end{figure}
为了更好的理解,我们还在图{RPEs}\ref{fig:info_loss}中可视化了RPE。 蓝色的线表示忽略不同维度数时的平均{RPEs}误差,我们将平均的
RPE(成本)除以所进行的维度数(收益),如图\ref{fig:info_loss}中的黄线所示,该比率可视为成本收益指标。只保留z轴平移和y轴旋转的成本收益比相对较小。
\subsubsection{Pose Displacement Improvement by Motion Decoupling}
\label{sec:info_decouple}
运动解耦的目标是减少忽略x轴平移时的姿势位移,其方法将在章节\ref{sec:motion_decouple}中介绍。
为了显示所提出的运动解耦的效率,我们定量评估了运动解耦所减少的姿态位移,如表\ref{tab:info_loss_1}所示。表中第一行表示当忽略X轴的平移时将导致更多的姿势偏移(从2.06%到2.20%)。
根据公式\eqref{eq:r_t_ratio},平移角$\alpha$和旋转角$\theta$之间是线性关系。然而,线性映射的斜率不是固定的,而是相对于不固定的前向运动$z$而言的。

为了简化问题,我们首先使用固定斜率参数来变换所有的前向运动,并使用不同比例测试RPE。结果如图所示\ref{fig:static_decouple}。当比值设为1.7时,我们得到的RPE最小。可以被解释为当车辆处于旋转状态时,车辆的平均前移距离约为$\frac{1.7-0.5}{l}$米。
为了更好地理解{图\ref{fig:static_decouple}中的}条形图,我们{使用}不同的颜色来表示不同的情形。运动解耦的目标是减少忽略x轴平移时的姿势位移,其方法将在章节\ref{sec:motion_decouple}中介绍。
为了显示所提出的运动解耦的效率,我们定量评估了运动解耦所减少的姿态位移,如表\ref{tab:info_loss_1}所示。表中第一行表示当忽略X轴的平移时将导致更多的姿势偏移(从2.06%到2.20%)。
根据公式\eqref{eq:r_t_ratio},平移角$\alpha$和旋转角$\theta$之间是线性关系。然而,线性映射的斜率不是固定的,而是相对于不固定的前向运动$z$而言的。
\begin{figure}[ht]
    \centering
    \begin{subfigure}[b]{0.45\textwidth}
        \centering
        \includegraphics[width=0.9\textwidth]{datavo/r_t_ratio.pdf}
        \caption{}
        \label{fig:static_decouple}
    \end{subfigure}
    \begin{subfigure}[b]{0.45\textwidth}
        \centering
        \includegraphics[width=0.9\textwidth]{datavo/r_t_ratio_2.pdf}
        \caption{}
        \label{fig:dynamic_decouple}
    \end{subfigure}
    \caption{Decoupling performance {visualization}. (a) Different fixed ratio; (b) Different dynamic ratio}.
    \label{fig:r_t_ratio}
\end{figure}


固定的比例会忽略车辆前行距离的影响,所以我们尝试使用动态比例。我们使用不同的摄像机位置$l$,用公式\eqref{eq:r_t_ratio}计算比率,然后评估重建路径的RPE,如图\ref{fig:dynamic_decouple}所示。
当摄像头位置设置为距离后轴0.4m时,RPE误差最小。
图\ref{fig:dynamic_decouple}中的黑色条形图代表着平移角$\alpha=0.5\theta$,这与图\ref{fig:static_decouple}中的黑色条形图相同。
计算比率时,用{\eqref{eq:r_t_ratio}计算,则评估重建路径RPE,如图\ref{fig:dynamic_decouple}所示。
当摄像头位置设置为距离后轴0.4m时,RPE误差最小。图\ref{fig:dynamic_decouple}中的黑条也代表着平移角$\alpha=0.5\theta$,这与图\ref{fig:static_decouple}中的黑条相同。
图\ref{fig:decouple}中可以直观地看到动态解耦和静态解耦的比较。{在图\ref{fig:decouple}中,蓝色条形代表没有运动解耦的RPE,黄色和绿色条形分别代表
静态解耦(比例设为1.7)和动态解耦(摄像机位置$l$设为0.4m)的RPE,红色条形代表我们同时保持x轴和z轴平移运动时的RPE}。
可以发现,动态解耦和静态{解耦}都可以降低RPE。当我们同时保持x轴平移和z轴平移时,动态{解耦}的RPE比静态{解耦}的RPE要低,而且更接近RPE。
\begin{figure}[ht]
    \centering
    \includegraphics[width=0.6\textwidth]{datavo/decouple-crop.pdf}
    \caption{采用固定的解耦比率与动态比率时解耦性能比较。}
\end{figure}
\subsubsection{{Performance Improvement by Motion Focusing and Decoupling}}
\label{sec:ego_improvement}
\begin{table}[ht]
    \caption{The improvement of motion focusing}
    \label{tab:info_improve}
    \begin{center}
    \begin{tabular}{c c c c c c }
    \toprule
    % \hline
    \multirow{3}*{Train} & \multirow{3}*{Test} &\multicolumn{2}{c}{Learn All Motion} & \multicolumn{2}{c}{Learn $R_y, t_z$}\\
    & &Trans & Rot & Trans & Rot\\
    & & (\%) & (deg/m)  & (\%) & (deg/m)\\
    %  % \hline%\hline
    \midrule
     00 & 02 04 06 08 10 &26.8 & 0.137 & 23.9 & 0.110 \\
     00 02 & 04 06 08 10 &18.3 & 0.095 & 16.7 & 0.070 \\
     00 02 04 & 06 08 10 &17.6 & 0.091 & 16.9 & 0.076 \\
     00 02 04 06 & 08 10 &15.3 & 0.082 & 13.2 & 0.065   \\
    % \hline
    \bottomrule
    \end{tabular}
    \end{center}
 \end{table}

我们通过实验研究运动聚焦和解耦的影响。我们使用相同的训练数据来训练两种模型:
1)MFM(运动聚焦模型),只学习Y轴的旋转和Z轴的平移;2)AMM(全运动模型),学习六个自由度的运动。
在同一测试集上的RPE可以作为显示改进的指标。实验是在不同的训练-测试数据{分集}上进行的,以避免偶然性。我们记录了训练集的损耗变化曲线和测试集的{RPE}。
如图\ref{fig:training_loss}所示。对于所有的训练分割,MFM模型{收敛}比{AMM}模型快。
\begin{figure}[ht]
    \centering
    \begin{subfigure}[b]{0.45\textwidth}
        \includegraphics[width=\textwidth]{datavo/training_loss_0.pdf}
        \caption{}
        \label{fig:tl_0}
        \vspace{4pt}
    \end{subfigure}
    \begin{subfigure}[b]{0.45\textwidth}
        \includegraphics[width=\textwidth]{datavo/training_loss_0-2.pdf}
        \caption{}
        \label{fig:tl_02}
        \vspace{4pt}
    \end{subfigure}
    \begin{subfigure}[b]{0.45\textwidth}
        \includegraphics[width=\textwidth]{datavo/training_loss_0-4.pdf}
        \caption{}
        \label{fig:tl_024}
    \end{subfigure}
    \begin{subfigure}[b]{0.45\textwidth}
        \includegraphics[width=\textwidth]{datavo/training_loss_0-6.pdf}
        \caption{}
        \label{fig:tl_0246}
    \end{subfigure}
\caption{训练损失曲线比较。(a) 在KITTI 00上的训练;(b) 在KITTI 00 02上的训练;(c) 在KITTI 00 02 04上的训练;(d) 在KITTI 00 02 04 06上的训练。}    
{\label{fig:training_loss}}
\end{figure}
\begin{figure}[h]
    \centering
    \includegraphics[width=0.8\textwidth]{datavo/focusing_train.png}
    \caption{聚焦训练后性能的提升。}
    \label{fig:focucing_train}
\end{figure}
不同训练模型的测试RPE记录在表\ref{tab:info_improve}中,并直观地显示在图\ref{fig:focucing_train}中。
我们可以发现,在不同的训练数据模式下,运动聚焦模型的结果都比学习所有6自由度的运动模型结果好,位移误差提高了约2\%,旋转提高了0.2degree/m。
从测试结果中还可以观察到,随着训练数据的增加,运动聚焦模型和所有运动模型的测试效果都越来越好。同时我们注意到,运动聚焦模型是由带有漂移姿势的地面真值{运动聚焦和解耦后}进行训练的,但测试性能仍然较好。

\subsubsection{与其他方法的比较}
\label{sec:compare}
\begin{table*}[!htbp]
    \caption{Comparison with other Learning-based Methods}
    \begin{center}
    \begin{tabular}{c c c c c c c c c c c c c c}
    \toprule
    % \hline
    \multirow{4}*{Seq} & \multicolumn{2}{c}{Zhan et al.} &\multicolumn{2}{c}{DeepVO} & \multicolumn{2}{c}{SfM-Learner.}& \multicolumn{2}{c}{GeoNet}&  \multicolumn{2}{c}{\multirow{2}*{Our Method }}\\
                       & \multicolumn{2}{c}{(from \cite{zhan2018unsupervised})}  & \multicolumn{2}{c}{(from \cite{wang2017deepvo})}&\multicolumn{2}{c}{(from \cite{zhou2017unsupervised})} &\multicolumn{2}{c}{(from \cite{yin2018geonet})} &\\
    %  % \hline%\hline
    \cline{2-3}  \cline{4-5}  \cline{6-7} \cline{8-9} \cline{10-11} \cline{12-13}
        & Trans & Rot  & Trans & Rot  & Trans & Rot &Trans & Rot& Trans & Rot\\ 
    & (\%) & (deg/m)  & (\%) & (deg/m)  & (\%) & (deg/m)& (\%) & (deg/m) & (\%) & (deg/m) \\
    \midrule
        09&11.92&0.0360&-&-&17.84&0.0678&26.93&0.0954&9.26&0.0229 \\
        10&12.62&0.0343&8.11&0.0883&37.91&0.1778&24.69&0.0843&9.10&0.0221 \\
    \midrule
    % \textbf{Avg.} & \textbf{84.0}\\
    Avg & 12.27 & 0.0351 &8.11 &0.0883  & 28.88 &0.1228 &25.81& 0.0899& 9.18&\textbf{0.0225}\\
    % \hline
    \bottomrule
    \end{tabular}
    \end{center}
    \label{tab:kitti_compare}
    \end{table*}
    
    \begin{table}[!htbp]
        \caption{Comparison with Popular Geometry-based Methods}
        \begin{center}
        \begin{tabular}{c c c c c c c c c c c c c c c c c c}
        \toprule
        % \hline
        \multirow{4}*{Seq} & \multicolumn{2}{c}{LIBVISO2} &  \multicolumn{2}{c}{ORBSLAM}& \multicolumn{2}{c}{\multirow{2}*{Our Method }}\\
                           &  \multicolumn{2}{c}{(from \cite{Song2015MoncularScale})}&\multicolumn{2}{c}{(from \cite{raul2015orb})}  &\\
        %  % \hline%\hline
        \cline{2-3}  \cline{4-5}  \cline{6-7} 
            & Trans & Rot  & Trans & Rot  & Trans & Rot\\ 
        & (\%) & (deg/m)  & (\%) & (deg/m)  & (\%) & (deg/m)& \\
        \midrule
            09&4.04& 0.0143&15.30&0.0026& 9.26&0.0229\\
            10&25.20 &0.0388&3.68&0.0048 &9.10&0.0221\\
        \midrule
        % \textbf{Avg.} & \textbf{84.0}\\
        Avg &14.62 & 0.0266& 9.49&0.0037 &\textbf{9.18}&0.0225\\
        % \hline
        \bottomrule
        \end{tabular}
        \end{center}
        \label{tab:kitti_compare_ge}
        \end{table}
    
我们将我们的算法与其他基于深度学习和{基于传统几何的}方法的算法进行比较。我们的模型在KITTI序列00-08上进行训练,在KITTI 09和10上进行测试,数据分割与其他基于卷积神经网络(CNN)方法相同
\cite{zhan2018unsupervised,zhou2017unsupervised,yin2018geonet}。测试的RPE记录在表\ref{tab:kitti_compare}和\ref{tab:kitti_compare_ge}中。
由于SfM-Learner \cite{zhou2017unsupervised}和GeoNet \cite{yin2018geonet}的模型都是以自监督的方式进行无绝对尺度的训练,因此在评价前,它们的路径与地面路径真值是一致的。Zhan等人的模型\cite{zhan2018unsupervised}、DeepVO\cite{wang2017deepvo}和我们的方法都是用绝对尺度训练的,所以不需要对齐。单目模式的ORB-SLAM\cite{raul2015orb}和LIBVISO\cite{Geiger2011IV}的尺度也是与地面真实路径对齐的。

如表\ref{tab:kitti_compare}所示,我们的方法优于其他基于(自我运动模型主要由卷积层构建的)CNN的方法\cite{zhan2018unsupervised,zhou2017unsupervised,yin2018geonet},并与基于CNN-RNN{(RNN是Recurrent Neural Network的缩写)}的方法\cite{wang2017deepvo}竞争,后者可以利用时间信息优化姿势。

与两种效果较好的主流传统方法LIBVISO单目法(LIBVISO monocular \cite{Geiger2011IV})和ORB-SLAM单目法(ORB-SLAM monocular \cite{raul2015orb})比较,我们得到了较好的平均位移性能。


\subsection{Discussion}

在本节中,我们将对结果进行总结,对性能进行分析,并说明所提出的方法的局限性。
\subsubsection{The Efficiency of the Proposed Method}

从上面的实验结果来看,可以得出四个方面的结论。
1)运动聚焦不会带来太大额外的位姿偏移。平均RPE只有2\%左右,也就是说{车辆姿态}跑了100米后会有2米左右的漂移。路径可视化显示,重建后的路径是可以接受的。
2)运动解耦可以减少姿势位移。运动解耦利用y轴旋转和x轴平移的相关性来降低重构的姿势位移偏移。动态解耦的性能优于静态解耦。
对于动态解耦,需要对摄像机的位置有所要求。在上述实验中,摄像头位置是根据地面真实数据计算出来的。在实际操作中,如果训练数据是自己采集的,也可以直接测量得到。
3)运动聚焦{和解耦}可以从两个方面提高自我运动估计性能。首先,它{缩短了训练时间,所有的训练实验都在20个epoch内收敛,但所有运动的模型在大约60个epoch后才收敛,所以运动聚焦模型与所有运动模型相比,可以减少大约2/3的训练时间。另外,尽管运动聚焦和运动解耦后训练的地面真值姿势有一定的漂移,但{MFM}的测试性能比{AMM}更好。
4)在与基于几何的方法比较的过程中,发现基于几何的方法并不稳健,在不同序列上获得的性能各异。我们的方法更加稳定和稳健,平均性能更好。我们的方法也优于其他基于CNN的方法,在与利用RNN提高性能的DeepVO\cite{wang2017deepvo}比较时,我们也获得了更好的相对旋转性能,但位移性能比DeepVO差。


\subsubsection{Why Motion Focusing { and Decoupling} Works}

运动聚焦{和解耦}性能较好的原因有三点:首先,地面车辆的运动是受约束的,且分布不均衡,忽略不明显的运动不可能造成太大的姿态{位移}误差,这一点已在第\ref{sec:info_loss}节中得到证明。而这也是运动聚焦的根本基础};第二,不重要的运动太有限,没有足够的信噪比,所以模型在瞄准它们建模时,很容易被噪声干扰;第三,当我们只聚焦于二维运动时,训练任务变得{简单很多}。当采用{轻型}模型时,训练数据相对丰富。
实验证明,增加训练数据量确实能提高测试性能,如表\ref{tab:info_improve}所示。

\subsubsection{The Limitation} 

当汽车有近似{平面}运动时,所提出的方法可以获得更好的性能。
当有明显的{x轴旋转时,性能将下降。}如图\ref{fig:decouple}所示,序列09和10的RPE误差相对较高,因为在{这}两个序列中,地面车辆的运动不是{平面}的,如图\ref{fig:path_recon}所示。为了解决这个局限性,一个可行的方法是采用其他传感器,如IMU(惯性测量单元)来估计X轴的旋转,这可以作为所提出的模型所估计的平面运动的补充。

此外,{另一个限制是}因为我们的算法是基于摄像机是平放且是向前看的假设,所以如果摄像机有俯仰角,那么车辆的向前运动将被映射成Z轴运动和Y轴运动。如果摄像机有一个俯仰角,那么车辆的向前运动将被映射成Z轴运动和Y轴运动。在这种情况下,摄像机俯仰角$\sigma$应在训练前进行校准。然后,平移运动应该被重新定义为
\begin{equation}
    t_\sigma = \begin{pmatrix} 1 & 0 & 0 \\ 0 & \cos(\sigma) & -\sin(\sigma)\\ 0 & \sin(\sigma) & \cos(\sigma) \end{pmatrix} t
    \label{eq:pitch_correction}
\end{equation}

这些局限性也可以通过利用视觉里程测量的多模型结构来解决,每个子模型只关注地面车辆的一个维度运动,六自由度可以用六个分离的模型来学习,在这种情况下,应该分析模型权重分摊的影响。
\section{本章小结}
\label{sec:conclusion}
本文通过提出运动聚焦和运动解耦,将地面车辆的运动聚焦为两个自由度上的运动。实验证明了运动聚焦的可行性,并通过定量的姿态位移评估,进一步降低了姿态位移。基于地面车辆的旋转模型,我们提出了运动解耦,进一步降低了姿势位移。我们构建了一个轻型CNNs网络模型来模拟二自由度运动,它可以在CPU上实时运行。实验证明,运动聚焦和解耦可以提高小我运动估计性能,缩短收敛时间。在KITTI数据集上与其他方法的比较表明,所提出的方法的性能与其他端到端的视觉里程测量方法相当,甚至更好,并且比基于几何的方法更稳健。

           %
\include{data/VisLoc}               %deep
%\include{data/rotation_correction}
%\chapter{基于同态性误差的深度视觉里程计损失函数设计方法}
\label{ch:homovo}
%\section{引言}
% Problem
本论文前面两章(第\ref{ch:mvosr}章和第\ref{ch:deepsr}章)所介绍方法为传统视觉里程计算法的延伸,通过工程化的方式利用场景中的绝对尺度信息计算单目运动估计的绝对尺度并抑制尺度漂移,其中第三章所介绍的基于深度估计的尺度计算方法通过场景绝对尺度的离线训练学习,降低了尺度计算时对路面平面假设的依赖。但前面两章的工作均需要初始视觉里程计算法提供相对尺度下的运动估计和特征测量,本章将以另外一种思路实现单目视觉里程计绝对计算:通过映射学习的方式,端到端的实现单目视觉里程计的初始估计的尺度计算。

传统视觉里程计方法一般采用几何优化的方式计算相机运动:通过最小化重投影误差\cite{raul2015orb}或光度误差\cite{Engel-et-al-pami2018}
求解相机运动矩阵。但传统方法的精准度依赖于准确地传感器标定和大量参数(包括特征类型选取、特征提取阈值、特征匹配阈值等)的手动调节,而不同场景往往需要不同的参数组合,
且参数调节过程一般是工程师在环,如图\ref{fig:engineer_in_loop}所示,因为传统的视觉里程计方法的误差方程是不连续且相对参数不可导,无法直接利用梯度下降对这些参数进行调整,需要工程师观察算法效果然后根据经验和实际情况对关键参数进行手动调节。

\begin{figure}[h]
    \centering
    \includegraphics[width=0.95\columnwidth]{figures/homovo/engineer_in_loop-crop.pdf}
    \caption{工程师在环示意图}
    \label{fig:engineer_in_loop}
  \end{figure}

近年来,深度学习在图像分类、目标检测和语义分割等领域取得了很大的突破\cite{krizhevsky2012imagenet,girshick2015fast,long2015fully},很多研究者开始借助深度学习方法解决视觉里程计问题,以端到端训练和学习方式避免繁琐复杂的手动参数调节。
大多数基于学习的方法将视觉里程计定义为一个回归问题,拟合一个从两帧图像组成的图像对到相机运动的映射模型,拟合过程做出了如下假设:
1)输入的两帧图像具有相同的维度($\mathbf{I}_1,\ \mathbf{I}_2 \in \mathbb{I}^{w\times h \times c}$);2)输入图像内参一致(焦距$(f_x,f_y)$,光心$(c_x,c_y)$);
3)两帧图像存在重叠区域($\mathbf{I}_1 \cap \mathbf{I}_2 \neq \emptyset $)。

在映射学习过程中,映射输入的两帧图像一般在通道维度直接叠加到一起组成图像对$\mathbf{I}_m^n=\{\mathbf{I}_m,\mathbf{I}_n\} \in \mathbb{I}^{w\times h \times 2c}$;映射输出的三维空间相机运动存在6个自由度,包括3个平移运动
自由度和三个旋转运动的自由度。平移运动可直接用三维实数向量表示,而旋转运动本质上属于特殊正交群(Special Orthogonal Group)在表征方面较平移运动更为复杂,由于没有完美的表征方式存在,所以存在多种表征方式:欧拉角、四元数角轴(Axis-angle)
以及旋转矩阵等。其中四元数和旋转矩阵使用冗余的表达方式表示三个自由度,在训练过程中需要附加额外约束(四元数的模值为1,旋转矩阵为正交阵);
剩余的两个表征方式中欧拉角存在诸多问题(死锁问题,不连续问题等等),其中最为严重的问题是对于同一个旋转运动,欧拉角的表达方式不唯一,
只有通过增加额外限定才能获取唯一解,限定包括但不限于旋转角的最大幅度;角轴模式使用三个数表示三维空间的旋转,
其中这三个数所构成的向量表示旋转轴,向量的模表示绕着旋转轴的旋转角度,对于一种运动,角轴存在唯一的表达的方式。但角轴表示也存在一个问题,
就是角轴中的三个元素如果单独来看,并不具备清晰明了的物理意义,这方面不如欧拉角。所以在实际情况中,如果已知幅度限制等可以得到欧拉角的
唯一解,那么欧拉角是理想的旋转表征方式;如果无法得到欧拉角的唯一解,且不需要三个表征像素的物理意义,则可使用角轴表征方式。在本章算法中我们选择使用角轴模式。

综上分析,在基于学习的视觉里程计问题中,网络模型的输出为三维实数表征的平移向量和三维角轴表示的旋转向量,
这使运动估计的映射学习与深度神经网络所擅长的分割
检测问题有着本质的差异。分割检测问题的输出属于标签变量,其意义为人为赋予,本身不具备任何数学意义,举例来说:对于一个分割问题,
可以使用标签0表示道路、标签1表示建筑、标签2表示车辆,但0与1和建筑与交通并没有任何关系,道路(0)与建筑(1)的差异也不会比道路(0)与车辆(2)的差距小。

\begin{figure}[h]
    \centering
    \includegraphics[width=0.95\columnwidth]{figures/homovo/vo_diff.pdf}
    \caption{运动估计与语义分割的区别}
    \label{fig:vo_diff}
  \end{figure}
但视觉里程计的输出是具备数学意义的,其直接代表着相机的运动。若将输入图像对调换顺序,输出的运动应该变为之前的逆运动,而传统的网络结构和损失函数等并不能保证这一点。

然而,大部分端到端的视觉里程计方法依然直接使用神经网络模型去学习运动映射模型,并没有考虑到视觉里程计输出的数学意义,本文认为这是当前基于学习的视觉里程计精度不高的一个重要原因,
于是本文考虑映射性质重新定义基于学习的视觉里程计问题的损失函数(如图\ref{fig:homo_loss_problem}),使学习到的视觉里程计模型可以满足实际的映射关系:输入两张相同的图片,网络必然会输出单位运动(不运动);
将输入图像的顺序对调,那么将输出原运动的逆运动等等,即映射满足同态性模型(Homomorphism)。本章定义基于学习的视觉里程计问题为:
\textit{基于学习的视觉里程计是一种通过数据拟合方法得到从连续图像对到相机运动的满足同态性关系的映射模型的映射学习问题。}
\begin{figure}[t]
    \centering
    \includegraphics[width=0.95\columnwidth,height=0.6666662\columnwidth]{figures/homovo/loss_crop_small.pdf}
    \caption{损失函数示意图 }
    \label{fig:homo_loss_problem}
  \end{figure}

本文主要研究映射同态性以及群性质中的封闭性(Closure),
单位元(Identity Element)和逆元(Inverse Element)对运动估计学习的影响。
同时本文主要探索回答如下三个问题:1)卷积神经网络是否可以用来学习视觉里程计模型;2)如何定义合适的损失函数使卷积神经网络可以更好的学习;
3)如何提升基于学习的单目视觉里程计精度。
基于对上述问题的研究本文有如下贡献:
\begin{enumerate}
    \item 本文首次通过实验证明只用传统方法训练的卷积神经网络所表征的映射模型并不具备映射同态性,间接证明传统方法并没有学习到单目视觉里程计的运动本质;
    \item 针对传统训练方法的问题,以及对视觉里程计映射方式的分析,本文提出了基于映射同态性的损失函数,并通过实验证明同态性损失函数可以使训练得到的神经网络模型满足视觉里程计的运动特性,同时提高了运动估计的精度;
    \item 本文提出了基于图优化方法对网络输出的运动估计进一步提升的方法,同时实验证明本文所提出算法优于同时期其它端到端视觉里程计算法。
\end{enumerate}

本文结构安排如下:
首先在第\ref{sec:homo_approach}节介绍算法的数学原理和实现方式;然后在第\ref{sec:homo_experiments}节,在KITTI数据集\cite{geiger2012kitti}定性和定量的评价本章算法;
最后在\ref{sec:homo_conclusion}节总结本章工作。

\section{基于同态性的单目视觉里程计方法}
\label{sec:homo_approach}
本节中首先在第\ref{sec:homo_method_loss}节介绍同态性损失函数的数学原理和设计方法,然后在第\ref{sec:homo_method_train}节和第\ref{sec:homo_method_predict}节分别介绍视觉里程计网络模型的训练和预测方法。
\subsection{同态性损失函数}
\label{sec:homo_method_loss}
基于学习的视觉里程计从数学上讲属于监督学习中的回归问题(Regression),通过深度神经网络学习从图像对到相机运动的映射:
\begin{equation}
    F: \mathbb{R}^{2c\times w\times h} \to se(3)
\end{equation}
该映射可以表示为函数
\begin{equation}
    \mathbf{\tau} = \{\mathbf{t},\mathbf{r}\}= F(\mathbf{I}_m^n;\mathbf{\omega})
    \label{eq:objective}
\end{equation}
其中模型$F$多层神经网络, 网络参数记为$\mathbf{\omega}$, $\mathbf{I}_m^n =\{\mathbf{I}_m,\mathbf{I}_n\}$ 为输入图像对,
$\mathbf{\tau}\in se(3)$ 为机器人的平移运动和旋转运动组成的运动向量。
为了更好描述输入图像对与输出相机运动之间的映射关系,本章将图像对定义为广义集合(Set)
\begin{equation}
   \Omega_{\mathbf{I}_m^n} = \{\mathbf{I}_m^n\} \quad \mathbf{I}_m^n=\text{stack}(\mathbf{I}_m,\mathbf{I}_n),\quad \mathbf{I}_m,\  \mathbf{I}_n \in \Omega_{\mathbf{I}}
\end{equation}
同时定义在广义图像对集合上的二元运算$\otimes$ 
\begin{equation}
    \mathbf{I}_m^n = \mathbf{I}_m^t \otimes\mathbf{I}_t^n
\end{equation}

\begin{figure}
    \centering
    \includegraphics[width=0.9\textwidth]{homovo/operator.png}
    \caption{二元运算示意图}
    \label{fig:homo_operator}
\end{figure}
其表示图像对$\mathbf{I}_m^t$和$\mathbf{I}_t^n$可以通过二元运算$\otimes$得到$\mathbf{I}_m^n$,其中图像$\mathbf{I}_t$同时存在于两个图像对中且处于不同的次序,这是$\otimes$运算有意义的必然条件(如图\ref{fig:homo_operator})。
广义集合 $\{\mathbf{I}_m^n\}$和所定义的二元运算$\{\mathbf{I}_m^n,\otimes\}$可构成一个广义群(Group),可以证明其满足群的四条性质: 
\begin{enumerate}
    \item 封闭性(Closure):$\mathbf{I}_m^n \in \Omega_{\mathbf{I}_m^n}\quad \forall \mathbf{I}_m^t,\ \mathbf{I}_t^n \in  \Omega_{\mathbf{I}_m^n}$ \\
    \textbf{证明}: 
    \begin{enumerate}
        \item $\mathbf{I}_m^t \in \Omega_{\mathbf{I}_m^n} \Rightarrow \mathbf{I}_m, \mathbf{I}_t \in \Omega_{\mathbf{I}}$
        \item $\mathbf{I}_t^n \in \Omega_{\mathbf{I}_m^n} \Rightarrow \mathbf{I}_t, \mathbf{I}_n \in \Omega_{\mathbf{I}}$
        \item $\mathbf{I}_m, \mathbf{I}_n \in \Omega_{\mathbf{I}}\Rightarrow  \mathbf{I}_m^n \in \Omega_{\mathbf{I}_m^n} $
    \end{enumerate}
    \item 结合律(Associativity):$\mathbf{I}_m^t\otimes (\mathbf{I}_t^s \otimes \mathbf{I}_s^n) = (\mathbf{I}_m^t \otimes \mathbf{I}_t^s )\otimes \mathbf{I}_s^n $\\
    \textbf{证明}:
    \begin{enumerate}
        \item $\mathbf{I}_m^t\otimes (\mathbf{I}_t^s \otimes \mathbf{I}_s^n) = \mathbf{I}_m^t\otimes \mathbf{I}_t^n = \mathbf{I}_m^n$ 
        \item $(\mathbf{I}_m^t \otimes \mathbf{I}_t^s )\otimes \mathbf{I}_s^n = \mathbf{I}_m^s\otimes \mathbf{I}_s^n = \mathbf{I}_m^n$
        \item $\text{(a)},\text{(b)} \Rightarrow \mathbf{I}_m^t\otimes (\mathbf{I}_t^s \otimes \mathbf{I}_s^n) = (\mathbf{I}_m^t \otimes \mathbf{I}_t^s )\otimes \mathbf{I}_s^n$
    \end{enumerate}
    \item 单位元(Identity):$\exists \mathbf{I}_m^m,\ \mathbf{I}_n^n \quad  \mathbf{I}_m^m \otimes \mathbf{I}_m^n = \mathbf{I}_m^n\otimes \mathbf{I}_n^n = \mathbf{I}_m^n$
    \item 逆元 (Invertibility):$\mathbf{I}_m^n\otimes \mathbf{I}_n^m = \mathbf{I}_m^m$\\
    \textbf{证明}
    \begin{enumerate}
        \item $\forall \mathbf{I}_m^n \in \Omega_{\mathbf{I}_m^n},\  \exists  \mathbf{I}_n^m \in \Omega_{\mathbf{I}_m^n}$
        \item $\mathbf{I}_m^n\otimes \mathbf{I}_n^m = \mathbf{I}_m^m = \mathbf{I}_{iden}$
    \end{enumerate}
\end{enumerate}

其中单位元被定义为 $\mathbf{I}_{iden} = \{\mathbf{I}_n,\mathbf{I}_n\}=\mathbf{I}_n^n$,即任意两个相同的图像组成的图像对;
$\mathbf{I}_m^{n}$ 的逆元$ -\mathbf{I}_m^n = \{\mathbf{I}_n,\mathbf{I}_m\} = \mathbf{I}_n^{m}$,即将图像对的图像顺序对调。
已知网络输出的运动矩阵属于特殊欧式群(Special Euclidean Group)$SE(3)$,将输入图像对定义为群之后,视觉里程计学习被公式化为群的映射学习
\begin{equation}
    F: \Omega_{\mathbf{I}_m^n} \to se(3)
\end{equation}
显然从群 $\{\Omega_{\mathbf{I}_m^n},\otimes\}$ 到特殊欧式群$SE(3)$的映射满足同态性:
 \begin{equation}
     \exp(F(\mathbf{I}_m^n\otimes\mathbf{I}_n^t))=\exp(F(\mathbf{I}_m^n))\exp(F(\mathbf{I}_n^t))
 \end{equation}
 其中 $\mathbf{T} = \exp(\{\mathbf{t},\mathbf{r}\})$ 为李群(Lie Group)函数可以将李代数空间的运动向量$\{\mathbf{t},\mathbf{r}\} \in se(3)$
 映射到李群空间的运动矩阵 $\mathbf{T} \in SE(3)$ \cite{onishchik1993lie}。

根据同态性关系可以得到如下约束:
\begin{equation}
    \exp(F(\mathbf{I}_{iden})) = \mathbf{E} 
\end{equation}
\begin{equation}
    \exp(F(-\mathbf{I}_m^{n}))\exp(F(\mathbf{I}_m^{n})) = \mathbf{E} 
\end{equation}
\begin{equation}
    \exp(F(\mathbf{I}_{m}^{n}))\exp(F(\mathbf{I}_{n}^{t})) = \exp(F(\mathbf{I}_{m}^{t}))
\end{equation}
其中 $\mathbf{E}$ 为特殊欧式群$SE(3)$中的单位阵.

考虑到如上性质,可以定义了三个对应的损失函数:单位估计损失函数、逆运算损失函数和封闭损失函数。
\paragraph{单位估计损失函数} 
根据两个群之间的映射关系,当输入为图像对广义群中的单位元$\mathbf{I}_i^i = \{\mathbf{I}_i,\ \mathbf{I}_i\} \quad \forall \mathbf{I}_i \in \Omega_{\mathbf{I}}$时,深度神经网络所输出的运动应该为特殊欧式群中的单位元,于是我们可以定义输出运动与单位运动之间的差异定义损失函数:
\begin{equation} 
    L_{iden} = \sum_{i=1}^{n}\|F(\mathbf{I}_{i}^{i})-\{0,0\}\|_2 = \sum_{i=1}^{n}\|F(\mathbf{I}_{i}^{i})\|_2 
    \label{eq:identity_loss}
\end{equation}
其中$n$为所有训练样本。
\paragraph{逆运算损失函数}
当我们将输入的图像对$\mathbf{I}_m^n$做逆运算得到$\mathbf{I}_n^m = -\mathbf{I}_m^n$,网络对应的输出也会相应的求逆运算。根据这一性质,可定义逆运算的损失函数为:
\begin{equation}
    L_{inv} = \sum_{i=1}^{n-1} \|\log(\exp(F(\mathbf{I}_{i}^{i+1}))\exp(F(\mathbf{I}_{i+1}^{i})))\|_2
    \label{eq:inverse_loss}
\end{equation}
其中 $\{\mathbf{t},\mathbf{r}\} = \log(\mathbf{T})$ 为李群到对应李代数的映射函数将运动矩阵 $\mathbf{T} \in SE(3)$ 
映射为运动向量 $\{\mathbf{t},\mathbf{r}\} \in se(3)$。因为运动与逆运动乘积应为单位运动,所以可将其乘积与单位运动之间的差异定义为损失函数。

\paragraph{封闭性损失函数} 描述给定三组图像对$\mathbf{I}_m^n$,$\mathbf{I}_n^t$ 和$\mathbf{I}_m^t$,可知$\mathbf{I}_m^t = \mathbf{I}_m^n \otimes \mathbf{I}_n^t$,根据映射关系的同态性,可以得到
\begin{equation}
    \exp(F(\mathbf{I}_m^t)) = \exp(F(\mathbf{I}_m^n))\exp(F(\mathbf{I}_n^t))
    \label{eq:closure_con}
\end{equation}
在实现过程中,可使用三张连续的图片$\mathbf{I}_i$,$\mathbf{I}_{i+1}$,$\mathbf{I}_{i+2}$构建图像对$\mathbf{I}_i^{i+1}$,$\mathbf{I}_{i+1}^{i+2}$ 和$\mathbf{I}_i^{i+2}$。然后根据
公式\eqref{eq:closure_con},构建损失函数。
\begin{equation}
    L_{clo} = \sum_{i=1}^{n-2} \|\log(l_{i,i+1,i+2})\|_2
    \label{eq:closure_loss}
\end{equation}
$$ l_{i,i+1,i+2} = \exp(F(\mathbf{I}_{i}^{i+1}))\exp(F(\mathbf{I}_{i+1}^{i+2}))\exp(F(\mathbf{I}_{i}^{i+2}))^{-1}$$
除上述定位的三个同态性损失函数之外,本章同时使用了其它监督学习方法中使用的L2误差\cite{wang2017deepvo}
\begin{equation}
    L_2 = \sum_{i=1}^{n}\|F(\mathbf{I}_{i}^{i+1})-\{\mathbf{\underline{t}}_i,\mathbf{\underline{r}}_i\}\|_2
    \label{eq:l2_loss}
\end{equation}
其中$\{\mathbf{\underline{t}},\mathbf{\underline{r}}\}$ 为旋转和平移运动的真实值。
所以总损失函数为:
\begin{equation}
    L = L_2 + \alpha_1 L_{iden} +\alpha_2L_{inv}+\alpha_3L_{clo}
    \label{eq:loss}
\end{equation}
其中$\alpha_1$,$\alpha_2$和$\alpha_3$为三个同态性损失函数的权重系数。
%We simplely set the weight of each loss to 1 though better performance may be achieved by adjusting the weight parameters.
\subsection{训练方法}
\label{sec:homo_method_train}
%
\begin{figure}[t]
    \centering
    \includegraphics[width=0.95\columnwidth]{figures/homovo/homo_structure.pdf}
    \caption{网络架构示意图}
    \label{fig:homo_structure}
  \end{figure}

本文的核心贡献为损失函数的设计,网络架构主要基于\cite{zhou2017unsupervised}提出的多层卷积网络,网络架构如图\ref{fig:homo_structure}所示,网络主体由卷积层构成,卷积核均为正方形卷积核,每个卷积层后都经过一个反射线性单元(ReLU),网络的输出层为全局均值池化层(Global Average Pooling)。本章使用单通道灰度图像作为输入,并将其压缩为原始尺寸的一半,网络输出为归一化之后的运动向量
\begin{equation}
\{\mathbf{\underline{r}},\mathbf{\underline{t}}\} = (\{\mathbf{\underline{r}},\mathbf{\underline{t}}\} - \mu(\{\mathbf{\underline{r}},\mathbf{\underline{t}}\}))/\sigma(\{\mathbf{\underline{r}},\mathbf{\underline{t}}\})
\end{equation}
其中 $\mu(\{\mathbf{\underline{r}},\mathbf{\underline{t}}\})$ 和$\sigma(\{\mathbf{\underline{r}},\mathbf{\underline{t}}\})$
为训练集中运动向量的均值和标准差。 权重参数 $\alpha_1$,$\alpha_2$和$\alpha_3$均设为1。
使用Adam优化器,迭代优化损失函数并获取神经网络中的参数:
\begin{equation}
    \mathbf{\omega} = \argmin_{\mathbf{\omega}}\left( L_2 + \alpha_1 L_{iden} +\alpha_2L_{inv}+\alpha_3L_{clo}\right)
\end{equation}

\subsection{估计方法}
\label{sec:homo_method_predict}
由于在训练阶段增加了同态性损失函数,来强制保证所学习到的网络可以满足同态性映射约束。在网络模型训练好之后
可以利用这一性质来进一步优化所估计的运动。首先使用所学习到的模型估计出正向运动$T_i^{i+1}=\exp(F(\mathbf{I}_i^{i+1};\mathbf{\omega}))$,反向运行
$T_{i+1}^i=\exp(F(\mathbf{I}_{i+1}^i;\mathbf{\omega}))$以及隔帧运动$T_i^{i+2}=\exp(F(\mathbf{I}_i^{i+2};\mathbf{\omega}))$,然后根据这些运动之间的相互约束如图\ref{fig:homo_infer}构建图模型,对所估计的运动进行联合优化,得到更精准的运动估计。
\begin{figure}[t]
    \centering
    \includegraphics[width=0.95\columnwidth]{figures/homovo/homo_infer-crop.pdf}
    \caption{运动估计图优化示意图}
    \label{fig:homo_infer}
  \end{figure}

图中三个节点$\mathbf{P}_i$,$\mathbf{P}_{i+1}$和$\mathbf{P}_{i+2}$分别为三个时刻的位姿,每条边 ${e}_{ij}$ 约束运动
$\mathbf{T}_{i}^{j}$ 和姿态$\mathbf{P}_i$以及 $\mathbf{P}_j$ 

\begin{equation}
    e_{ij} = \|\mathbf{P}_i\mathbf{T}_{i}^{j}\mathbf{P}_j^{-1}\|_2
\end{equation}
总能量函数为六条边的能量总和
\begin{equation}
    E = \sum_{i,j \in \{t,t-1,t-2\}} e_{ij}
\end{equation}
在优化过程中,由于无法获得每个运动的可靠性,所以信息增益矩阵设为相同,此外$\mathbf{P}_i$固定,通过最小化能量函数$E$获取优化之后的$\mathbf{P}_{i+1}$和$\mathbf{P}_{i+2}$。
\begin{equation}
    \label{eq:optimize_out}
    \mathbf{P}_{i+1},\mathbf{P}_{i+2} = \argmin E
\end{equation}

\section{同态性视觉里程计验证实验}
\label{sec:homo_experiments}
本节在KITTI视觉里程计测评数据集\cite{geiger2012kitti}上,设计了三个实验用以验证本章所提出算法的有效性:首先在第\ref{sec:training_loss}节,验证本文算法所基于的两个基本假设,包括传统损失函数的局限性假设和本文基本思想的可行性假设;
然后在第\ref{sec:exp_homo}节,通过对比实验证明本文所提出的同态性损失函数的有效性;最后在第\ref{sec:exp_homo_compare}节,将本文结果与其它算法进行比较。所有定量比较均使用相对位姿误差(Relative Pose Error,RPE) \cite{geiger2012kitti}作为评价指标,用以评价不同距离下相对定位误差的平均值。



在算法实现方面,本章主要包括两个部分,分别为用于映射学习的深度学习部分和用于预测优化的图优化部分。本章使用Python程序设计语言基于PyTorch深度学习框架\cite{paszke2017automatic}实现本文算法中的深度学习部分,在训练过程中学习率设置为0.01,网络训练100个周期,整个训练周期内
学习率不做调整。图优化部分使用C++程序设计语言基于图优化开源框架g2o\cite{kummerle2011g}实现,为保证程序设计的整体性和两个模块之间相互调用的便捷性,使用PyBind11将优化部分封装成Python接口。在测试过程中,使用装置Ubuntu操作系统配备因特尔Intel Core i7处理器和英伟达NVIDIA GeForce GTX 1060显卡的笔记本电脑,运动估计时每帧大约耗时6.3毫秒,占用显存423MB。

%and compiled to be a python library by PyBind11.
\subsection{同态性损失函数验证假设}
%
\label{sec:training_loss}
本节将在第一个实验验证本文算法所依赖的两条假设:
1)仅依靠$L2$损失函数,卷积神经网络无法拟合出满足映射同态性性质运动估计模型;
2)使用同态性损失函数,网络可以正常收敛。
大多数基于卷积神经网络的视觉里程计映射学习方法一般仅使用$L2$损失函数,在映射学习过程中忽略了运动估问题本身的同态性约束,本文提出仅使用$L2$损失函数训练出来的网络模型并不能满足运动估计所必须具备的性质,若使用$L2$损失函数训练出来的网络可以满足同态性映射模型,那本文损失函数的设计则没有必要性,所以假设1是本文工作的必要性假设;此外,在增加了复杂的同态性损失函数之后,网络存在无法收敛的可能性,若无法收敛则说明卷积神经网络在一定程度上不能去学习视觉里程计的映射的函数,若无法学习,则损失函数设计则不具备可行性,所以假设2是本文工作的可行性假设。

\subsubsection{L2损失函数局限性验证}
\label{sec:homo_assumation_1}
为了验证$L2$损失函数的局限性,需要设计实验证明使用$L2$损失函数学习到的映射模型并不能满足同态性。本节选择验证其是否可以满足$F(\mathbf{I}_n^m;\mathbf{\omega})=F(\mathbf{I}_m^n;\mathbf{\omega})^{-1}$,在具体实现上,本节使用KITTI视觉里程计00-08序列作为训练集,使用$L2$损失函数训练映射模型100个周期(epoch),在训练过程中同时记录网络的损失函数下降曲线以及不同周期的训练集正向误差和训练集反向误差。
如图\ref{fig:homo_training_loss}中红线所示,网络损失函数逐渐减小,代表网络可以正常收敛;然而在模型的正向训练误差(图\ref{fig:homo_training_error}中红色实线)随着训练逐渐
减小时,模型的反向训练误差(图\ref{fig:homo_training_error}中红色虚线)却没有变小。反向误差反应所训练模型对图像反向运动的预测能力:当调转数据图像对的顺序,网络理论上可以输出原运动的逆运动,即$F(\mathbf{I}_n^m;\mathbf{\omega})= F(\mathbf{I}_m^n;\mathbf{\omega})^{-1}$。综上,实验证明仅基于$L2$损失函数训练出来的卷积神经网络并无法直接具备这一性质,进而可以得出结论:
卷积神经网络无法在L2损失函数下学习到满足同态性性质的运动估计模型,本文的必要性假设成立。
\begin{figure}[h]
    \centering
    \begin{subfigure}[b]{0.48\textwidth}
    \includegraphics[width=\textwidth]{homovo/training_loss_compare.pdf}
    \caption{损失函数}
    \label{fig:homo_training_loss}
    \end{subfigure}
    \begin{subfigure}[b]{0.48\textwidth}
        \includegraphics[width=\textwidth]{homovo/training_error_compare.pdf}
        \caption{测试误差}
    \label{fig:homo_training_error}
    \end{subfigure}
    \caption{假设验证训练结果图}
    {\label{fig:homo_assumation}}
    \end{figure}
    %
\subsubsection{同态性损失函数可行性验证}
同态性损失函数的可行性验证相对简单,只需要证明网络在同态性损失函数的训练下可以正常收敛即可。本节同样使用KITTI:00-08作为训练集,训练卷积神经网络,网络架构与验证1一致,但损失函数更改为融合同态性误差的损失函数(即公式\eqref{eq:loss})。网络的损失函数曲线如图\ref{fig:homo_training_loss}中绿线所示,训练集正向误差和反向误差分别如图\ref{fig:homo_training_error}中绿色实线和虚线所示。从图中可以看出网络可以正常收敛,即本文的可行性假设成立。同时,我们观察到增加同态性损失函数之后的其它两个现象:
1)网络的反向训练误差同样逐渐减小;2)网络正向误差下降的也更快。这两个现象可以定性的说明同态性损失函数对运动估计的提升效果,下一节的实验中将具体分析同态性损失函数对运动估计学习的影响。



\subsection{同态性损失函数有效性验证实验}
\label{sec:exp_homo}
为证明本文提出的同态性损失函数的有效性,本节设计对比实验来分析不同损失函数下的测试误差。在具体实现上,控制网络结构相同,训练和测试数据相同(00-08用作训练,09用来测试)以及其它所有超参(Hyper Parameter)相同,仅按照损失函数的不同分为三个实验组,分别为:
\begin{enumerate}
    \item \textit{正向训练组(Forward Training)}: 训练过程中仅使用$L2$损失函数 (公式\eqref{eq:l2_loss}); 
    \item \textit{正反向训练组(Cycle Training)}: 使用$L2$损失函数和逆元损失函数(公式\eqref{eq:inverse_loss});
    \item \textit{群训练组(Group Training)}: 使用$L2$损失函数和本文提出的全部损失函数,包括逆元损失函数、单位元损失函数 (公式\eqref{eq:identity_loss})、和
    封闭性损失函数(公式\eqref{eq:closure_loss})。
\end{enumerate}
将每个实验组训练100个周期,并将每个周期得到的模型通过三种测试模式在KITTI-09上测试,
1)\textit{正向测试(Forwatd Testing)}:正向输入训练数据;
2)\textit{反向测试(Backward Testing)}:反向输入训练数据;
3)\textit{优化测试(Backward Testing)}:使用本文提出的后端优化方式对输出结果进行优化后测试。
测试误差曲线记录于图\ref{fig:homo_loss_compare}。
\begin{figure}[h]
    \centering
    \begin{subfigure}[b]{0.48\textwidth}
        \includegraphics[width=\textwidth]{homovo/backward_testing_compare.pdf}
        \caption{反向测试结果}
        \label{fig:backward_testing}
    \end{subfigure}
    \vspace*{2mm}
    \begin{subfigure}[b]{0.48\textwidth}
        \includegraphics[width=\textwidth]{homovo/cycle_optimized.pdf}
        \caption{正反向测试优化}
        \label{fig:cycle_opti}
    \end{subfigure}
    \vspace*{2mm}
    \begin{subfigure}[b]{0.48\textwidth}
        \includegraphics[width=\textwidth]{homovo/group_optimized.pdf}
        \caption{群训练优化}
        \label{fig:group_opti}
    \end{subfigure}
    \begin{subfigure}[b]{0.48\textwidth}
        \includegraphics[width=\textwidth]{homovo/forward_testing_compare.pdf}
        \caption{正向测试}
        \label{fig:forward_testing}
    \end{subfigure}
    \caption{不同损失函数的测试结果图}
    {\label{fig:homo_loss_compare}}
\end{figure}
同时,为了定量地分析结果,本节评价了每个实验组最后30个训练周期的平均测试误差,并记录于表\ref{tab:loss_function}。
\begin{table}[t]
    \caption{不同训练方法的测试误差}
    \label{tab:loss_function}
\begin{center}
\begin{tabular}{l c c c c c c c c}
\toprule
\multirow{1}*{sequence}&\multicolumn{1}{c}{Forward Training}&\multicolumn{1}{c}{Cycle Training}&\multicolumn{1}{c}{Group Training}\\
\midrule
    09 R(deg/m) &0.0289&0.0177&\textbf{0.0151}\\ 
    09 t(\%)&16.52&8.62&\textbf{8.04}\\

\bottomrule
\end{tabular}
\end{center}
\end{table}


从图\ref{fig:homo_loss_compare}中曲线及表中数据可以观察到:
1)前向训练组训练得到网络模型的前向测试误差不断减小(图 \ref{fig:forward_testing}中红色虚线),而反向测试误差
并没有持续降低(图\ref{fig:backward_testing}中红色实线),收敛于60\%左右。
由此可知,前向训练的网络仅可以进行前向测试,而无法估计反向数据的运动,即前向训练网络无法直接泛化到反向数据。这一观察再次通过实验证明第\ref{sec:training_loss}所验证的$L2$损失函数局限性假设。
2)正反向训练组和群训练组得到网络模型的前向测试的误差(图\ref{fig:forward_testing}中的黄色和绿色虚线)和反向测试误差(图\ref{fig:backward_testing}中的黄色和绿色实线)同时都在减小。这一观察可以说明同态性损失函数训练出来的网络模型可以同时估计前向运动和反向运动,在一定程度上可以满足同态性。
3)对于正反向模型和群模型,根据其直接输出结果,利用所提出的后端图优化方法对其进行优化(优化方法描述于公式\eqref{eq:optimize_out}),
发现优化的结果被进一步提升(如图\ref{fig:cycle_opti}和图\ref{fig:group_opti})。
4)正反向模型和群模型的测试前向测试效果均优于前向模型,群模型的测试结果优于正反向模型(见表\ref{tab:loss_function})。

基于上述分析可以得出如下结论:正反向训练和群训练可以使模型具备预测反向运动的能力;二者可以提高前向估计的精度;
正反向训练和群训练因考虑映射的同态性而提供了结果后期优化的可能性,且数据表明优化之后精度得到了提升。总而言之,本文提出的同态性损失函数有效。


%
%


\subsection{与同时期其它算法精准度比较}
\label{sec:exp_homo_compare}
%
\begin{table}[!htbp]
\caption{单应性运动估计与其他算法精度比较}
\begin{center}
\begin{tabular}{c c c c c c c c c c c c c c}
\toprule
% \hline
\multirow{4}*{Seq} & \multicolumn{2}{c}{Zhan et al.} &\multicolumn{2}{c}{DeepVO} & \multicolumn{2}{c}{Zhou et al.}& \multicolumn{2}{c}{GeoNet}& \multicolumn{2}{c}{\multirow{2}*{Our Method }}\\
                   & \multicolumn{2}{c}{(from \cite{zhan2018unsupervised})}  & \multicolumn{2}{c}{(from \cite{wang2017deepvo})}&\multicolumn{2}{c}{(from \cite{zhou2017unsupervised})} &\multicolumn{2}{c}{(from \cite{yin2018geonet})} &\\
%  % \hline%\hline
\cline{2-3}  \cline{4-5}  \cline{6-7} \cline{8-9} \cline{10-11} 
    & Trans & Rot  & Trans & Rot  & Trans & Rot &Trans & Rot& Trans & Rot\\ 
& (\%) & (deg/m)  & (\%) & (deg/m)  & (\%) & (deg/m)& (\%) & (deg/m) & (\%) & (deg/m) \\
\midrule
    09&11.92&0.0360&-&-&17.84&0.0678&26.93&0.0954&8.04&0.0151 \\
    10&12.62&0.0343&8.11&0.0883&37.91&0.1778&24.69&0.0843& 6.23&0.0097 \\
\midrule
% \textbf{Avg.} & \textbf{84.0}\\
Avg & 12.27 & 0.0351 &8.11 &0.0883  & 28.88 &0.1228 &25.81& 0.0899 &\textbf{7.14}&0.0124\\
% \hline
\bottomrule
\end{tabular}
\end{center}
\label{tab:homovo_kitti_compare}
\end{table}

\begin{table}[!htbp]
    \caption{单应性运动估计与传统算法比较}
    \begin{center}
    \begin{tabular}{c c c c c c c c c c c c c c c c c c}
    \toprule
    % \hline
    \multirow{4}*{Seq} & \multicolumn{2}{c}{LIBVISO2} &  \multicolumn{2}{c}{ORBSLAM}& \multicolumn{2}{c}{\multirow{2}*{Our Method }}\\
                       & \multicolumn{2}{c}{(from \cite{Song2015MoncularScale})}&\multicolumn{2}{c}{(from \cite{raul2015orb})}  &\\
    %  % \hline%\hline
    \cline{2-3}  \cline{4-5}  \cline{6-7} 
        & Trans & Rot  & Trans & Rot  & Trans & Rot \\ 
    & (\%) & (deg/m)  & (\%) & (deg/m)  & (\%) & (deg/m)\\
    \midrule
        09&4.04& 0.0143&15.30&0.0026 &8.04&0.0151 \\
        10&5.20 &0.0388&3.68&0.0048 & 6.23&0.0097 \\
    \midrule
    % \textbf{Avg.} & \textbf{84.0}\\
    Avg & 14.62 & 0.0266& 9.49&0.0037 &\textbf{7.14}&0.0124\\
    % \hline
    \bottomrule
    \end{tabular}
    \end{center}
    \label{tab:homovo_kitti_compare_geo}
    \end{table}

\iffalse
\begin{table*}[!htbp]
\caption{Comparison of translation and rotation errors for our method versus other visual odometry methods on the KITTI benchmark.}
\begin{center}
\begin{tabular}{c c c c c c c c c c c c c c c c c c}
\toprule
% \hline
\multirow{4}*{Seq} & \multicolumn{2}{c}{Zhang et al.} &\multicolumn{2}{c}{DeepVO} & \multicolumn{2}{c}{Zhou et al.}& \multicolumn{2}{c}{GeoNet}& \multicolumn{2}{c}{ENG} &  \multicolumn{2}{c}{Mahjourian et al.}& \multicolumn{2}{c}{\multirow{2}*{Our Method }}\\
                   & \multicolumn{2}{c}{(from \cite{zhan2018unsupervised})}  & \multicolumn{2}{c}{(from \cite{wang2017deepvo})}&\multicolumn{2}{c}{(from \cite{zhou2017unsupervised})} &\multicolumn{2}{c}{(from \cite{yin2018geonet})}&   \multicolumn{2}{c}{(from \cite{Lee2015MoncularScale})}&\multicolumn{2}{c}{(from \cite{mahjourian2018unsupervised})}  &\\
%  % \hline%\hline
\cline{2-3}  \cline{4-5}  \cline{6-7} \cline{8-9} \cline{10-11} \cline{12-13} \cline{14-15}
    & Trans & Rot  & Trans & Rot  & Trans & Rot &Trans & Rot& Trans & Rot& Trans & Rot &Trans & Rot\\ 
& (\%) & (deg/m)  & (\%) & (deg/m)  & (\%) & (deg/m)& (\%) & (deg/m) & (\%) & (deg/m) &(\%) & (deg/m)&(\%) & (deg/m)\\
\midrule
    09&11.92&0.0360&-&-&17.84&0.0678&&&&&&&8.04&0.0151 \\
    10&12.62&0.0343&8.11&0.0883&37.91&0.1778&&&- &-&-&- & 6.23&0.0097 \\
\midrule
% \textbf{Avg.} & \textbf{84.0}\\
Avg & 12.27 & 0.051  & &  &  & &  &&& &&\\
% \hline
\bottomrule
\end{tabular}
\end{center}
\label{tab:kitti_compare}
\end{table*}
\fi

本节将本章提出的方法与其它算法比较以证明本方法的有效性。训练数据与其它方法
\cite{zhou2017unsupervised,zhan2018unsupervised,yin2018geonet}
一致:使用KITTI数据集中的序列00-08训练(共九条轨迹),KITTI数据集中的09-10测试(共两条轨迹)。在测试集中的轨迹可视化于图\ref{fig:path_result},图中同时可视化了Zhang等人的工作和ORB-SLAM2不带闭环修正的结果。从图中可以看出,本文算法所预测轨迹更接近于真实轨迹。
\begin{figure}[h]
    \centering
    \mysubfigure[09]{0.48}{
    \includegraphics[width=\textwidth]{homovo/path/09_compare_orb.pdf}
    }{\label{fig:09_path}}
    \mysubfigure[10]{0.48}{
    \includegraphics[width=\textwidth]{homovo/path/10_compare_orb.pdf}
    }{\label{fig:10_path}}
    \caption{与其它算法轨迹对比图}
    {\label{fig:path_result}}
    \end{figure}
为了定量的评测算法的性能,本节将测试结果使用RPE误差评测\cite{geiger2012kitti},并将测试误差
记录于表\ref{tab:homovo_kitti_compare}和表\ref{tab:homovo_kitti_compare_geo}中。

由于本文算法使用监督学习模式训练,且测试数据集与训练数据集尺度分布相近,所以无需做尺度对齐。

由于非监督学习在相对尺度空间进行学习,无法直接学习到绝对尺度,在使用RPE评测Zhou等人的算法
\cite{zhou2017unsupervised}以及GeoNet \cite{yin2018geonet}之前,首先对结果进行了尺度对齐。
从表\ref{tab:homovo_kitti_compare}中看出,本章方法的精度要高于这两种方法。
Zhan等人\cite{zhan2018unsupervised}也是非监督方法,但其使用双目数据进行训练,可以通过已知的基线长度学习到运动的绝对尺度,
在评测时不对其做尺度对齐,由表\ref{tab:homovo_kitti_compare}中数据可以看出,本文算法优于Zhan等人的算法,精度提高约40\%。

由以上分析可以初步得出结论,尽管非监督训练方法可以不依赖于数据标签,提高了训练便捷性,但其精度相对较为弱势。

DeepVO \cite{wang2017deepvo} 是截至本工作完成时的精度最高的算法,其使用监督学习模式,融合CNN和RNN,考虑到了视觉运动估计的
上下文信息。为了使结果具备可比性,本节采用与之相同的训练数据(KITTI 00,02,08和09),在KITTI10的测试RPE误差为8.08\%和0.0123(deg/m)。
稍微优于DeepVO。可知在本文提出的同态性损失函数训练下,即使考虑上下文信息,也可以得到很好的精度。

此外本节还将算法与传统的视觉里程计算法进行比较,比较结果如表\ref{tab:homovo_kitti_compare_geo}所示,
可知,本章算法优于LIBVISO\cite{Geiger2011IV} ,
与ORB-SLAM单目版本\cite{raul2015orb}不相上下。

\section{本章小结}
\label{sec:homo_conclusion}
本章提出了一种基于映射同态性的视觉里程计损失函数设计方法。我们研究了运动估计网络的输入图像对于输出运动之间的映射关系,从数学原理上分析出运动估计模型应具备映射同态性,提出直接使用深度神经网络通过普通损失函数学习出的运动估计模型无法满足映射关系,并通过设计实验证明这一论点。
在此基础上,依据同态性约束和群的封闭性、单位元和逆元等性质,创新地提出了三个同态性损失函数,并通过实验证明同态性损失函数的引入可使网络学习到的映射模型可以满足同态性关系。同时,依据运动估计网络的同态性,在运动估计阶段增加了图优化模块,并通过实验证明图优化模块可以进一步提高运动估计的精度。此外我们将所提出的算法与同时期其它基于学习以及基于几何计算的传统方法进行对比,本文提出的运动估计网络取得了更好的结果。


%\chapter{基于区域一致性深度视觉里程计网络架构设计方法}
\label{ch:padvo}
在前文(第\ref{ch:homovo}章)中,基于单目视觉里程计网络输入到输出之间映射关系的群同态性而提出的同态性损失函数提升了映射学习算法的精度。本章工作将介绍一种基于单目视觉运动估计区域一致性的网络架构设计方法,以及其对运动估计的精度和泛化性的提升效果。

%点名冗余的重要性
什么是区域一致性?单目视觉里程计的核心问题为根据相邻图像的变化计算机器人的运动,但在计算过程中整幅图像的信息往往是冗余的。在传统方法中,基于特征点的运动估计方法可以仅使用五个特征点计算出机器人运动的本征矩阵 (Essential Matrix) 并进一步分解得到机器人的旋转运动和平移运动\cite{nister2004efficient};不基于特征点的直接法也具备通过优化局部图像区域的光度误差计算机器人运动的能力\cite{engel2014lsd}。总之,从局部图像的变化进行运动估计在原理上存在可行性;同时,依据各个局部区域所计算出来的运动势必是一致的,因为这些运动均对应于机器人的实际运动(如图\ref{fig:pad_des}所示)。
\begin{figure}[h]
  \centering
  \includegraphics[width=0.95\columnwidth]{figures/padvo/pad_des.pdf}
  \caption{区域一致示意图}
%  \caption{\wsnote{font is too small}}
  \label{fig:pad_des}
\end{figure}
本章工作首先定义依据各个子区域所计算出的运动必然一致这一性质为区域一致性,并基于区域一致性提出了一种新的视觉里程计网络架构。在具体实现上,该网络首先独立地根据每块图像子区域进行机器人运动计算,同时评估各个子区域所计算运动的可靠性,进而依据所估计的运动和可靠性综合计算出机器人的最终运动,网络架构如图\ref{fig:pad_problem}所示。这种架构设计方式存在两个优点:首先,因为不同场景的局部图像区域的相似性会高于整幅图片的相似性,使用图像区域作为运动估计的基本单位有潜力提升算法的泛化性,其次,该架构将运动估计拆解到多个局部区域进行,结合可靠性评估的最终运动估计会降低算法结果的不确定性。
\begin{figure}[h]
  \centering
  \includegraphics[width=0.95\columnwidth]{figures/padvo/problem.pdf}
  \caption{区域一致模型网络架构图}
%  \caption{\wsnote{font is too small}}
  \label{fig:pad_problem}
\end{figure}

在区域一致性网格架构的研究和实现中,本章做出如下贡献:
\begin{enumerate}
    \item 本章提出了一种新的运动估计网络架构设计方法。局部图像区域被首次用作运动估计映射学习的基本单元,并通过融合的方式得到机器人运动。这种方法可以降低运动估计对全局图像颜色分布的依赖,能够在一定程度上提高算法在不同数据集上的泛化能力。
    \item 针对图像子区域的可靠性差异,本章提出增加一个分支网络预测所估计运动的可靠性,使问题定义更加完备。同时本章利用区域运动可靠性计算相机运动可靠性,其可为失效判断和多传感器融合提供基础。
    \item 本章所提出的算法在公开数据集上进行了验证,实验表明,该算法在一定程度上提高了基于学习的视觉里程计的精度和泛化性(本章算法的代码实现已经开源\footnote{https://github.com/TimingSpace/PADVO})。
\end{enumerate}


% structure
%
本章结构安排如下:
首先在第\ref{sec:padvo_approach}节介绍算法的数学原理和实现方式;然后在第\ref{sec:experiments}节,我们在KITTI数据集\cite{geiger2012kitti}和Robotcat数据集\cite{RobotCarDatasetIJRR}定性和定量地评价本章算法;
最后在\ref{sec:conclusion}节总结本章工作。


\section{区域一致性视觉里程计方法}
\label{sec:padvo_approach}
%\input{content/table_network_structure.tex}
本节将依次介绍区域一致网络架构设计方法的数学基础、网络结构和训练细节。

\subsection{区域一致性方法的数学基础}
\subsubsection{区域一致性的数学定义}
基于深度学习的端到端视觉里程计问题可定义为映射学习问题:
映射的定义域为图像对$\{\mathbf{I}_t,\mathbf{I}_{t+1}\}$,
(表示为 $\mathbf{I}_t^{t+1}$ ),映射值域为与图像对相对应的相机运动向量$\mathbf{\tau}_t^{t+1}$。
\begin{equation}
  \mathbf{\tau}_t^{t+1} = F(\mathbf{I}_t^{t+1}|\mathbf{\omega})
\end{equation}
其中,$F(\mathbf{I}_t^{t+1}|\mathbf{\omega})$表示网络模型,$\mathbf{\omega}$为模型中待学习的权重参数。
用于训练网络的训练集可以表示为$\{\mathbf{I}_t^{t+1},\mathbf{\tau}_t^{t+1}\}\quad t=0,1,...,N$,监督学习模式下,可通过优化器最小化损失函数以获取模型参数:
\begin{equation}
  \mathbf{\omega} = \argmin_w \sum_{t=0}^N \|F(\mathbf{I}_t^{t+1}|\mathbf{\omega}) - \left(\mathbf{\tau}_t^{t+1}\right)\|
\end{equation}

上述方法直接将整幅图像所在的高维空间集合作为映射函数的定义域。假设输入图像的尺寸为$w\times h$,通道数为$c$,那么输入图像对$\mathbf{I}_m^n \in \mathbb{I}^{w\times h\times 2c}$,其中$\mathbb{I}$为0到255之间所有整数的集合。图像尺寸越大,$\mathbb{I}^{w\times h\times 2c}$集合所在的空间维度越高,而集合中全集的元素数量与集合维度呈指数映射关系,所以,当图像尺寸较大时,需要更多次更均匀的训练样本才能较好的拟合集合全集\cite{lynn2002principles}。然而,由于现有数据集样本数量的不足以及场景内抽样样本的各个维度的相互耦合,样本空间一般仅分布在集合全集中的一小部分,使不同数据集的分布相差较大,学习到的映射模型无法很好的直接迁移。综上,本章认为全集空间的过大以及其间接导致的不完全采样是视觉里程计映射学习泛化性较差的一个重要原因。

既然整幅图像所在的全局空间过大,那是否可以将图像的局部区域作为视觉里程计映射学习的基本单元呢?答案是肯定的,观察发现,对于视觉里程计
问题,相机的运动估计并不依赖于整幅完整的图片,前后帧组成的图像对中信息量足够重叠子区域也可以用来估计相机运动,且所估计的运动应该与使用完整图像估计的
运动相同。如果存在整幅图像到运动估计的映射函数$F(\mathbf{I}_t^{t+1};\mathbf{\omega}) = \mathbf{\tau}_t^{t+1}$,那么每个图像子区域都存在一个到某一相同运动的映射函数$^pF(^p\mathbf{I}_t^{t+1};^p\mathbf{\omega})$,即
\begin{equation}
  ^pF(^p\mathbf{I}_t^{t+1};^p\mathbf{\omega}) = \mathbf{\tau}_t^{t+1} \, p=0,1,2,...,N_p-1,
  \label{eq:patch_mapping}
\end{equation}
其中$^p\mathbf{I}_t^{t+1}$为图像对${\mathbf{I}}_t^{t+1}$中的一块子区域。
$N_p$ 为子区域的数量,子区域之间可以相互重叠。上述公式显然成立,因为不同子区域反应的运动均为相机的实际运动,本章定义图像对不同子区域必然映射到相同运动的这一性质为\textbf{区域一致性}。使用图像子区域进行运动估计可以有效降低映射学习中定义域所在的空间维度以及定义域样本全集的总量,进而提升训练集在全集中的分布比,理论上可以达到较好的泛化性。
在本章所提出算法中,根据视觉里程计的数学计算原理,本章使用相同的映射模型去学习不同图像子区域到相机运动的映射,同时为了降低算法的复杂度和参数量,各个子区域的模型共享参数,于是区域一致性可以重新数学描述为:
\begin{quote}
  对任意图像对$\mathbf{I}_t^{t+1}$,若存在图像对到相机运动的映射函数$F$使$F(\mathbf{I}_t^{t+1};\mathbf{\omega})=\mathbf{\tau}_t^{t+1}$,则必然存在从图相对子区域到相机运动的映射函数$F_{pa}$,可以将图像中的任意子区域$^p\mathbf{I}_t^{t+1}$映射到相同的相机运动,即
\begin{equation}
  F_{pa}(^p\mathbf{I}_t^{t+1};\mathbf{\omega}_{pa}) = \mathbf{\tau}_t^{t+1} \quad \forall\  ^p\mathbf{I}_t^{t+1} \subset \mathbf{I}_t^{t+1} 
  \label{eq:patch_mapping_2}
\end{equation}
\end{quote}

在依据每个子区域估计出相机运动之后,可以使用所有运动的平均值求取机器人的最终运动:

\begin{equation}
    F(\mathbf{I}_t^{t+1}) = \sum_{p=0}^{N_p-1}\frac{^p\mathbf{\tau}_t^{t+1}}{N_p}
\end{equation}
根据方差法则可知,如果每个子区域的估计方差为$\sigma^2$,那么最终运动估计的方差为
\begin{equation}
    \sigma^2 \left(F(\mathbf{I}_t^{t+1})\right) = \frac{\sigma^2}{N_p}
    \label{eq:pad_var_1}
\end{equation}
即用区域一致性的估计方法降低运动估计的不确定性。
由于并没有考虑各个子区域可靠性的差异性,本章将这种模式定义为均一区域一致性(PADVO without Reliability),将在实验章节进行评测 \ref{sec:experiments}。

\subsubsection{区域的可靠性估计和差异性表征}
然而直接在子区域空间上进行映射学习存在两个问题:1)因为视觉里程计的计算依赖于每个区域的像素位置,故虽然不同图像区域均可以用来计算相机运动,但每个区域所处的图像位置需要予以不同考虑;2)并不是所有的图像区域都具备可以估计出相机运动的足够信息,无信息或少信息区域(如图\ref{fig:pad_featureless}中所示的蓝天区域和无特征的路面区域)到机器人运动之间的映射存在歧义性。
针对第一个问题,本章通过在图像对的通道维度上增加了坐标层让网络可以感知图像的不同位置,下文将主要介绍第二个问题的解决方案。
\begin{figure}[t]
  \centering
  \includegraphics[width=0.9\columnwidth]{figures/padvo/featureless.png}
  \caption{特征较少区域示意图}
  \label{fig:pad_featureless}
\end{figure}

为了解决不同子区域的可靠性差异问题,需要赋予各个子区域所估计出来运动的一个相应的方差,于是基于学习的运动估计问题重新数学描述为

\begin{equation}
    ^p\mathbf{\tau}_t^{t+1}=F_{pa}(^p\mathbf{I}_t^{t+1}|\mathbf{\omega}) \in N( \mathbf{\tau}_t^{t+1},\  ^p\mathbf{\sigma}^2) \quad p=0,1,..,N_p-1
\end{equation}
其中 $^p\mathbf{\sigma}^2$ 表示子区域估计运动的方差。即网络在使用每个子区域估计相机运动的同时,也估计出所估计运动的可靠性,
然后通过子区域运动的加权平均获取相机的最终估计运动$\mathbf{\tau}_t^{t+1}$,对应方差越大的子区域所估计的运动,权重越小:
\begin{equation}
    \mathbf{\tau}_t^{t+1} = F(\mathbf{I}_t^{t+1}) = \frac{\sum_{p=0}^{N_p-1}\tfrac{1}{^p\mathbf{\sigma}^2}{^p\mathbf{\tau}_t^{t+1}}}{\sum_{p=0}^{N_p-1}\tfrac{1}{^p\mathbf{\sigma}^2}}
    \label{eq:ego_mean}
\end{equation}
最终运动的估计方差为
\begin{equation}
    \sigma \left(F(\mathbf{I}_t^{t+1})\right) =  \frac{1}{\sum_{p=0}^{N_p-1}\frac{1}{^p\mathbf{\sigma}^2}}
    \label{eq:pad_var}
\end{equation}

然而由于所使用表征旋转运动的角轴(李代数$\mathbf{\tau} \in \textit{se}(3)$ ),\textit{se}(3)并非欧式空间,其二元运算被定义为
\begin{equation}
  \label{eq:lie}
  \ln(\exp(\mathbf{\tau}_1)\exp(\mathbf{\tau}_2)) = \mathbf{\tau}_1+\mathbf{\tau}_2 +[\mathbf{\tau}_1,\mathbf{\tau}_2]+\dots
\end{equation}
其中$[\,\cdot \,]$为李括号\cite{bourbaki2003elements}。
理论上并不能直接使用
公式\eqref{eq:ego_mean}进行加权平均
但由于每一个区域估计出的运动差异较小,所以$[\mathbf{\tau}_1,\mathbf{\tau}_2]\approx 0$,其之后的其它项同样接近于0。
于是
\begin{equation}
    \label{eq:lie_approx}
    ln(\exp(\mathbf{\tau}_1)\exp(\mathbf{\tau}_2)) \approx \mathbf{\tau}_1+\mathbf{\tau}_2 
\end{equation}
所以本章依然直接使用公式\eqref{eq:ego_mean} 运动的加权平均以降低运算量。

从公式\eqref{eq:pad_var}和公式\eqref{eq:pad_var_1}中可以看出,加权平均之后的方差会比之前的方差变小,这也是区域一致性视觉里程计有效性的数学依据,
但方差变小的前提是可以对各个子区域的方差进行准确的估计,那么如何估计子区域的方差呢?
本文提出同时使用网络估计运动的均值和方差,网络架构与损失函数将在下一章说明。



\subsection{区域一致性网络架构和损失函数}
\label{sec:padvo_stru_loss}

\begin{table*}[!htbp]
    \caption{区域一致性模型网络架构}
    \label{tab:network_structure}
    \begin{center}
    \begin{tabular}{c c c c c c c c c c c c c c c c c c c c c c c c}
    \toprule
    % \hline
    \multirow{2}*{Network} & \multicolumn{3}{c}{Ego-Motion} &\multicolumn{3}{c}{Reliability}\\
    %  % \hline%\hline
    \cline{2-4}  \cline{5-7}  
        & kernel size & feature number   & stride & kernel size & feature number   & stride\\
    \midrule
        1&7&16&2&7&16&2\\
        2&5&32&2&5&32&2\\
        3&3&64&2&3&64&2\\
        4&3&128&1&3&128&1\\
        5&3&256&2&3&256&2\\
        6&3&512&1&3&16&1\\
        7&1&6&1&1&1&1\\
    % \hline
    \bottomrule
    \end{tabular}
    \end{center}
    \end{table*}
    



网络架构设计灵感来源于YOLO\cite{redmon2016you},YOLO是一种高速物体检测方法,其将输入图像分散成多个区域同时输入给模型以获取更快的物体识别速率。
本文同样将输入图像分为多个子区域,对于每一个子区域,网络同时估计相机运动和运动的可靠性。网络从架构上分为两个子模块:运动估计网络$F_{pa}(^p\mathbf{I}_t^{t+1}|\mathbf{\omega}_{pa})$用来估计相机运动;可靠性估计网络$F_{re}(^p\mathbf{I}_t^{t+1}|\mathbf{\omega}_{re})$
用来估计相机运动的方差,网络架构如图\ref{fig:structure}所示。 在模型细节上,两个模型都只含有卷积层,核函数尺寸$k$,特征层数量$n$,以及
跳跃宽度(Stride)$s$,描述于表 \ref{tab:network_structure}。每个卷积层都跟随着一个组标准化(Group Normalization)层
和一个修正线性单元层(Rectified Linear Unit,ReLU。此外,运动估计网络和可靠性估计网络在前五层共享参数。
\begin{figure}[t]
  \centering
  \includegraphics[width=0.9\columnwidth]{figures/padvo/structure.pdf}
  \caption{区域一致模型网络细节图}
  \label{fig:structure}
\end{figure}

从网络架构中可以看出,本文其实并没有显式地将图像分成多个区域,而是通过最后一层的感受野隐式的限定了每一个运动估计所取用的输入图像区域,区域之间存在相互
重叠覆盖。当前的网络架构隐式地将输入图像分割为$\frac{w}{16}\times\frac{h}{16}$个子区域,每个子区域的大小为$87\times87$。其中$w$和$h$
为属于图像对的宽度和高度。

在网络得到估计运动以及运动可靠性之后,损失函数最朴素的构建方式为
\begin{equation}
    \mathbf{L} = \sum_t \sum_p {F_{re}\left(^p\mathbf{I}_t^{t+1}\right)}\|F_{pa}\left(^p\mathbf{I}_t^{t+1}\right)-\mathbf{\underline{\tau}}_t^{t+1}\|_2
\end{equation}
其中根据每个子区域的可靠性,给每个所估计的运动的误差赋予不同的权重,可靠性越高,对应的误差权重就越大,因为系统更应具备让可靠性更高的
区域估计出更准的运动的能力;而对于可靠性不高的区域则给予较小的误差权重,因为对于根本不包含太多信息的图像区域(比如蓝天)网络是无法进行运动估计的,强制要求其学到准确的
运动只会适得其反。
\begin{figure}[h]
  \centering
  \includegraphics[width=0.9\columnwidth]{figures/padvo/regularization.pdf}
  \caption{正则化示意图}
  \label{fig:regularization}
\end{figure}

但如果仅用上述公式作为损失函数训练网络,系统则倾向于不断降低所有区域的可靠性,这样损失函数同样会变小。如图\ref{fig:regularization}所示,图中可视化了当估计运动与真实运动的误差为固定值时,不同的可靠性指数对整体误差的影响,当不加正则化时,可靠性越小,系统误差越小。所以需要对可靠性进行正则化
\begin{equation}
  \mathbf{L} = \sum_t \sum_p  {^pL_t}
\end{equation}
\begin{equation}
    ^pL_t = \exp(a)\|d\|_2+ \lambda\|a\|_2
\end{equation}
其中$d =|F_{pa}\left(^p\mathbf{I}_t^{t+1}\right)-_g\mathbf{\tau}_t^{t+1}\|_2 $ 为运动误差, $a = F_{re}\left(^p\mathbf{I}_t^{t+1}\right)$为可靠性网络的输出。$^p\mathbf{l}_t$ 相对于 $a$ 的偏导为:
\begin{equation}
   \frac{ \partial ^pL_t}{\partial a}= \exp(a)\|d\|_2 + 2\lambda a.
   \label{eq:pad_loss}
\end{equation}
给定特定的误差,网络收敛时
\begin{equation}
    -2\lambda a \exp(-a) = \|d\|_2
    \label{eq:var_by_reli}
\end{equation}
所以使用 $-2\lambda a \exp(-a)$ 作为所估计运动的方差。
\begin{figure}[h]
  \centering
  \includegraphics[width=0.9\columnwidth]{figures/padvo/reliability.pdf}
  \caption{可靠性与运动误差关系示意图}
  \label{fig:reliability}
\end{figure}
%\subsection{证明收敛}
\subsection{区域一致性网络训练与推断}
本章使用Python程序设计语言实现了所提出的基于区域一致性的网络模型,主要基于PyTorch深度学习框架。为了降低运算量,训练时KITTI数据集压缩至$640\times180$,RobotCar数据集压缩至$450\times180$(由于模型不涉及到全连接层,本文算法支持不同维度的图像输入);运动使用角轴表示$\textit{se}(3)$
使用Adam优化器优化公式\eqref{eq:pad_loss}所描述的损失函数,其中$\lambda$ 设置为 0.1,学习率设置为0.01,批大小设置为60。

推断时,子区域估计出的的相机运动 $^p\mathbf{\tau}$ 和可靠性 $^pa$ 由训练好的网络模型 $F$ 和 $F_a$计算得到
然后使用公式\eqref{eq:ego_mean}计算相机运动,计算时根据每一个子区域的可靠性$^pa$计算方差:
\begin{equation}
  ^p\sigma^2= -2\lambda (^pa) \exp(-(^pa)) 
\end{equation}



\begin{algorithm}
  \caption{区域一致性视觉里程计算法框架}
  \KwIn{训练集$\{\mathbf{I}^t,\mathbf{P}_t\} \quad t= 0,1,...,N$;测试图像序列$\mathbf{I}^{i'}\quad i = 0,1,...,M$}
  \KwOut{测试图像对应的位姿$\mathbf{P}_i^{'} \quad i = 0,1,...,M$}
  训练阶段:\\
  构建图像对$\{\mathbf{I}_t^{t+1}\}$,同时计算相机运动$\mathbf{\tau}_t^{t+1} = \log(\mathbf{P}_t^{-1}\mathbf{P}_{t+1})$\\
  得到用于训练模型的输入输出组合$\Omega_{\{\mathbf{I}_t^{t+1},\mathbf{\tau}_t^{t+1}\}}=\{\mathbf{I}_t^{t+1},\mathbf{\tau}_t^{t+1}\}\quad t=0,1,...,N$\\

  \While{$\text{epoch}<100$}
  {   
      \For {$\forall \{\mathbf{I}_t^{t+1},\mathbf{\tau}_t^{t+1}\} \in \Omega_{\{\mathbf{I}_t^{t+1},\mathbf{\tau}_t^{t+1}\}}$}
      {
        Loss $L = 0$\\
        \For{$\forall\; ^p\mathbf{I}_t^{t+1} \subset \mathbf{I}_t^{t+1}$}
        {
          $^p\mathbf{\tau} =F_{pa}(\mathbf{I}_t^{t+1};\mathbf{\omega}_{pa})$ \\
          $^pa =F_{re}(\mathbf{I}_t^{t+1};\mathbf{\omega}_{re})$ \\
          $L = L + \exp(a)\|d\|_2+ \lambda\|a\|_2$ \\

        }
      }
      $\mathbf{\omega}_{pa} = \mathbf{\omega}_{pa} - \delta \frac{\partial L}{\partial \mathbf{\omega}_{pa}}$\\
      $\mathbf{\omega}_{re} = \mathbf{\omega}_{re} - \delta \frac{\partial L}{\partial \mathbf{\omega}_{re}}$\\
  }
  测试阶段:\\
  $\mathbf{P}_0^{'} = E$
  \For {$i \in (1,M)$}
  {
    $\mathbf{I}_{i-1}^{i} = \{\mathbf{I}_{i-1},\mathbf{I}_{i}\}$\\
    \For{$\forall\; ^p\mathbf{I}_i^{i+1} \subset \mathbf{I}_i^{i+1}$}
    {
      $^p\mathbf{\tau} =F_{pa}(\mathbf{I}_{i-1}^{i};\mathbf{\omega}_{pa})$ \\
      $^pa =F_{re}(\mathbf{I}_{i-1}^{i};\mathbf{\omega}_{re})$ \\
      $ ^p\sigma^2= -2\lambda ^pa \exp(-^pa)$\\
    }
    根据公式\eqref{eq:ego_mean}计算$\mathbf{\tau}_{i-1}^{i}$\\
    $\mathbf{P}_i^{'} =\mathbf{P}_{i-1}^{'}\exp(\mathbf{\tau}_{i-1}^{i})$
  }
  得到 $\mathbf{P}_i^{'} \quad i = 0,1,...,M$
\label{alg:padvo}
\end{algorithm}


%\subsection{自监督训练}
%所提出方法的核心思想为运动估计网络以图像区域为基本单位,每块区域分别输出所估计的运动,以及运动的可靠性。将本文提出的运动估计网络融入到自监督训练框架中存在一个问题,就是我们没有相机运动真实值,通过重投影误差同时约束深度估计和运动估计。区域一致思想的融入有两种方案:1)针对每一个区域所估计出的运动均计算重投影误差; 2)评估运动估计的离散程度作为损失函数,因为区域一致思想的本质为各个子区域所估计的运动一致。


\section{区域一致性网络架构验证实验}
\label{sec:experiments}

%\input{content/table_pad_compare.tex}
本节设计了四个实验用以证明本文所提出方法的有效性,实验在KITTI视觉里程计数据集\cite{geiger2012kitti}和RobotCar数据集 \cite{RobotCarDatasetIJRR}进行开展。算法效果使用平均姿态误差(Relative Pose Error,RPE)\cite{geiger2012kitti}进行评估。首先在第\ref{sec:es_pad_check}节进行消冗实验,验证区域一致模型对算法性能的提升;然后在第\ref{sec:pad_es_relia}节评测可靠性估计的准确性;在第\ref{sec:pad_es_compare}节与其它视觉里程计算法进行效果对比用以验证整体算法是否有效;最后,第\ref{sec:pad_es_visu}节对神经网络中间的特征层进行了可视化,用以观察网络模型的学习细节。

\begin{figure}[t]

  \centering
  \begin{subfigure}[b]{0.48\textwidth}
    \includegraphics[width=\textwidth]{figures/padvo/kitti_pose_06.pdf}
    \caption{KITTI 06}
    \label{fig:model_test_kitti_06}
  \end{subfigure}
  \vspace*{2mm}
  \begin{subfigure}[b]{0.48\textwidth}
    \includegraphics[width=\textwidth]{figures/padvo/kitti_pose_08.pdf}
    \caption{KITTI 08}
    \label{fig:model_test_kitti_08}
  \end{subfigure}
  \vspace*{2mm}
  \begin{subfigure}[b]{0.48\textwidth}
    \includegraphics[width=\textwidth]{figures/padvo/kitti_pose_10.pdf}
    \caption{KITTI 10}
    \label{fig:model_test_kitti_10}
  \end{subfigure}
  \begin{subfigure}[b]{0.48\textwidth}
    \includegraphics[width=\textwidth]{figures/padvo/robocar_1511_02_remap_pose.pdf}
    \caption{RobotCar}
    \label{fig:model_test_RobotCar}
  \end{subfigure}

  \caption{不同模型测试效果图}
  {\label{fig:model_test}}
  \end{figure}



\subsection{区域一致性有效性验证实验}
\label{sec:es_pad_check}
本节将通过对比实验验证本文所提出的区域一致性对基于学习的视觉里程计算法的提升。在实验中,本节将对比如下三个不同模型的测试效果:
\begin{enumerate}
  \item 非区域一致模型(记为 Non-PAD ):与本章所提出的运动估计模型架构基本一致,但不考虑区域一致性,使用全局均值池化 (Global Average Pooling)对各个区域的运动估计直接进行平均池化并作为输出结果(与Zhou 等人\cite{zhou2017unsupervised}相似);
  \item 带坐标层的非区域一致模型(记为 Non-PAD with Coor):此模型同非区域一致模型,但输入层增加图像坐标层;
  \item 均一区域一致模型(记为 PAD without Reli):运动模型使用所提出的区域一致模型,但不估计每个子区域的可靠性,在网络训练学习过程中使用相同的区域可靠性计算损失函数;
\end{enumerate}

以上三种模型使用相同的训练数据、优化方法、学习率以及批处理大小等超参,分别在不同KITTI和RobotCar数据集上进行两组训练和六组测试。

首先,在KITTI数据集00(记为KITTI-00)序列上分别训练上述三个模型,然后分别在KITTI-06,KITTI-08和KITTI-10进行测试,测试轨迹分别可视化于图 \ref{fig:model_test_kitti_06}、\ref{fig:model_test_kitti_08}和\ref{fig:model_test_kitti_10}中。
然后,同样在RobotCar的一个数据集上进行训练(记为RobotCar-00),并在RobotCar中不同的数据(记为RobotCar-01)上进行测试,测试轨迹可视化如图\ref{fig:model_test_RobotCar}所示。从图中可以看出,使用区域一致模型训练得到的轨迹比使用其它模型得到的结果更接近真实轨迹。

但测试轨迹与真实轨迹的相似性只能定性评价模型的测试性能,为了得到更明确的结果,本节使用RPE评价指标定量地评价了每条轨迹的测试误差,并记录于表\ref{tab:pad_model_compare}和表\ref{tab:pad_model_compare_2}中,其中表\ref{tab:pad_model_compare}记录了KITTI-00中的训练模型的三组测试实验,表\ref{tab:pad_model_compare_2}中记录了RobotCar-01中的训练模型的一组测试实验和RobotCar数据集与KITTI数据集的交差验证实验。
\begin{table*}[h]
\caption{区域一致模型验证实验记录}
\begin{center}
\begin{tabular}{c c c c c c c c c c c c c }
\toprule
% \hline
\multirow{3}{*}{Model}  &\multicolumn{2}{c}{KITTI-06} &\multicolumn{2}{c}{KITTI-08} & \multicolumn{2}{c}{KITTI-10}& \multicolumn{2}{c}{Avg}\\
%  % \hline%\hline
\cline{2-3}  \cline{4-5}  \cline{6-7} \cline{8-9} 
& Trans & Rot  & Trans & Rot &Trans & Rot& Trans & Rot\\ 
& (\%) & (deg/m)  & (\%) & (deg/m)& (\%) & (deg/m) & (\%) & (deg/m)\\
\midrule
 Non-PAD     &27.66&0.1593&26.22&0.1473&23.67&0.108&25.85&0.1382 \\
 Non-PAD-C   &25.38&0.1519&\textbf{19.13}&0.1005&20.53&0.0563&21.67&0.1029\\
 \midrule
 PAD wo Reli &\textbf{16.54}&\textbf{0.0566}&19.34&\textbf{0.0861}&\textbf{13.19}&\textbf{0.0349}&\textbf{16.35}& \textbf{0.0592}\\
% \hline
\bottomrule
\end{tabular}
\end{center}
\label{tab:pad_model_compare}
\end{table*}

\begin{table*}[h]
  \caption{区域一致模型交差验证实验记录}
  \begin{center}
  \begin{tabular}{c c c c c c c c c c c }
  \toprule
  % \hline
Training &\multicolumn{4}{c}{RobotCar-01} &\multicolumn{2}{c}{KITTI-00} \\
\cline{2-7}
Testing  &\multicolumn{2}{c}{RobotCar-02} &\multicolumn{2}{c}{KITTI-10} & \multicolumn{2}{c}{RobotCar-02}\\
  %  % \hline%\hline
  \cline{1-3}  \cline{4-5}  \cline{6-7} \cline{8-9} 
  \multirow{2}{*}{Model}  & Trans & Rot  & Trans & Rot &Trans & Rot\\ 
  & (\%) & (deg/m)  & (\%) & (deg/m)& (\%) & (deg/m) \\
  \midrule
   Non-PAD     &\textbf{14.23}&\textbf{0.0405}&34.79&0.1612&52.82&0.1663\\
   Non-PAD-C   &19.54&0.0550&35.84&0.1825&32.40&0.1135\\
   \midrule
   PAD wo Reli &16.23&0.0510&\textbf{23.73}&\textbf{0.1189}&\textbf{29.76}&\textbf{0.0784} \\
  % \hline
  \bottomrule
  \end{tabular}
  \end{center}
  \label{tab:pad_model_compare_2}
  \end{table*}

%lr=0.01 bs=40
从表\ref{tab:pad_model_compare}的实验数据中可以发现均一区域一致模型(PAD wo Reli)相比于其它两个非区域一致模型的测试误差有着显著地下降:区域一致模型相比于不带坐标层的
非区域一致模型,在KITTI数据集上平均相对位置误差从25.85\%降到了16.35\%左右,提升了9.5\%,相比于带坐标层的非区域一致模型也提高了5.3\%;此外均一区域一致模型在三组测试集中平均相对姿态误差相比于非区域一致模型从0.1382deg/m和0.1029deg/m降到了0.05deg/m。

从表\ref{tab:pad_model_compare_2}的实验数据中可以看出区域一致模型提升了算法跨数据集泛化性:在RobotCar-01上训练的均一区域一致模型在KITTI-10上的相对位置误差下降超过10\%,相对姿态误差下降了0.05deg/m;在KITTI-00上训练的均一区域一致模型相比于非区域一致模型也取得了大幅提升。

综上,对比实验的测试数据中说明均一区域一致模型可以提升映射学习的精度和泛化能力。


\subsection{可靠性估计有效性验证实验}
\label{sec:pad_es_relia}
\begin{table*}[h]
    \caption{可靠性性能对比实验记录}
    \begin{center}
    \begin{tabular}{c c c c c c c c c c c c c }
    \toprule
    % \hline
    \multirow{3}{*}{Model}  &\multicolumn{2}{c}{KITTI 06} &\multicolumn{2}{c}{KITTI 08} & \multicolumn{2}{c}{KITTI 10}& \multicolumn{2}{c}{Avg}\\
    %  % \hline%\hline
    \cline{2-3}  \cline{4-5}  \cline{6-7} \cline{8-9} 
    & Trans & Rot  & Trans & Rot &Trans & Rot& Trans & Rot\\ 
    & (\%) & (deg/m)  & (\%) & (deg/m)& (\%) & (deg/m) & (\%) & (deg/m)\\
    \midrule
     PAD wo Reli &16.54&0.0566&19.34&0.0861&\textbf{13.19}&0.0349&16.35& 0.0592\\
     PADVO       &\textbf{14.11}& 0.0669&\textbf{17.46}& 0.0829&16.25& 0.0705&\textbf{15.94}&0.0734\\
    % \hline
    \bottomrule
    \end{tabular}
    \end{center}
    \label{tab:padvo_reli_compare_1}
    \end{table*}
    
    \begin{table*}[h]
      \caption{跨数据集可靠性性能对比实验记录}
      \begin{center}
      \begin{tabular}{c c c c c c c c c c c }
      \toprule
      % \hline
    Training &\multicolumn{4}{c}{R-01} &\multicolumn{2}{c}{KITTI 09} \\
    \cline{2-7}
    Testing  &\multicolumn{2}{c}{Robot 02} &\multicolumn{2}{c}{KITTI 10} & \multicolumn{2}{c}{Robot 02}\\
      %  % \hline%\hline
      \cline{1-3}  \cline{4-5}  \cline{6-7} \cline{8-9} 
      \multirow{2}{*}{Model}  & Trans & Rot  & Trans & Rot &Trans & Rot\\ 
      & (\%) & (deg/m)  & (\%) & (deg/m)& (\%) & (deg/m) \\
      \midrule
       PAD wo Reli &16.23&0.0510&\textbf{23.73}&0.1189&29.76&0.0784 \\
       PADVO       &\textbf{12.37}& 0.0361&24.20& 0.0705&\textbf{29.33}& 0.1663\\
      % \hline
      \bottomrule
      \end{tabular}
      \end{center}
      \label{tab:padvo_reli_compare_2}
      \end{table*}
    
    %lr=0.01 bs=40
本节分别从定性和定量两个角度对区域可靠性和帧可靠性的有效性进行验证。
\subsubsection{区域可靠性验证实验}

本节通过区域可靠性估计对运动估计效果的影响来定量评价区域可靠性的有效性。我们训练带可靠性估计的区域一致模型,并使用第\ref{sec:es_pad_check}节中的无可靠性区域一致模型作为对照组,控制其它条件一致,对运动估计结果进行评价,定量结果记录于表\ref{tab:padvo_reli_compare_1}和表\ref{tab:padvo_reli_compare_2}中。在表中的六组对比实验中有四组实验带可靠性的区域一致模型优于不带可靠性估计的区域一致模型。

但发现可靠性估计网络仅提升了平移运动估计的精度,但对旋转运动估计并没有明显提升,甚至略有下降。
本节分析这种现象的一个可能原因为:KITTI数据集和RobotCar数据集为使用车辆采集的数据,以前后平移运动为主,大部分数据中车辆并没有旋转运动,导致可靠性估计网络所估计的可靠性以平移运动为主,所以其对平移运动有一定提升,但对旋转运动没有效果。
\begin{figure}[h]
  \centering
  \mysubfigure[]{0.45}{
  \includegraphics[width=\textwidth, trim={0 0 5cm 0},clip]{padvo/relia_1.png}
  }{\label{fig:reli_1}}
  \vspace*{2mm}
  \mysubfigure[]{0.45}{
  \includegraphics[width=\textwidth, trim={0 0 5cm 0},clip]{padvo/relia_2.png}
  }{\label{fig:reli_2}}
  \vspace*{2mm}
  \mysubfigure[]{0.45}{
  \includegraphics[width=\textwidth, trim={0 0 5cm 0},clip]{padvo/relia_3.png}
  }{\label{fig:reli_3}}
  \vspace*{2mm}
  \mysubfigure[]{0.45}{
  \includegraphics[width=\textwidth, trim={0 0 5cm 0},clip]{padvo/relia_4.png}
  }{\label{fig:reli_4}}
  \vspace*{2mm}
  \mysubfigure[]{0.45}{
  \includegraphics[width=\textwidth, trim={0 0 5cm 0},clip]{padvo/relia_5.png}
  }{\label{fig:reli_5}}
  \mysubfigure[]{0.45}{
  \includegraphics[width=\textwidth, trim={0 0 5cm 0},clip]{padvo/relia_6.png}
  }{\label{fig:reli_6}}
  \caption{区域可靠性效果图}
  \label{fig:pad_pad_reli}
  \end{figure}
本节定性的可视化了区域估计方差与区域实际误差(如图\ref{fig:pad_pad_reli}所示),从图\ref{fig:pad_pad_reli}中可以看出二者的相关性。

\subsubsection{帧可靠性验证实验}
本节定义帧可靠性有效性的评价原则为:有效的可靠性估计值应该与运动估计的实际误差负相关,即根据可靠性所计算出的运动方差估计值(公式\eqref{eq:var_by_reli}))与运动估计的实际误差正相关。于是,我们将运动方差估计值和运动误差实际值的相关性作为可靠性的有效性指标,相关性越高,则可靠性估计的有效性越高。相关性评价存在很多种定量评价方式,包括皮尔森相关性指数(Pearson Correlation)、斯皮尔曼相关性指数(Spearman Correlation)、互信息(Mutual Information,MI)和KL散度(KL Divergence)等。
其中互信息和KL散度用于评价两个概率分布模型之间的差异,而本文需要评价的为数值差异,所以旋转用于数值评价的皮尔森指数和斯皮尔曼指数。皮尔森指数相比于斯皮尔曼指数对数据的数值差异更敏感,于是本文采用皮尔森指数来评价运动方差估计值和运动误差实际值之间的相关性。
\begin{equation}
  \label{eq:re}
  \mathbf{m} = \rho\left({\|\mathbf{e}\|_2},{-\mathbf{r}\exp(-\mathbf{r})}\right)
\end{equation}
\begin{equation}
  \rho(X,Y) =\text{corr}(X,Y)= \frac{\text{cov}(X,Y)}{\sigma(X)\sigma(Y)}
\end{equation}

其中,$\mathbf{e} \in R^{n}$ 为每一帧所估计的运动与真实运动比较的误差;
 $\mathbf{r} \in R^{n}$为所估计的运动的可靠性,
$n$为图像的数量。从第
\ref{sec:padvo_stru_loss}节可知,$-\mathbf{r}\exp(-\mathbf{r})$为所估计的运动方差。
\begin{table}[h]
\caption{帧可靠性评价对比}
\label{tab:frame_reliability}
\begin{center}
\begin{tabular}{c  c c c c c }
\toprule
% \hline
\multicolumn{2}{c}{Seq} &Same & Random& Our & Ideal\\
\midrule
Training&00-08   &0.000&0.068&0.383&1.000\\
\midrule   
\multirow{3}*{Testing} & 09 &0.000&0.071&0.234&1.000\\  
     &10      &0.000&0.062&0.360&1.000\\
     &RobotCar&0.000&0.067&0.243&1.000\\
% \hline
\bottomrule
\end{tabular}
\end{center}
\end{table}

由于本文是首次提出映射学习中帧可靠性的数值量化,所以无法与其它工作进行定量比较,为更加直观的理解定量评价结果的数值,我们评价如下情况的皮尔森指数,并记录于表\ref{tab:frame_reliability}:
\begin{enumerate}
  \item 当根据所估计的方差计算的误差等于真实误差,此时可靠性完美,皮尔森指数为1;
  \item 当将方差赋予随机值,评价估计误差与真实误差的皮尔森指数;
  \item 当将方差赋予相同值,评价估计误差与真实误差的皮尔森指数,其值应为0;
  \item 评价真实估计的方差与真实误差的皮尔森指数。
\end{enumerate}

表中00-08为训练集,09、10和RobotCar为测试集。
从表中可以看出,对于帧可靠性,所估计出的结果无论在训练集还是测试集,明显优于随机可靠性,
值得说明的是在KITTI上训练的网络在RobotCar
数据集测试依然有效,可以证明方法的泛化性。
此外本节可视化了估计方差与实际误差之前的对比图(如图\ref{fig:frame_reliability}所示),从图中可以定性的看出二者的相关性较强。

\begin{figure}[h]
  \centering
  \begin{subfigure}[b]{0.95\textwidth}
    \includegraphics[width=\textwidth]{padvo/00_08_relia_error.pdf}
    \caption{Training Set (KITTI 00-08)}
    \label{fig:reli_error_1}
  \end{subfigure}
  \vspace*{2mm}

  \begin{subfigure}[b]{0.95\textwidth}
    \includegraphics[width=\textwidth]{padvo/10_relia_error.pdf}
    \caption{Testing Set (KITTI 10)}
    \label{fig:reli_error_2}
  \end{subfigure}
  \vspace*{2mm}

  \begin{subfigure}[b]{0.95\textwidth}
    \includegraphics[width=\textwidth]{padvo/RobotCar_relia_error.pdf}
    \caption{Testing Set (RobotCar)}
    \label{fig:reli_error_3}
  \end{subfigure}
  \caption{帧可靠性效果图}
  \label{fig:frame_reliability}
\end{figure}

%improve compare without reliability
%use reliability to filter
\subsection{与其它视觉里程计方法比较}
\label{sec:pad_es_compare}
\begin{table*}[!htbp]
\caption{区域一致网络与其他基于学习的方法比较结果}
\begin{center}
\begin{tabular}{c c c c c c c c c c c c c c}
\toprule
% \hline
\multirow{4}*{Seq} & \multicolumn{2}{c}{Zhan et al.} &\multicolumn{2}{c}{DeepVO} & \multicolumn{2}{c}{SfM-Learner}& \multicolumn{2}{c}{GeoNet}&  \multicolumn{2}{c}{\multirow{2}*{Our Method }}\\
                   & \multicolumn{2}{c}{(from \cite{zhan2018unsupervised})}  & \multicolumn{2}{c}{(from \cite{wang2017deepvo})}&\multicolumn{2}{c}{(from \cite{zhou2017unsupervised})} &\multicolumn{2}{c}{(from \cite{yin2018geonet})} &\\
%  % \hline%\hline
\cline{2-3}  \cline{4-5}  \cline{6-7} \cline{8-9} \cline{10-11} \cline{12-13}
    & Trans & Rot  & Trans & Rot  & Trans & Rot &Trans & Rot& Trans & Rot\\ 
& (\%) & (deg/m)  & (\%) & (deg/m)  & (\%) & (deg/m)& (\%) & (deg/m) & (\%) & (deg/m) \\
\midrule
    09&11.92&0.0360&-&-&17.84&0.0678&26.93&0.0954&7.77&0.0476 \\
    10&12.62&0.0343&8.11&0.0883&37.91&0.1778&24.69&0.0843&6.68&0.0485 \\
\midrule
% \textbf{Avg.} & \textbf{84.0}\\
Avg & 12.27 & 0.0351 &8.11 &0.0883  & 28.88 &0.1228 &25.81& 0.0899&\textbf{7.23}&0.0481\\
% \hline
\bottomrule
\end{tabular}
\end{center}
\label{tab:pad_kitti_compare}
\end{table*}

\begin{table*}[!htbp]
    \caption{基于一致方法与传统方法比较结果}
    \begin{center}
    \begin{tabular}{c c c c c c c c c c c c c c c c c c}
    \toprule
    % \hline
    \multirow{4}*{Seq} & \multicolumn{2}{c}{LIBVISO2} &  \multicolumn{2}{c}{ORB-SLAM2}& \multicolumn{2}{c}{\multirow{2}*{Our Method }}\\
                       &  \multicolumn{2}{c}{(from \cite{Song2015MoncularScale})}&\multicolumn{2}{c}{(from \cite{raul2015orb})}  &\\
    %  % \hline%\hline
    \cline{2-3}  \cline{4-5}  \cline{6-7} 
        & Trans & Rot  & Trans & Rot  & Trans & Rot\\ 
    & (\%) & (deg/m)  & (\%) & (deg/m)  & (\%) & (deg/m)& \\
    \midrule
        09&4.04& 0.0143&15.30&0.0026 &7.77&0.0476 \\
        10&25.20 &0.0388&3.68&0.0048 & 6.68&0.0485 \\
    \midrule
    % \textbf{Avg.} & \textbf{84.0}\\
    Avg &14.62 & 0.0266& 9.49&0.0037 &\textbf{7.23}&0.0481\\
    % \hline
    \bottomrule
    \end{tabular}
    \end{center}
    \label{tab:pad_kitti_compare_ge}
    \end{table*}


本节将本文提出的算法在KITTI数据集上进行评测,并与其它基于学习的算法以及传统算法进行比较。模型
在序列00-08上进行训练,在09-10上测试。由于单目尺度在泛化性上存在问题,我们将网络所
得到的运动通过乘以固定的尺度系数来与真实轨迹进行对齐。所比较的方法中,因为SfM-Learner\cite{zhou2017unsupervised}
和GeoNet \cite{yin2018geonet}不具备全局尺度一致性,其尺度对齐是通过每5帧乘尺度系数进行对齐。我们将比较结果记录于表\ref{tab:pad_kitti_compare}。
可以看出本章算法优于其它基于学习的算法。另外,从表\ref{tab:pad_kitti_compare_ge},本章算法也优于LIBVISO2和ORB-SLAM2等传统算法的单目版本。

\subsection{特征层可视化}
\label{sec:pad_es_visu}

\begin{figure*}[h]
  \centering
  \mysubfigure[KITTI 10-40]{0.19}{
  \includegraphics[width=\textwidth]{padvo/000040.png}}{\label{fig:img_40}}
  \vspace*{2mm}
  \mysubfigure[]{0.19}{
  \includegraphics[width=\textwidth]{padvo/f1_0040_7.png}}{\label{fig:40_7}}
  \vspace*{2mm}
  \mysubfigure[]{0.19}{
  \includegraphics[width=\textwidth]{padvo/f1_0040_2.png}}{\label{fig:40_2}}
  \vspace*{2mm}
  \mysubfigure[]{0.19}{
  \includegraphics[width=\textwidth]{padvo/f1_0040_5.png}}{\label{fig:40_5}}
  \vspace*{2mm}
  \mysubfigure[]{0.19}{
  \includegraphics[width=\textwidth]{padvo/f1_0040_9.png}}{\label{fig:40_9}}
  \vspace*{2mm}
  \mysubfigure[KITTI 10-119]{0.19}{
  \includegraphics[width=\textwidth]{padvo/000119.png}}{\label{fig:119}}
  \vspace*{2mm}
  \mysubfigure[]{0.19}{
  \includegraphics[width=\textwidth]{padvo/f1_0119_7.png}}{\label{fig:119_7}}
  \vspace*{2mm}
  \mysubfigure[]{0.19}{
  \includegraphics[width=\textwidth]{padvo/f1_0119_2.png}}{\label{fig:119_2}}
  \vspace*{2mm}
  \mysubfigure[]{0.19}{
  \includegraphics[width=\textwidth]{padvo/f1_0119_5.png}}{\label{fig:119_5}}
  \vspace*{2mm}
  \mysubfigure[]{0.19}{
  \includegraphics[width=\textwidth]{padvo/f1_0119_9.png}}{\label{fig:119_9}}
  \vspace*{2mm}
  \mysubfigure[KITTI 10-200]{0.19}{
  \includegraphics[width=\textwidth]{padvo/000200.png}}{\label{fig:200}}
  \vspace*{2mm}
  \mysubfigure[]{0.19}{
  \includegraphics[width=\textwidth]{padvo/f1_0200_7.png}}{\label{fig:200_7}}
  \vspace*{2mm}
  \mysubfigure[]{0.19}{
  \includegraphics[width=\textwidth]{padvo/f1_0200_2.png}}{\label{fig:200_2}}
  \vspace*{2mm}
  \mysubfigure[]{0.19}{
  \includegraphics[width=\textwidth]{padvo/f1_0200_5.png}}{\label{fig:200_5}}
  \vspace*{2mm}
  \mysubfigure[]{0.19}{
  \includegraphics[width=\textwidth]{padvo/f1_0200_9.png}}{\label{fig:200_9}}
  \vspace*{2mm}
   \mysubfigure[KITTI 10-437]{0.19}{
  \includegraphics[width=\textwidth]{padvo/000437.png}}{\label{fig:437}}
  \mysubfigure[]{0.19}{
  \includegraphics[width=\textwidth]{padvo/f1_0437_7.png}}{\label{fig:437_7}}
  \mysubfigure[]{0.19}{
  \includegraphics[width=\textwidth]{padvo/f1_0437_2.png}}{\label{fig:437_2}}
  \mysubfigure[]{0.19}{
  \includegraphics[width=\textwidth]{padvo/f1_0437_5.png}}{\label{fig:437_5}}
  \mysubfigure[]{0.19}{
  \includegraphics[width=\textwidth]{padvo/f1_0437_9.png}}{\label{fig:437_9}}
  \caption{网络特征学习效果图}
  \label{fig:feature_1}
\end{figure*}
为了可以更好的理解神经网络学到了什么,本节将第一个卷积层之后的特征图可视化于图
\ref{fig:feature_1}。第一个卷积层之后有16个特征层,本节可视化了其中四个。 图中,红色代表激活函数输出更高,表示网络更注重观察这些区域,绿色与之相反。
如图所示,各个特征层在关注于不同的图像区域,可视化的四个特征层分别关注于:路面、路的边界、
楼宇树木和天际线。

\section{本章小结}
\label{sec:conclusion}
本章提出了一种基于区域一致性的深度单目视觉里程计网格构架设计方法。依据单目运动估计映射的全局冗余性和区域一致性,以隐式地
增加训练数据集的数据量并减少训练集与测试集之间的差异为目的,本章首次提出使用卷积神经网络学习从图像子区域到相机运动之间的映射模型。为了适应不同图像区域的特征差异,本章同时对区域运动估计的可靠性进行建模,并以其为权重计算最终的相机的运动。
我们在KITTI公开数据上对本章提出的算法进行了验证:通过消冗
实验证明了本文的方法可以增强算法的准确性以及跨数据集测试的泛化性;同时,在与其它方法比较中可以发现,本章所提出的方法优于
同时期其它基于学习的运动估计方法。
%\chapter{Geometry-based End to End visual odometry}
Visual odoemtry is an important method for localization and ego-motion estimation for mobile robots. Learning-based solutions for visual odometry become more and more popular as it reduce the efforts for sensor calibration and parameter tuning. However, the current performance of learning-based method is not good. We argue that one reason should be that most current learning-based methods utilize multiple  convolution layers to model the mapping from image-pair to ego-motion, however convolution layers are designed for modeling feature extraction and description, not for modeling geometry calculation. We proposed to model the ego-motion mapping with a new structure which contains new kind of layer in order to improve the performance. 

The structure of proposed ego-motion estimation network is shown in Fig. \ref{}. Taking the geometry-based method in to consideration, we can know that, the solution for visual odometry should contains three parts: feature extraction, feature matching and ego-motion calculation. We imitate that structure and keep the three parts in our designed structure. Convolution layers are used for feature extraction; we proposed a matching layer to imitate the feature matching process; ego-motion calculation model is realized by fully connected layers.

The main contribution of this work are: 
\begin{itemize}

\item We proposed a new neural network structure for ego-motion estimation. In this new structure, we utilize a novel matching layer. 

\item We test our proposed method in Kitti dataset, TUM dataset and our collected dataset, and found that( the performance overcome the state of the art.)

\item We test the training model on new unknown datasets  and found that our method is with good generalization ability. 
\end{itemize}

\section{Geometry-based Ego-motion Estimation Model}
Imitating geometry-based ego-motion estimation method, we separate the ego-motion estimation model into three parts: feature extraction, feature matching and ego-motion estimation. They are trained end to end. We descried our proposed method in case which it is trained supervised, but the proposed method can also be trained unsupervised. 

\subsection{Model Structure}
We utilize convolution layers for feature extraction.
\begin{equation}
    \mathbf{F} =M_f \left( \mathbf{I}\right)
\end{equation}
%\chapter{基于运动模型的路面车辆深度视觉里程计}
\label{ch:datavo}
在前两章中,我们分别从损失函数(第\ref{ch:homovo}章)和网络架构(第\ref{ch:padvo}章)两个角度对基于学习的视觉里程计方法进行了改进。而在深度学习算法中,存在关键的三个因素:损失函数、网络架构和训练数据。本章将从路面车辆的运动约束和数据分布的角度对基于学习的视觉里程计模型进行改进。

大部分基于深度学习的视觉里程计算法尝试学习从连续图像对到相机运动的映射模型,其中相机运动一般由三自由度的平移运动和三自由度旋转运动组成。
然而,我们发现,对于路面轮式车辆系统来说,由于运动模式受其自身机械结构的限制和动力机制的约束,其在三维空间的运动并不具备完整的6自由度,大部分运动局限在$Z$轴方向的平移运动和$Y$轴方向的旋转运动(如图\ref{fig:car_simplify}),受限的运动模式进而使网络模型的训练集数据在各个运动维度上分布不均。
\begin{figure}[h]
    \centering
    \includegraphics[width=0.95\textwidth]{datavo/car_simplify.pdf}
    \caption{车辆运动简化示意图}
    \label{fig:car_simplify}
\end{figure}
本文认为这是当前基于学习的视觉里程计问题精度较低的原因之一,因为所有基于监督学习的问题都依赖大量的标签数据,如分类问题的Imagenet数据集\cite{deng2009imagenet}和语义分割问题的CityScapes数据集\cite{Cordts2016Cityscapes}。而对于视觉里程计问题,主流的数据集KITTI\cite{geiger2012kitti}在数据量和数据多样性上都存在局限。本章工作基于车辆的运动模型,仅对车辆的主体运动维度进行建模,而忽略运动较小的自由度,并探索运动维度的聚焦和解耦对基于学习的视觉里程计问题的影响。

在基于几何计算的传统视觉里程计问题中,路面车辆的运动模型已经被广泛使用。由于车辆在$Y$轴方向的平移运动幅度极小,固定在车辆上的相机高度
不易发生变化,包括本文第二章在内的很多工作\cite{Song2015MoncularScale,Lee2015MoncularScale,zhou2016reliable,7898840}以此作为绝对尺度参考恢复单目视觉运动估计的绝对尺度。
Scaramuzza等人\cite{scaramuzza2009real}根据车辆Ackermann运动模型,简化车辆运动估计,提出基于单特征点的随机抽样一致进行运动估计,提
高算法实时性。Choi等人\cite{choi2015simplified}在基于路面车辆模型的基础之上,考虑车辆震荡,放松了严格平面运动的约束,使算法更加鲁棒。
此外Scaramuzza等人\cite{4625958}还提出了基于单应性的全景相机视觉里程计。


本章首次将车辆的运动模型引入到基于学习的单目视觉里程计算法中,整体结构如图\ref{fig:datavo_system_structure}。在具体实现上,首先定量评价忽略受限制运动维度所造成的车辆运动轨迹的偏移;然后根据车辆的
旋转运动模型对车辆的旋转运动和平移运动解耦,以减弱在忽略受限制维度运动时的轨迹偏移;
同时本章设计并构建了针对车辆主要运动进行建模的轻量化卷积神经网络模型。基于对上述工作,本章做出如下贡献:
\begin{figure}[h]
    \centering
    \includegraphics[width=0.95\textwidth]{datavo/system_structure.pdf}
    \caption{路面车辆视觉里程计映射学习系统架构图}
    \label{fig:datavo_system_structure}
\end{figure}
\begin{enumerate}
    \item 本章通过实验定量地分析了运动聚焦所造成的轨迹偏移,并证明了运动聚焦的可行性;
    \item 本章根据车辆的运动模型,分析了车辆沿$X$轴的平移运动与绕$Z$轴旋转运动之间的映射关系,并通过运动解耦减低了运动聚焦时的运动偏移;
    \item 本章实验证明了运动聚焦和运动解耦可以提高基于学习的单目视觉里程计的性能,包括减少训练时间、提升训练精度;
    \item 本章构建了一个十分轻量化的运动估计网络,该网络在训练时仅需占用2G的图形处理单元(GPU)显存并可以快速收敛,并可以在取得与其它算法可比精准度的前提下以200帧每秒的速度实时运行在CPU上,算法已开源\footnote{https://github.com/TimingSpace/DMVOGV}.
\end{enumerate}

本章结构如下:
首先在第\ref{sec:motion}节介绍算法的数学原理和实现方式;然后在第\ref{sec:datavo_experiments}节,我们在KITTI数据集\cite{geiger2012kitti}定性和定量的评价我们算法;
最后在\ref{sec:datavo_conclusion}节总结本章工作。

\section{路面车辆视觉里程计方法}
运动聚焦,即忽略车辆运动维度较小的维度,聚焦于车辆的主体运动。根据车辆模型的约束,以简化运动估计为目的,本章提出了运动聚焦算法,并通过运动解耦算法以降低运动聚焦带来的位姿偏移。
本节,我们首先在第\ref{sec:motion}节介绍运动聚焦和运动解耦方法;然后在第\ref{sec:model}节介绍模型架构以及训练和测试方法。
%\subsection{Data Processing}

\subsection{运动聚焦与解耦}
\label{sec:motion}

\begin{figure}[ht]
    \centering
    \begin{subfigure}[b]{0.48\textwidth}
        \includegraphics[width=\textwidth]{datavo/motion_dis.png}
        \caption{运动分布}
        \label{fig:motion_dis} 
    \end{subfigure}
    \begin{subfigure}[b]{0.48\textwidth}
        \includegraphics[width=\textwidth]{datavo/rotation_corr.pdf}
        \caption{运动耦合}
        \label{fig:rotation_corr}
    \end{subfigure}
    \caption{路面车辆运动模态分析}

\end{figure}
\subsubsection{运动聚焦}
根据车辆的机械结构和动力模型的限制,车辆的主要运动为沿$Z$轴的平移运动(前进)和绕$Y$轴的旋转运动(转向)。KITTI视觉里程计数据集中车辆在各个维度上的运动方差给这一论点提供了数据支撑,如图\ref{fig:motion_dis}所示(此图中仅可视化了KITTI数据集00序列的运动,为了说明其代表性,附录\ref{ch:app_motion_dis}中可视化了更多的数据)。运动表征选择使用标准的相机坐标系,其为以相机光心为原点的右手坐标系,向前为$Z$轴方向,向右和向下分别$X$轴方向和$Y$轴方向。车辆绕$X$轴、$Y$轴和$Z$轴的旋转运动的欧拉角分别表示为 $\psi$, $\varphi$和 $\theta$。从图\ref{fig:motion_dis}中可看出,车辆的平移运动确实主要集中在$Z$轴方向,而车辆的旋转运动主要集中在$Y$轴上,所以本章选择仅建模输入图像到车辆$Z$轴平移运动和$Y$轴旋转运动的映射,这个过程称之为为运动聚焦。
\begin{figure}[ht]
    \centering
    \begin{subfigure}[b]{0.48\textwidth}
        \centering
        \includegraphics[width=\textwidth]{datavo/r_t_2d_hist.pdf}
        \caption{二维联合分布}
        \label{fig:rt_2d} 
    \end{subfigure}
    \begin{subfigure}[b]{0.48\textwidth}
        \centering
        \includegraphics[width=\textwidth]{datavo/r_t_1d_hist.pdf}
        \caption{比例分布}
        \label{fig:rt_1d}
    \end{subfigure}
    \caption{$X$轴平移运动与$Y$轴旋转运动的相关性}
    \label{fig:rotation_corr_analysis}
\end{figure}

\subsubsection{运动解耦}
然而,我们发现车辆依然存在着一定幅度(约10\%)的$X$轴平移运动(如图\ref{fig:motion_dis})。但由于动力约束,路面车辆系统从原理上是无法沿$X$轴有大幅度的运动的,那么这10\%的轴平移运动来自哪里呢?
通过仔细分析无人车的旋转模式可知$X$轴的平移运动产生的原因为车辆的运动表征方式:

\begin{equation}
    \begin{pmatrix} \mathbf{R} & \mathbf{t}\\ 0 & 1  \end{pmatrix} = \begin{pmatrix} \mathbf{I}& \mathbf{t}\\ 0 & 1  \end{pmatrix}\begin{pmatrix} \mathbf{R}& \mathbf{0}\\ 0 & 1  \end{pmatrix}
    \label{eq:ftlr}
\end{equation}
其中
$\mathbf{I}$ 为3x3的单位矩阵。 
在这种表征方式中,机器人首先进行相对于初始基坐标系平移运动$\mathbf{t}$,然后在运动之后的基坐标系上进行旋转运动$\mathbf{R}$(基坐标系指机器人自身坐标系)\cite{sun1995robot},那么如果机器人同时进行了旋转和平移运动,继而车辆的参考坐标系会因为旋转运动发生变化,本来仅沿
$Z$轴的平移运动也会由于旋转的存在映射成$Z$轴平移分量与$X$轴平移分量,如图\ref{fig:rotation_model}所示。参考坐标系变化带来的平移偏角$\alpha$定义为:
\begin{equation}
    \alpha = \arctan\left(\frac{x}{z}\right)
\end{equation}
其中$x$和$z$分别表示$X$轴和$Z$轴的平移运动。
从图\ref{fig:rotation_corr}中可以看出,$X$轴的平移运动与$Y$轴的旋转运动相关性
很强,这一点恰好证明$X$轴运动源自于旋转后运动映射。由于图\ref{fig:rotation_corr}只能评价局部的运动相关性,为了得到更具代表性的结果,我们通过直方图可视化旋转角度$\theta$和平移偏角$\alpha$的全局关系。 图\ref{fig:rt_1d}的一维直方图可视化了$\alpha/\theta$的分布曲线;图\ref{fig:rt_2d}中的二维直方图可视化了$\alpha$和$\theta$的联合概率分布。从两个直方图中都可以看出,车辆旋转角$\theta$和平移偏角$\alpha$有着很强的相关性。

那么如何通过改变运动表征方式来解除$X$轴平移运动与$Y$轴旋转运动之间的耦合呢?一种朴素的方法为调换旋转
运动和平移运动的顺序,车辆先进行旋转运动,然后在旋转之后的坐标系下进行平移运动,这样车辆的平移运动就不会被映射到$X$轴上,这种运动表征方式可以公式化为:
\begin{equation}
    \begin{pmatrix} \mathbf{R} & \mathbf{t}\\ 0 & 1  \end{pmatrix} = \begin{pmatrix} \mathbf{R'}& \mathbf{0}\\ 0 & 1  \end{pmatrix}\begin{pmatrix} \mathbf{I}& \mathbf{t'}\\ 0 & 1  \end{pmatrix}
    \label{eq:frlt}
\end{equation}
可以推出公式中
$\mathbf{R'} = \mathbf{R}$, $\mathbf{t'} = \mathbf{R}^{-1}\mathbf{t}$,相当于把由于旋转重映射的平移运动再反映射回来。
\begin{figure}[ht]
    \centering
    \begin{subfigure}[b]{0.48\textwidth}
        \includegraphics[width=\textwidth]{datavo/vehicle_rotation_1-crop.pdf}
        \caption{车辆旋转示意图}
        \label{fig:vehicel_rotation_model} 
    \end{subfigure}
    \begin{subfigure}[b]{0.48\textwidth}
        \includegraphics[width=\textwidth]{datavo/vehicle_rotation_2-crop.pdf}
        \caption{简化版示意图}
        \label{fig:vehicel_rotation_model_s} 
    \end{subfigure}
    \caption{车辆旋转模型示意图}
    \label{fig:rotation_model}
\end{figure}
然而,如图\ref{fig:vehicel_rotation_model}所示,由旋转运动引发的车辆平移偏角$\alpha$其实并不等于车辆的旋转角$\theta$。重映射依赖于平移偏角$\alpha$与旋转角$\theta$ 之间的关系,我们定义其间关系为$\alpha = f(\theta)$。在已知这个关系的基础之上,可以
使用车辆的前进距离和车辆的旋转角这两个运动参数来表征车辆的平面运动,平面运动中的平移运动为
\begin{equation}
    (x,z) = z(sin(f(\theta)),cos(f(\theta)))
    \label{eq:car_angle}
\end{equation}

图\ref{fig:vehicel_rotation_model}中,A点表示车辆后轴的中心,B点表示相机的安装位置,设定A点和B点之间的距离为$l$,称之为相机偏距。
视觉里程计之间估计的运动为B点和B'之间的距离,记为$z'$。我们将图\ref{fig:vehicel_rotation_model}简化为图\ref{fig:vehicel_rotation_model_s}。

根据车辆Ackermann运动模型\cite{siegwart2011introduction},$OA \bot AB$ 且 $OA' \bot A'B'$,于是可以得到 $\phi = 0.5 \beta = 0.5 \theta$。
在三角形$CBB'$中,根据正弦定理:
\begin{equation}
    \frac{\sin(\gamma)}{\sin(\beta)}  = \frac{l-\frac{\hat{z}}{2} / \cos(\frac{\theta}{2})}{z'} 
 \end{equation}
考虑到$\theta$的绝对值很小且接近0, 可做如下近似 $\cos(\frac{\theta}{2}) \approx 1$ 且 $ \frac{\gamma}{\beta} \approx  \frac{\sin(\gamma)}{\sin(\beta)} $,于是
\begin{equation}
    \frac{\gamma}{\beta}  \approx \frac{l-\frac{\hat{z}}{2}}{z'} 
\end{equation}
记$d = |AC| \approx 0.5|AA'| =0.5\hat{z}$, 在三角形$CBB'$中根据余弦定理:
\begin{equation}
    z'^2 = (l+d)^2 + (l-d)^2- 2(l+d)(l-d)\cos(\beta) = 2l^2+2d^2 - 2(l^2-d^2)\cos(\beta)\approx 4d^2
\end{equation}
所以 $z'\approx \hat{z}$.
最后可得到了平移偏角$\alpha$和旋转角$\theta$之间的关系
\begin{equation}
    \alpha = \beta + \gamma \approx (\frac{l}{z'}+0.5)\beta =(\frac{l}{z'}+0.5)\theta
    \label{eq:r_t_ratio}
\end{equation}
根据平移偏角$\alpha$构建旋转矩阵$R_\alpha$,
\begin{equation}
    \mathbf{R}_\alpha = \begin{pmatrix}
        \cos(\alpha)& 0 & \sin(\alpha)\\ 
        0 & 1 & 0\\ 
        -\sin(\alpha)& 0 & \cos(\alpha)\\ 
    \end{pmatrix} 
    \label{eq:r_alpha}
\end{equation}
然后重映射平移向量 $\mathbf{t}'$
\begin{equation}
    \mathbf{t}' = \mathbf{R}_\alpha^{-1}\mathbf{t}
    \label{eq:decouple_z}
\end{equation}
车辆的运动距离$Z$为平移向量$\mathbf{t}'$中的第三个分量。至此车辆的平面运动可以被车辆旋转角$\theta$和前进距离$z’$两个参数近似表示,
本章简化运动估计的目标,仅学习这两个运动参数,网络模型将会在下一节介绍,效果会在\ref{sec:ego_improvement}进行评测。

\subsection{网络模型与训练}
\label{sec:model}
\label{sec:approach}
\begin{figure}[t]
    \centering
    \includegraphics[width=0.95\textwidth]{datavo/network_structure_2-crop.pdf}
    \caption{轻量化网络架构结构图}
    \label{fig:nerwork_structure}
\end{figure}

本章构建了一个轻量化的网络来学习路面车辆的主要运动,网络架构如图\ref{fig:nerwork_structure}所示。
网络主要由卷积层构成,除最后一个卷积层以外,每个卷积后面都附加一个批归一化层(Batch Normalization) \cite{wu2018group} 和修正线性单元层(ReLU)。
同Zhou等人 \cite{zhou2017unsupervised}的架构一样,网络的最后一层没有使用全连接层,而是使用全局均值池化层\cite{lin2013network},
用以降低参数量,增强泛化性。此外,由于车辆运动引发的图像光流大部分为水平方向,尤其是在车辆旋转的情况下,如图\ref{fig:optical_flow}所示。
所以网络没有使用正方形的卷积核,而是通过使用宽大于高的卷积核来扩大水平方向的感受野。同时,本章使用了膨胀卷积(Dilated Convolution) \cite{yu2015multi},进一步在参数量相同的情况下扩大感受野。

\begin{figure}[ht]
    \centering
    \begin{subfigure}[b]{0.48\textwidth}
        \includegraphics[width=\textwidth]{datavo/flow_61.png}
        \caption{车辆前行}
        \label{fig:optical_flow_f} 
    \end{subfigure}
    \vspace*{2mm}
    \begin{subfigure}[b]{0.48\textwidth}
        \includegraphics[width=\textwidth]{datavo/flow_52.png}
        \caption{车辆后退}
        \label{fig:optical_flow_b} 
    \end{subfigure}
    \vspace*{2mm}
    \begin{subfigure}[b]{0.48\textwidth}
        \includegraphics[width=\textwidth]{datavo/flow_196.png}
        \caption{车辆向左侧旋转}
        \label{fig:optical_flow_l} 
    \end{subfigure}
    \begin{subfigure}[b]{0.48\textwidth}
        \includegraphics[width=\textwidth]{datavo/flow_96.png}
        \caption{车辆向右侧旋转}
        \label{fig:optical_flow_r} 
    \end{subfigure}
    \caption{路面车辆运动时的光流分析}
    \label{fig:optical_flow}
\end{figure}
模型的输入为由灰度图像叠加而成图像对,本章不仅使用相邻帧图像组合成的图像对(图像间隔为0)对作为输入,而是随机生成闭区间[-4,4]之间的一个整数,然后使用这个数作为图像对中两个图像之间的间隔,当做一个数据增加的手段。
模型的输出为与图像对对应的相机运动,表示为$Y$轴旋转运动$\theta $和前进运动$z'$。其中$Y$轴旋转运动$\theta$为旋转矩阵计算得来的欧拉角所对应的$Y$轴分量;
前进运动$z'$可使用公式\eqref{eq:decouple_z}计算。本章使用L2距离作为损失函数:
\begin{equation}
    L_2 = \|\underline{\theta} -\theta_p\|_2 +\|\underline{z} -z_p \|_2
\end{equation}
其中$\theta_p$和$z_p$为模型估计出的运动,$\underline{\theta}$和$\underline{z}$运动的真值。
本章使用ADAM优化器\cite{kingma2014adam}最小化损失函数获取模型参数,初始学习率设置为0.001,在第50个周期之后,学习率线性衰减,至100个周期衰减为0.0006。

模型训练成功之后,模型输入连续图像后可以得到与之对应的机器人旋转角 $\theta$前进距离$Z$。
首先根据公式\eqref{eq:r_t_ratio}计算平移偏角$\alpha$,并假设其它维度旋转均为0,然后构建旋转矩阵
$\mathbf{R}_\theta$和$\mathbf{R}_\alpha$,车辆的平移向量计算为$\mathbf{t}_\alpha = \mathbf{R}_\alpha (0,0,z)^T$。
运动矩阵可表示为:
\begin{equation}
    \mathbf{T}_i =\begin{pmatrix} \mathbf{R}_\theta & \mathbf{t}_\alpha\\ 0 & 1  \end{pmatrix} 
    \label{eq:rt_final}
\end{equation}
车辆的位姿可以通过运动矩阵的累积获取:
\begin{equation}
    \mathbf{P}_i = \mathbf{P}_{i-1}\mathbf{T}_i
    \label{eq:pose}
\end{equation}

\section{路面车辆模型实验}
\label{sec:datavo_experiments}
我们在公开数据集KITTI的视觉里程计数据\cite{geiger2012kitti}上对本章提出的算法进行评测和分析,使用其提供评价标准定量测量相对位姿误差(RPE),包括相对平移误差和相对姿态误差\cite{geiger2012kitti}。

本章算法使用Python程序语言实现,基于深度学习框架PyTorch进行网络模型搭建和模型训练,目前本章算法已经开源\footnote{https://github.com/TimingSpace/DMVOGV}。本章算法在一台具备16GB内存、因特尔Core i7(Intel i7-7700 CPU @ 2.80GHz )、英伟达GPU(GeForce GTX 1060)的笔记本电脑上进行测试。测试系统配备CUDA10.0,Python 3.6.9的Ubuntu操作系统。
由于所提出模型的轻量化,当训练批大小(Batch Size)为30时,模型训练仅需要2G的显存;在预测时,本章算法可以在CPU上达到200帧每秒的实时效果。

\subsection{实验结果}
我们首先评价运动聚焦引发的位置偏移;然后评价运动解耦对位置偏移的抑制效果;之后评价运动聚焦和解耦以及本章的轻量化网络架构带来的算法性能提升;最后与其它算法进行比较。

\subsubsection{运动聚焦造成的运动偏移的定量评测}
\label{sec:info_loss}
由于车辆运动受其机械结构和动力模型的约束,车辆主要运动集中于$Z$轴上的平移运动和$Y$轴上的旋转运动。此实验将定量分析在去除其它部分和全部维度的次要运动后,车辆位姿的偏移。
在具体实现上,首先忽略次要运动,然后重构机器人轨迹,并使用相对位置误差评价重构后的轨迹与原始估计之间的差异。在KITTI序列00至序列10上的相对位置误差的平均值记录于表\ref{tab:info_loss_1},并可视化于图\ref{fig:info_loss}中。
\begin{table}[h]
    \caption{Average RPE When Only Keeping Part of Vehicle Motion}
    \label{tab:info_loss_1}
    \begin{center}
    \begin{tabular}{c c c c c}
    \toprule
    % \hline
    \multirow{2}*{R / t} &{z}&{c}{xz}&{yz}&{zyz}\\
    & RPE(\%) /NID& RPE(\%) /NID& RPE(\%) /NID& RPE(\%) /NID\\
    %  % \hline%\hline
    \midrule
     y   &2.20  /4 & 2.06 /3 & 2.45 /3 & 2.34 /2 \\
     xy  &1.92  /3 & 1.77 /2 & 1.76 /2 & 1.56 /1 \\
     zy  &2.05  /3 & 1.91 /2 & 1.47 /2 & 1.27 /1 \\
     xyz &1.92  /2 & 1.81 /1 & 0.49 /1 & 0    /0   \\
    % \hline
    \bottomrule
    \end{tabular}
    \end{center}
 \end{table}
 \iffalse
 \begin{table}[t]
    \caption{Information Loss by Focusing only on translation along $z$ and roation about $y$}
    \label{tab:info_loss_2}
    \begin{center}
    \begin{tabular}{c c c c c c c c c c c c }
    \toprule
    % \hline
    seq & 00 & 01 & 02 & 03 & 04 & 05 & 06 & 07 & 08 & 09 & 10\\
    %  % \hline%\hline
    \midrule
     RPE(\%) &1.31 & 1.93 & 3.05 & 2.80& 2.22&1.30&1.34&1.19&1.44&3.79&3.85 \\
     ATE(m) &1.31 & 1.93 & 3.05 & 2.80& 2.22&1.30&1.34&1.19&1.44&3.79&3.85 \\
    % \hline
    \bottomrule
    \end{tabular}
    \end{center}
 \end{table}
\fi
\begin{figure}[ht]
    \centering
    \includegraphics[width=0.95\textwidth]{datavo/info_loss.pdf}
    \caption{运动聚焦位置偏移可视化图}
    \label{fig:info_loss}
\end{figure}
表\ref{tab:info_loss_1}中,每一行保留着相同的旋转运动维度,每一列保留着相同的平移运动维度。 当仅保留$Y$轴旋转运动和$Z$轴平移运动(第一行第一列)时,平均位置误差为2.20\%。部分重构的轨迹可视化于图\ref{fig:path_recon},可见运动简化后的轨迹在水平面上依然与真实轨迹较为接近,但在竖直方向会因不同序列的高度变化产生不同的偏差(附录B中可视化了更多的重构轨迹)。

\begin{figure}[ht]
    \centering
    \begin{subfigure}[b]{0.48\textwidth}
    \includegraphics[width=\textwidth]{datavo/path_recon_07.pdf}
    \caption{KITTI 07}
    \label{fig:recon_07}
    \end{subfigure}
    \begin{subfigure}[b]{0.48\textwidth}
        \includegraphics[width=\textwidth]{datavo/path_recon_10.pdf}
        \caption{KITTI 10}
        \label{fig:recon_10}
    \end{subfigure}
    \caption{运动聚焦后的轨迹重构图} 
    \label{fig:path_recon}
\end{figure}
我们将平均位姿误差(损失)与忽略维度的数量(增益)求商,如\ref{fig:info_loss}中的黄色线所示。这里比例系数可以作为一个相对平均损失。可见在仅保留$Z$轴平移运动和$Y$轴旋转运动时,相对损失比较小。


\subsubsection{运动解耦评测}
\label{sec:info_decouple}
本节将评价运动解耦对位姿偏移的矫正效果。根据公式\eqref{eq:r_t_ratio},车辆的平移偏角$\alpha$和车辆的旋转角$\theta$为线性关系。然而,其比例系数依赖于动态的车辆前进距离$z'$,并不是固定不变的。
首先研究使用静态的比例系数(即将比例系数设为固定值)的实验效果。我们使用不同的比例系数计算车辆偏移角,然后重映射车辆运动,得到的车辆评价位置误差可视化于图\ref{fig:static_decouple}。从图中可以看出,当比例系数为1.7时,评价误差最小,说明旋转时车辆的运动距离约为$\frac{1.7-0.5}{l}$米。为了更好的理解图中不同比例系数的物理意义,图中使用了不同颜色标出几个特殊的比例系数。红色图表示比例系数为0,此时平移运动不做任何映射,为原始的运动表示;比例系数为1时,认为平移偏角和旋转角度相同,根据公式\eqref{eq:frlt}的朴素映射解耦;黑色表示平移偏角为旋转角度的一半,这种情况仅在相机安装在后轴中心时才成立,但相机距离后轴中心相比于车辆前进距离较小时,此系数也可近似成立。

\begin{figure}[ht]
    \centering
    \begin{subfigure}[b]{0.48\textwidth}
        \centering
        \includegraphics[width=\textwidth]{datavo/r_t_ratio.pdf}
        \caption{静态解耦}
        \label{fig:static_decouple}
    \end{subfigure}
    \begin{subfigure}[b]{0.48\textwidth}
        \centering
        \includegraphics[width=\textwidth]{datavo/r_t_ratio_2.pdf}
        \caption{动态解耦}
        \label{fig:dynamic_decouple}
    \end{subfigure}
    \caption{运动解耦效果可视化}
    \label{fig:r_t_ratio}
\end{figure}

固定的比例系数无视了前进距离对平移偏角的影响,于是本节定量分析动态比例系数的位姿偏移。使用不同的相机偏距$l$,根据公式\eqref{eq:r_t_ratio},计算比例系数,然后求平均重构轨迹的相对位置误差,可视化于图 \ref{fig:dynamic_decouple},当相机偏距为0.4米时,重构误差最小。图中黑色为相机偏距为0时的重构误差,此时 $\alpha = 0.5 \theta$,和图\ref{fig:static_decouple}中的黑色物理意义相同,数值相等。

本节将动态映射和静态映射的效果进行了比较,记录于图\ref{fig:decouple}中。可以发现,两种运动解耦方式都降低了机器人轨迹的相对位置误差,动态解耦相比于静态解耦效果更好,更接近同时保留$X$轴平移和$Z$轴平移时的精度。

\begin{figure}[ht]
    \centering
    \includegraphics[width=0.95\textwidth]{datavo/decouple-crop.pdf}
    \caption{动态解耦和静态解耦效果对比图} 
    \label{fig:decouple}
\end{figure}

\subsubsection{运动简化的有效性验证}

\label{sec:ego_improvement}
\begin{table}[ht]
    \caption{The improvement of motion focusing}
    \label{tab:info_improve}
    \begin{center}
    \begin{tabular}{c c c c c c }
    \toprule
    % \hline
    \multirow{3}*{Train} & \multirow{3}*{Test} &\multicolumn{2}{c}{Learn All Motion} & \multicolumn{2}{c}{Learn $R_y, t_z$}\\
    & &Trans & Rot & Trans & Rot\\
    & & (\%) & (deg/m)  & (\%) & (deg/m)\\
    %  % \hline%\hline
    \midrule
     00 & 02 04 06 08 10 &26.8 & 0.137 & 23.9 & 0.110 \\
     00 02 & 04 06 08 10 &18.3 & 0.095 & 16.7 & 0.070 \\
     00 02 04 & 06 08 10 &17.6 & 0.091 & 16.9 & 0.076 \\
     00 02 04 06 & 08 10 &15.3 & 0.082 & 13.2 & 0.065   \\
    % \hline
    \bottomrule
    \end{tabular}
    \end{center}
 \end{table}

本节设计对比实验以证明运动聚焦和解耦的有效性。使用相同的训练数据训练如下两种模型:
1) 运动简化模型(Motion Focusing Model,MFM)仅学习车辆的前进运动和旋转运动;  2) 全运动模型 (All Motion Model,AMM), 学习车辆六个自由度
的全部运动。本节在多个训练和测试数据序列的组合上进行多次实验以减低随机性的影响。在实验过程中,我们记录了网络训练时的损失函数变化曲线和测试时的相对平移误差。
如图\ref{fig:training_loss}所示,运动简化模型收敛速度相比于全运动模型有着大幅提高。
\begin{figure}[ht]
    \centering
    \begin{subfigure}[c]{0.48\textwidth}
        \includegraphics[width=\textwidth]{datavo/training_loss_0.pdf}
        \caption{00}
        \label{fig:tl_0}
    \end{subfigure}
    \vspace*{2mm}
    \begin{subfigure}[c]{0.48\textwidth}
        \includegraphics[width=\textwidth]{datavo/training_loss_0-2.pdf} 
        \caption{00-02}
        \label{fig:tl_02}
    \end{subfigure}
    \vspace*{2mm}
    \begin{subfigure}[c]{0.48\textwidth}
        \includegraphics[width=\textwidth]{datavo/training_loss_0-4.pdf} 
        \caption{00-04}
        \label{fig:tl_024}
    \end{subfigure}
    \begin{subfigure}[c]{0.48\textwidth}
        \includegraphics[width=\textwidth]{datavo/training_loss_0-6.pdf} 
        \caption{00-06}
        \label{fig:tl_0246}
    \end{subfigure}
    \caption{训练损失函数收敛速度对比图}
    {\label{fig:training_loss}}
\end{figure}

\begin{figure}[ht]
    \centering
    \includegraphics[width=0.95\textwidth]{datavo/focusing_train.png}
    \caption{运动聚焦与解耦前后效果对比图}
    \label{fig:focucing_train}
\end{figure}
不同训练模型的测试误差记录在表\ref{tab:info_improve},并可视化于图\ref{fig:focucing_train}。
可以看出,运动简化模型的测试精度相比于全运动模型在不同数据组合的情况下均有一定提高:其中相对平移误差提高约2\%,
相对姿态误差提高约每米0.2度。可见,尽管运动聚焦和运动解耦使训练目标轨迹与真实轨迹相比出现了一定的偏移,但整体来看,测试精度
仍然有所提升。另外,我们发现,无论是运动简化模型还是全运动模型,随着训练数据集数据量的不断增加,测试误差均不断降低。

\subsubsection{与其它算法比较}
\label{sec:compare}
\begin{table}[!htbp]
    \caption{与其他基于学习的视觉里程计方法比较结果}
    \begin{center}
    \begin{tabular}{c c c c c c c c c c c c c c}
    \toprule
    % \hline
    \multirow{4}*{Seq} & \multicolumn{2}{c}{Zhan et al.} &\multicolumn{2}{c}{DeepVO} & \multicolumn{2}{c}{SfM-Learner.}& \multicolumn{2}{c}{GeoNet}&  \multicolumn{2}{c}{\multirow{2}*{Our Method }}\\
                       & \multicolumn{2}{c}{(from \cite{zhan2018unsupervised})}  & \multicolumn{2}{c}{(from \cite{wang2017deepvo})}&\multicolumn{2}{c}{(from \cite{zhou2017unsupervised})} &\multicolumn{2}{c}{(from \cite{yin2018geonet})} &\\
    %  % \hline%\hline
    \cline{2-3}  \cline{4-5}  \cline{6-7} \cline{8-9} \cline{10-11} \cline{12-13}
        & Trans & Rot  & Trans & Rot  & Trans & Rot &Trans & Rot& Trans & Rot\\ 
    & (\%) & (deg/m)  & (\%) & (deg/m)  & (\%) & (deg/m)& (\%) & (deg/m) & (\%) & (deg/m) \\
    \midrule
        09&11.92&0.0360&-&-&17.84&0.0678&26.93&0.0954&9.26&0.0229 \\
        10&12.62&0.0343&8.11&0.0883&37.91&0.1778&24.69&0.0843&9.10&0.0221 \\
    \midrule
    % \textbf{Avg.} & \textbf{84.0}\\
    Avg & 12.27 & 0.0351 &8.11 &0.0883  & 28.88 &0.1228 &25.81& 0.0899& 9.18&\textbf{0.0225}\\
    % \hline
    \bottomrule
    \end{tabular}
    \end{center}
    \label{tab:data_kitti_compare}
    \end{table}
    
    \begin{table}[!htbp]
        \caption{与其他传统视觉里程计方法比较结果}
        \begin{center}
        \begin{tabular}{c c c c c c c c c c c c c c c c c c}
        \toprule
        % \hline
        \multirow{4}*{Seq} & \multicolumn{2}{c}{LIBVISO2} &  \multicolumn{2}{c}{ORBSLAM}& \multicolumn{2}{c}{\multirow{2}*{Our Method }}\\
                           &  \multicolumn{2}{c}{(from \cite{Song2015MoncularScale})}&\multicolumn{2}{c}{(from \cite{raul2015orb})}  &\\
        %  % \hline%\hline
        \cline{2-3}  \cline{4-5}  \cline{6-7} 
            & Trans & Rot  & Trans & Rot  & Trans & Rot\\ 
        & (\%) & (deg/m)  & (\%) & (deg/m)  & (\%) & (deg/m)& \\
        \midrule
            09&4.04& 0.0143&15.30&0.0026& 9.26&0.0229\\
            10&25.20 &0.0388&3.68&0.0048 &9.10&0.0221\\
        \midrule
        % \textbf{Avg.} & \textbf{84.0}\\
        Avg &14.62 & 0.0266& 9.49&0.0037 &\textbf{9.18}&0.0225\\
        % \hline
        \bottomrule
        \end{tabular}
        \end{center}
        \label{tab:data_kitti_compare_ge}
        \end{table}
    
我们将算法与其它基于学习的以及基于几何计算的方法进行定量比较。模型的训练集和测试集与其它基于卷积神经网络的方法\cite{zhan2018unsupervised,zhou2017unsupervised,yin2018geonet}一致:使用KITTI数据集00-08进行训练,在序列09和10上进行测试。
测试平均误差记录于表\ref{tab:data_kitti_compare}和表\ref{tab:data_kitti_compare_ge}中。

由于SfM-Learner\cite{zhou2017unsupervised}和GeoNet \cite{yin2018geonet}训练时并无绝对尺度信息,所以在评价其误差之前,先将其与真实轨迹进行了对齐。
ORB-SLAM\cite{raul2015orb}的单目版本和LIBVISO\cite{Geiger2011IV}的单目版本也需要与真实轨迹对齐。

从表\ref{tab:kitti_compare}中可以看出本章算法的效果优于只基于卷积神经网络的方法\cite{zhan2018unsupervised,zhou2017unsupervised,yin2018geonet} 与融合卷积神经网络和递归神经网络的DeepVO\cite{wang2017deepvo}精度不相上下。
和传统方法LibVISO2\cite{Geiger2011IV}和ORB-SLAM单目\cite{raul2015orb}相比,本章算法获取了更好的平均精度(见表\ref{tab:data_kitti_compare_ge})。

\subsection{实验结果讨论}
本节将依据实验数据对算法是否有效,算法为什么有效以及算法的局限性和解决方案等几个角度对实验结果进行讨论。
\subsubsection{算法有效性}
根据上述实验结果,可以从如下四个方面总结。
1)运动聚焦并不会引入过多的位姿偏移。运动聚焦之后,轨迹平均位置误差为2\%,其意味着在机器人运行100m后,其平均偏移量大概为2m。从可视化的重构轨迹中可以看出,这个偏移量相对很小。2)运动解耦可以进一步减少位姿偏移。运动解耦通过解耦$Y$轴旋转运行和$X$轴平移运动之间的耦合性,抑制了去除x平移后的位姿偏移,其中所提出的动态解耦算法优于静态解耦算法。在动态解耦算法中,相机到后轴的距离作为一个主要参数,本章实验中该距离为通过数据计算得到,在实际情况中也可以通过标定测量获取。3)运动聚焦和运动解耦提升了算法性能,主要体现在两个方面:运动聚焦和解耦之后的模型可以更快的收敛,普通模型需要大约60个训练周期,而优化后模型在20个训练周期之后即可收敛,收敛时间降为原来的三分之一;此外,运动聚焦和解耦之后的模型提升了运动估计的精度,在所有的对比实验中,性能都优于普通模型。4)与其它算法相比,我们取得更好的效果。在对比中可以发现,几何方法并不鲁棒,在不同测试集中效果差异较大,而本章算法取得了更好的平均效果;另外,我们的算法优于其它基于卷积神经网络的算法,但与借助递归神经网络的DeepVO基本持平,我们取得了更小的姿态误差,而DeepVO取得了更小的平移误差。

\subsubsection{算法有效性原因分析}
运动聚焦和运动解耦的有效性可以从三个角度解释:1)首先,由于地面车辆的运动受其自身机械结构和动力机制的约束,其不具备完善的三维空间内的6自由度运动模式,所以在不考虑非主要运动维度上的运动时,并不会造成过大的位置偏移,这一论点在第\ref{sec:info_loss}节得到验证,是本章方法的可行性基础;2)由于非主要维度的运动幅度非常小,导致其信噪比不高,尝试去学习这些维度的运动,会使神经网络模型容易被噪声干扰。3)由于本章算法仅学习两个主要维度,所学习目标变得相对简单,进而一个轻量级的模型就可以去学习拟合这个映射,这样数据也就相对充足,而充足的数据会提升算法的性能(如表 \ref{tab:info_improve})。

\subsubsection{算法局限性及解决方案}
当车辆的运动大部分局限在水平面上时,本章所提出的算法可以取得较好的效果,如果车辆有较多的$X$轴转动时,算法的精度会下降。如图 \ref{fig:decouple}所示,由于序列09和10存在较大比例的非水平运动,所以序列09和10的RPE误差也相对较大。为了解决这个限制,一种可行的方案为,使用其它传感器(如惯性传感器)测量车辆$X$轴的转动,作为本章算法的一个补充。

此外,本章假设视觉传感器水平朝前安装,但当相机的安装角度不是水平时,车辆的前进运动会被映射为前进分量和上下运动分量。在这种情况下,需要在初始阶段,标定相机与水平线的俯仰角$\sigma$,然后使用如下公式对车辆平移运动进行变换
\begin{equation}
    t_\sigma = \begin{pmatrix} 1 & 0 & 0 \\ 0 & \cos(\sigma) & -\sin(\sigma)\\ 0 & \sin(\sigma) & \cos(\sigma) \end{pmatrix} t
    \label{eq:pitch_correction}
\end{equation}

\section{本章小结}
\label{sec:datavo_conclusion}

本章提出了一种针对路面车辆的视觉里程计映射学习方法。根据路面车辆的特点,对单目视觉里程计问题进行简化,提出只学习车辆的主要维度运动,并通过定量分析验证了其可行性;同时研究旋转运动与平移运动之间的耦合关系,并利用此关系降低忽略次要运动所造成的运动偏移。此外,根据车辆运动时光流的分布提出使用非正方形的卷积核进行特征提取。最终,设计了一个十分轻量的视觉里程计运动估计模型,可以在与其它方法取得可比精度的情况下以每秒200帧的速高效实时运行在CPU上。





%\chapter{结论与展望}
\section{本文主要工作总结}
移动机器人在人类的生产生活中扮演着重要角色,增量式定位是移动机器人在陌生环境中实现自主移动、地图构建和环境感知需要解决的重要问题,本文主要研究了增量式定位算法中的单目视觉里程计方法,重点解决单目视觉里程计的绝对尺度计算问题和端到端映射学习问题,并完成了以下工作:
\begin{enumerate}
    \item 提出了一种基于路面几何模型的单目视觉里程计尺度计算方法。我们首次提出使用路面几何结构进行路面特征点筛选,筛选依据为路面上特征点的深度一致性和法向一致性这两个必要条件。在具体实现上,我们使用三角剖分算法将路面特征点分割成三角形区域,并对每个三角形的三个顶点验证法向一致性,同时对每条边验证深度一致性,此外我们使用图模型来构建相邻三角形的相互影响以提高筛选准确性。最后使用筛选得到的路面特征点通过随机抽样一致方法计算路面模型,并进行尺度参数计算。

    \item 提出了一种结合单目深度估计的视觉里程计尺度计算方法。首先通过构建条件随机场模型,使用帧内临近像素点的深度连续约束和已知的机器人运动情况下相邻帧图像的光度一致约束提高深度估计精度,同时使用所估计的深度与单目视觉里程计根据运动视差和三角测量所计算获取相对深度进行统计分析,并使用学生t分布建模以获取尺度系数。将深度估计和尺度计算融合,通过迭代优化进行精度的提升。
    
    \item 提出了一种基于映射同态性的单目视觉里程计网络损失函数设计方法。我们分析视觉里程计的输入图像对和输出运动之间的映射关系,通过定义图像对集合的二元运算构建图像对广义群,并依据图像对群和运动所在李群之间的同态性设计了可以更好表征运动模型的损失函数,主要包括单位元损失函数、逆元损失函数和封闭性损失函数。并通过实验证明了,新定义的损失函数可以提升运动估计的准确性。

    \item 提出了一种基于区域一致性的单目视觉里程计网络架构设计方法。我们首次提出将图像区域作为基本估计单元,首先通过每个图像子区域计算相机运动,最后再融合得到最终相机运动。这种方法可以降低运动估计时对全局图像色彩分布的依赖,在一定程度上提高了算法在不同数据集上的泛化性。我们同时预测图像子区域所估计运动的可靠性差异,用以避免对无特征区域的过度拟合,同时我们利用区域运动可靠性计算相机运动可靠性,为失效判断和多传感器融合提供基础。
    
    \item 提出了一种针对路面车辆的视觉里程计映射学习方法。我们根据路面车辆的特点,简化了单目视觉里程计问题,只学习车辆主要维度运动,同时研究旋转运动与平移运动之间的耦合关系,并利用此关系降低忽略次要运动时造成的运动偏移。此外,我们根据车辆运动时光流的分布提出使用非正方形的卷积核进行特征提取。最终,我们设计了一个十分轻量的视觉里程计运动估计模型,可以每秒200帧的速度实时运行在CPU上。
\end{enumerate}

本文提出的所有算法均在公开数据集上进行了充分的测试和验证,同时我们开源了本文算法中所涉及到的大部分代码。

\section{进一步研究方向}
本文工作在一定程度上推进了单目视觉里程计的尺度计算和映射学习问题,但依然存在着一定的局限性,存在进一步研究的意义,以下四个方面为可以进一步研究的方向:
\begin{enumerate}
    \item 研究泛化性更强的尺度计算方法。本文所提出的基于路面几何模型和场景深度估计的尺度计算方法均受到场景限制,虽然基于场景深度估计的方法摆脱了路面平整和相机高度假设,但其依然依赖于测试场景和训练集场景的相似性。可研究场景中绝对尺度参考的自主寻找、使用和更新方法,在具体实现上,可以按照如下思路:首先构建绝对尺度参考库,将场景中常见的物体尺度记录到数据库中;在机器人运行过程中实时对场景进行实例分割,在场景中自主地寻找相对稳定的绝对参考,并用以抑制尺度漂移;同时根据尺度的一致性,对数据库中的绝对尺度参考进行更新和修正。
    \item 研究纯虚拟数据集训练。这个研究可包含两方面工作:首先可不考虑数据的真实性,生成完全随机的RGB图像、深度图像以及随机的相机运动,用以训练构建的模型。由于完全随机的数据无穷无尽,所以可以不间断地产生新的数据一直训练网络,进而得到一个泛化性更强的网络,同时可结合迁移学习来提高算法在真实数据的性能;另外可研究生成仿真数据,即在生成数据时考虑其与现实场景的可区分性,使用生成对抗网络来生成更接近真实数据的虚拟数据用以训练网络。
    \item 研究动态场景下单目视觉状态估计。静态环境假设是视觉里程计一个比较重要的假设,目前大部分方法均没有对动态环境下的视觉里程计问题进行系统化的研究,较为主流的解决方案依然是动态物体剔除。本文提出一种潜在的研究思路:首先按照场景中物体的运动模式将物体运动模态分为静态、模式化运动和随机性运动,然后对三种运动模式采用不同的方案处理:保留静态物体,建模分析模式化运动物体,直接剔除随机性运动物体。在具体实现上,可根据单目图像上物体的识别和跟踪以及机器人自身运动的初始估计判断场景中物体的运动模式;然后依据物体运动模式先验信息以及物体运动的实时观测对模式化运动物体的运动建模,并联合估计自身运动与场景中物体运动。
    \item 研究光照恶劣环境下的单目视觉里程计问题。光照恶劣环境下视觉里程计效果的局限性存在三方面原因:首先照度低的环境会使图像信噪比较低,进而影响到特征提取和匹配的稳定性;其次光照空间不均匀致使的图像亮度不均会导致提取特征无法在图像上均匀地进行;最后光照时间维度上的变化会导致特征匹配或光度误差计算出现偏差,影响状态估计精度。现有解决方案一般为两段式方法:首先将图像变换到适合进行状态估计的图像空间或者特征空间,然后在新的空间进行位姿估计。在这种方案中,新的特征空间一般为认为设定的,具有局限性。可研究图像变换与单目状态估计深度融合,在前期所研究的端到端状态估计方法基础之上,将图像变换网络与状态估计的网络串联,将状态估计的误差反向传播给图像变换网络,用以自主寻找更适合做状态估计的图像空间或特征空间。
\end{enumerate}


%%% 其它部分
\backmatter

% 致谢
\begin{acknowledgement}
\input{data/ack}
\end{acknowledgement}

% 参考文献
\printTJbibliography

% 附录
%\begin{appendix}
%\chapter{路面车辆运动分布}
\label{ch:app_motion_dis}
为了证明第\ref{ch:datavo}章中图\ref{fig:rotation_corr_analysis}的代表性,我们对KITTI数据中其他全部序列做了同样的运动分析,可视化于图\ref{fig:mp_00_06}和图\ref{fig:mp_07_10}中。从图中可以发现,仅有KITTI数据集中的序列4与图\ref{fig:rotation_corr_analysis}分布不一致,其原因为序列4中车辆没有任何旋转。

\begin{figure}[h]
    \centering
        \begin{subfigure}[b]{0.48\textwidth}
            \includegraphics[width=\textwidth]{figures/appl/motion_variance_kitti00.pdf}
        \end{subfigure}
        \begin{subfigure}[b]{0.48\textwidth}
            \includegraphics[width=\textwidth]{figures/appl/motion_correlation__kitti00.pdf}
        \end{subfigure}
        \begin{subfigure}[b]{0.48\textwidth}
            \includegraphics[width=\textwidth]{figures/appl/motion_variance_kitti01.pdf}
        \end{subfigure}
        \begin{subfigure}[b]{0.48\textwidth}
            \includegraphics[width=\textwidth]{figures/appl/motion_correlation__kitti01.pdf}
        \end{subfigure}
        \begin{subfigure}[b]{0.48\textwidth}
            \includegraphics[width=\textwidth]{figures/appl/motion_variance_kitti02.pdf}
        \end{subfigure}
        \begin{subfigure}[b]{0.48\textwidth}
            \includegraphics[width=\textwidth]{figures/appl/motion_correlation__kitti02.pdf}
        \end{subfigure}
        \begin{subfigure}[b]{0.48\textwidth}
            \includegraphics[width=\textwidth]{figures/appl/motion_variance_kitti03.pdf}
        \end{subfigure}
        \begin{subfigure}[b]{0.48\textwidth}
            \includegraphics[width=\textwidth]{figures/appl/motion_correlation__kitti03.pdf}
        \end{subfigure}
        \begin{subfigure}[b]{0.48\textwidth}
            \includegraphics[width=\textwidth]{figures/appl/motion_variance_kitti04.pdf}
        \end{subfigure}
        \begin{subfigure}[b]{0.48\textwidth}
            \includegraphics[width=\textwidth]{figures/appl/motion_correlation__kitti04.pdf}
        \end{subfigure}
        \begin{subfigure}[b]{0.48\textwidth}
            \includegraphics[width=\textwidth]{figures/appl/motion_variance_kitti05.pdf}
        \end{subfigure}
        \begin{subfigure}[b]{0.48\textwidth}
            \includegraphics[width=\textwidth]{figures/appl/motion_correlation__kitti05.pdf}
        \end{subfigure}
        \begin{subfigure}[b]{0.48\textwidth}
            \includegraphics[width=\textwidth]{figures/appl/motion_variance_kitti06.pdf}
        \end{subfigure}
        \begin{subfigure}[b]{0.48\textwidth}
            \includegraphics[width=\textwidth]{figures/appl/motion_correlation__kitti06.pdf}
        \end{subfigure}
        \caption{运动分布:KITTI序列00-06}
        \label{fig:mp_00_06}
    \end{figure} 
    \begin{figure}[h]
        \centering
        \begin{subfigure}[b]{0.48\textwidth}
            \includegraphics[width=\textwidth]{figures/appl/motion_variance_kitti07.pdf}
        \end{subfigure}
        \begin{subfigure}[b]{0.48\textwidth}
            \includegraphics[width=\textwidth]{figures/appl/motion_correlation__kitti07.pdf}
        \end{subfigure}
        \begin{subfigure}[b]{0.48\textwidth}
            \includegraphics[width=\textwidth]{figures/appl/motion_variance_kitti08.pdf}
        \end{subfigure}
        \begin{subfigure}[b]{0.48\textwidth}
            \includegraphics[width=\textwidth]{figures/appl/motion_correlation__kitti08.pdf}
        \end{subfigure}
        \begin{subfigure}[b]{0.48\textwidth}
            \includegraphics[width=\textwidth]{figures/appl/motion_variance_kitti09.pdf}
        \end{subfigure}
        \begin{subfigure}[b]{0.48\textwidth}
            \includegraphics[width=\textwidth]{figures/appl/motion_correlation__kitti09.pdf}
        \end{subfigure}
        \begin{subfigure}[b]{0.48\textwidth}
            \includegraphics[width=\textwidth]{figures/appl/motion_variance_kitti10.pdf}
        \end{subfigure}
        \begin{subfigure}[b]{0.48\textwidth}
            \includegraphics[width=\textwidth]{figures/appl/motion_correlation__kitti10.pdf}
        \end{subfigure}
        \caption{运动分布:KITTI序列07-10}
        \label{fig:mp_07_10}
\end{figure}
%\chapter{运动聚焦后的轨迹重构}
\label{ch:appl_recon}
\begin{figure}[h]
    \centering
    \begin{subfigure}[b]{0.48\textwidth}
        \includegraphics[width=\textwidth]{figures/appl/recon_path_00.pdf}
    \end{subfigure}
    \begin{subfigure}[b]{0.48\textwidth}
        \includegraphics[width=\textwidth]{figures/appl/recon_path_01.pdf}
    \end{subfigure}
    \begin{subfigure}[b]{0.48\textwidth}
        \includegraphics[width=\textwidth]{figures/appl/recon_path_02.pdf}
    \end{subfigure}
    \begin{subfigure}[b]{0.48\textwidth}
        \includegraphics[width=\textwidth]{figures/appl/recon_path_03.pdf}
    \end{subfigure}
    \begin{subfigure}[b]{0.48\textwidth}
        \includegraphics[width=\textwidth]{figures/appl/recon_path_04.pdf}
    \end{subfigure}
    \begin{subfigure}[b]{0.48\textwidth}
        \includegraphics[width=\textwidth]{figures/appl/recon_path_05.pdf}
    \end{subfigure}
    \begin{subfigure}[b]{0.48\textwidth}
        \includegraphics[width=\textwidth]{figures/appl/recon_path_06.pdf}
    \end{subfigure}
    \begin{subfigure}[b]{0.48\textwidth}
        \includegraphics[width=\textwidth]{figures/appl/recon_path_07.pdf}
    \end{subfigure}
    \begin{subfigure}[b]{0.48\textwidth}
        \includegraphics[width=\textwidth]{figures/appl/recon_path_08.pdf}
    \end{subfigure}
    \begin{subfigure}[b]{0.48\textwidth}
        \includegraphics[width=\textwidth]{figures/appl/recon_path_09.pdf}
    \end{subfigure}
    \caption{重构轨迹00-10}
    \label{fig:recon_00_10}
\end{figure}
%\chapter{术语解释}
本章我们将对论文中涉及到的术语进行简要解释
\begin{enumerate}
    \item 特征点法视觉里程计 (Feature-based Vidual Odometry)
    \item 直接法视觉里程计 (Direct Visual Odometry)
    \item 灭点(Vanishing Point)
    \item 条件随机场 (Conditional Random Field)
\end{enumerate}
%\end{appendix}

% 个人简历
\begin{resume}
\resumeitem{个人简历}
\noindent 1992 年 01 月 9 日出生于 安徽 省 宿州 市 砀山 县。\\
\noindent 2011 年 9 月考入 同安徽济 大学 电气工程及其自动化学院 自动化 专业,2015 年 7 月本科毕业并获得 工学 学士学位,并获得安徽大学优秀毕业生称号\\
\noindent 2015 年 9 月免试进入 同济 大学 控制科学与工程 系攻读 博士 学位至今。\\
\noindent 2017 年 9 月 - 2018年3月 受香港科技大学刘明教授和陈启军导师共同资助 去香港科技大学工学院机器人与多感知实验室(RAM-LAB)做 访问学者。

\resumeitem{发表论文:} 
\begin{enumerate}[{[}1{]}]
\item \textbf{Wang, X.}, Maturana, D., Yang, S., Wang, W., Chen, Q., and Scherer, S. (2019, November).
 Improving learning-based ego-motion estimation with homomorphism-based losses and drift correction. 
 In 2019 IEEE/RSJ International Conference on Intelligent Robots and Systems (IROS) (pp. 970-976). IEEE. (机器人领域顶级会议 CCF C类)
\item \textbf{Wang, X.}, Zhang, H., Yin, X., Du, M., and Chen, Q. (2018, May). 
Monocular visual odometry scale recovery using geometrical constraint. 
In 2018 IEEE International Conference on Robotics and Automation (ICRA) (pp. 988-995). IEEE. (机器人领域顶级会议 CCF B类)
\item Yin, $X^*$., {\bf Wang, $X^*$.}, Du, X., and Chen, Q. 
Scale Correction for Monocular Visual Odemetry Using Depth Estimated with Deep Convolutional Neural Fields,
\emph{Internation Conference on Computer Vision} 2017 (*Both authors contributed equally to this paper.)
 (计算机视觉顶级会议 CCF A类)
\item \textbf{Wang, X.}, and Chen, Q. (2015, August).
 Vision-based entity Chinese chess playing robot design and realization. 
 In International Conference on Intelligent Robotics and Applications (pp. 341-351). Springer, Cham. (EI)
 \item Zhang, H., \textbf{Wang, X.}, Du, X., Liu, M., and Chen, Q. (2017, July). 
 Dynamic environments localization via dimensions reduction of deep learning features.
  In International Conference on Computer Vision Systems (pp. 239-253). Springer, Cham. (EI)
\item Mingxiao, D., Xiaofeng, M., Zhe, Z., \textbf{Xiangwei, W.}, and Qijun, C. (2017, October).
 A review on consensus algorithm of blockchain. 
 In 2017 IEEE International Conference on Systems, Man, and Cybernetics (SMC) (pp. 2567-2572). IEEE. (EI)
\end{enumerate}

\resumeitem{已投稿论文:}
\begin{enumerate}[{[}1{]}]
  \item Zhang, H*.,  Wang, X*., Yin, X., Du, M., Liu, C., and Chen, Q., Geometric Constrained Scale Estimation for Monocular Visual Odometry.
  Submited to IEEE Transaction on Multimedia. 2020 (*Both authors contributed equally to this paper.)
  %\item PADVO: Patch Agreement Deep Visual Odometry, Submited to Robotics and Automation Letter
\end{enumerate}
\resumeitem{已授权专利:} % 有就写,没有就删除
\begin{enumerate}[{[}1{]}]
\item 一种基于三角剖分的单目视觉里程计尺度恢复方法 - 201710346708.6 发明专利,导师外第一发明人,已授权 
\end{enumerate}
\resumeitem{已公开专利:} % 有就写,没有就删除
\begin{enumerate}
  \item 一种基于图像特征降维的无人车单目视觉定位方法 - 201710333483.0 ,发明专利,导师外第三发明人,公开实质审查中,2017年
  \item 一种无人车单目视觉定位中对匹配矩阵的图像匹配方法 - 201710333485.X,发明专利,导师外第三发明人,公开实质审查中,2017年
\end{enumerate}

%\resumeitem{竞赛获奖} % 有就写,没有就删除
%\begin{enumerate}%[{[}1{]}]
%\item 全国研究所数学建模比赛 全国二等奖 2015年 
%\end{enumerate}


\end{resume}

\end{document}